\chapter{Cartesian systems}

\section{The graph of cartesian Selmer structures}

\begin{definition}
We consider the graph $\CART$ whose vertices are the cartesian Selmer structures on $T$ and there is an arrow $\GG\to \FF$ joining two Selmer structures $\FF$ and $\GG$ whenever $\FF\leq \GG$.
\end{definition}

\begin{proposition}
Let $\FF\leq\GG$ be cartesian Selmer structures. Then the module
\[\LL_{\GG/\FF}=\bigoplus_{\ku\in \UU(\Pb)} \bH^1_{\GG/\FF}(K_\ku,T)=\bigoplus_{\ku\in \UU(\Pb)} \frac{\bH^1_{\GG}(K_\ku,T)}{\bH^1_{\FF}(K_\ku,T)}\]
is a free, finitely generated $R$-module of rank $\chi(\GG)-\chi(\FF)$.
\end{proposition}

\begin{proof}
Since $\GG$ is cartesian, the $R$-module
\[\bigoplus_{\ku\in \UU(\Pb)}\bH^1_{/\GG}(K_\ku,T)\] 
is torsion-free. Since $\LL_{\GG/\FF}$ is a finitely-generated submodule, it is free by the structure theorem. By Proposition \ref{prop:patched:global_duality}, there is an exact sequence
\[\xymatrix{H^1_{\FF}(K,T)\ar@{>->}[r] & H^1_{\GG}(K,T)\ar[r] & \LL_{\GG/\FF} \ar[r] & H^1_{\FF^*}(K,T^*)^\vee\ar@{->>}[r]& H^1_{\GG^*}(K,T^*)^\vee}\]
By Proposition \ref{prop:patched:core_rank_limit}, we see that the rank of $\LL_{\GG/\FF}$ is $\chi(\GG)-\chi(\FF)$.
\end{proof}

We want to extend the definition of $\LL_{\GG/\FF}$ for every pair of Selmer structures $\FF$ and $\GG$, not necessarily comparable.
\begin{definition}
Let $\FF$ and $\GG$ be cartesian Selmer structures. We define the \emph{local quotient} as
\[\LL_{\GG/\FF}:=\bigoplus_{\ku\in \Sigma_{\FF< \GG}} \frac{\bH^1_{\GG}(K_\ku,T)}{\bH^1_{\FF}(K_\ku,T)}\oplus \bigoplus_{\ku\in \Sigma_{\GG<\FF}} \left(\frac{\bH^1_{\FF}(K_\ku,T)}{\bH^1_{\GG}(K_\ku,T)}\right)^+\]
where $\Sigma_{\FF<\GG}$ (resp. $\Sigma_{\GG<\FF}$) is the set of ultraprimes $\ku\in \Sigma_{\FF}\cup\Sigma_{\GG}$ such that $\bH^1_{\FF}(K_\ku,T)\subset \bH^1_{\GG}(K_\ku,T)$ (resp. $\bH^1_{\GG}(K_\ku,T)\subset \bH^1_{\FF}(K_\ku,T)$).
\label{def:local_quotient}
\end{definition}

\begin{proposition}
Let $\FF\leq \GG$ be cartesian Selmer structures. There is a canonical homomorphism
\[\phi_{\GG,\FF}:\ \bigcap^{\chi(\GG)} \bH^1_{\GG}(K,T)\otimes \det(\LL^+_{\GG/\FF})\to \bigcap^{\chi(\FF)} \bH^1_{\FF}(K,T)\]
\label{prop:cartesian_bidual_map}
\end{proposition}

\begin{proof}
Since $\LL_{\GG/\FF}$ is a free $R$-module of rank $\chi(\GG)-\chi(\FF)$, Proposition \ref{prop:bidual_map_dvr_free} constructs the map $\phi_{\GG,\FF}$.
\end{proof}

\begin{definition}
Fix a base cartesian Selmer structure $\FF_0$. For every cartesian Selmer structure $\FF$, define
\[\bX(\FF)=\bX_{\FF_0}(\FF)=\bigcap^{\chi(\FF)} \bH^1_{\FF}(K,T)\otimes \det(\LL^+_{\FF/\FF_0})\]
\end{definition}

\begin{proposition}
    Let $\FF\leq \GG$ be two cartesian Selmer structures. Then there is a map
    \[\phi_{\GG/\FF}:\ \bX_{\FF_0}(\GG)\to \bX_{\FF_0}(\FF)\]
    \label{prop:cartesian:map}
\end{proposition}

In order to prove this result, we need the following lemma.
\begin{lemma}
Let $\FF$, $\GG$ and $\FF_0$ be cartesian Selmer structures such that $\FF\leq \GG$. Then there is a canonical isomorphism
\[\det(\LL_{\GG/\FF_0}^+)=\det(\LL_{\GG/\FF}^+)\otimes \det(\LL_{\FF/\FF_0}^+)\]
\label{lem:cartesian:det_split}
\end{lemma}

\begin{proof}
We can split the determinant of $\LL_{\GG/\FF}^+$ as
\[\det\Bigl(\LL_{\GG/\FF_0}^+\Bigr)=\det\left(\bigoplus_{\ku\in \Sigma_{\FF_0< \GG}} \left(\frac{\bH^1_{\GG}(K_\ku,T)}{\bH^1_{\FF_0}(K_\ku,T)}\right)^+\right)\otimes  \det\left(\bigoplus_{\ku\in \Sigma_{\GG<\FF_0}} \frac{\bH^1_{\FF_0}(K_\ku,T)}{\bH^1_{\GG}(K_\ku,T)}\right)\]
Here, all the summands are free $R$-modules, so we can identify them with their biduals. Since $\FF\leq \GG$, there is a partition $\Sigma_{\FF_0<\GG}=\Sigma_{\FF\leq \FF_0<\GG}\sqcup \Sigma_{\FF_0< \FF\leq\GG}$
Hence we can use Propositions \ref{prop:det_quotient} and \ref{prop:det_dual} to split the first determinant as
\[\begin{aligned}
&\det\left(\bigoplus_{\ku\in \Sigma_{\FF_0< \GG}} \left(\frac{\bH^1_{\GG}(K_\ku,T)}{\bH^1_{\FF_0}(K_\ku,T)}\right)^+\right)=\\
&\det\left(\bigoplus_{\ku\in\Sigma_{\FF\leq \FF_0<\GG}} \left(\frac{\bH^1_{\GG}(K_\ku,T)}{\bH^1_{\FF}(K_\ku,T)}\right)^+\right)\otimes\det\left(\bigoplus_{\ku\in\Sigma_{\FF\leq \FF_0<\GG}} \frac{\bH^1_{\FF_0}(K_\ku,T)}{\bH^1_{\FF}(K_\ku,T)}\right)\otimes\\
&\det\left(\bigoplus_{\ku\in\Sigma_{\FF_0< \FF\leq\GG}} \left(\frac{\bH^1_{\GG}(K_\ku,T)}{\bH^1_{\FF}(K_\ku,T)}\right)^+\right)\otimes\det\left(\bigoplus_{\ku\in\Sigma_{\FF_0< \FF\leq\GG}} \left(\frac{\bH^1_{\FF}(K_\ku,T)}{\bH^1_{\FF_0}(K_\ku,T)}\right)^+\right)\\
\end{aligned}\]
Since $\FF\leq \GG$, the sets $\Sigma_{\GG<\FF_0}$ and $\Sigma_{\FF\leq \GG<\FF_0}\sqcup \Sigma_{\GG< \FF\leq\FF_0}$ coincide. Similarly, we can use Propositions \ref{prop:det_quotient} and \ref{prop:det_dual} to split the determinant as
\[\begin{aligned}
&\det\left(\bigoplus_{\ku\in \Sigma_{\GG<\FF_0}} \frac{\bH^1_{\FF_0}(K_\ku,T)}{\bH^1_{\GG}(K_\ku,T)}\right)=\\
&\det\left(\bigoplus_{\ku\in\Sigma_{\FF\leq \GG<\FF_0}} \frac{\bH^1_{\FF_0}(K_\ku,T)}{\bH^1_{\FF}(K_\ku,T)}\right)\otimes\det\left(\bigoplus_{\ku\in\Sigma_{\FF\leq \GG<\FF_0}} \left(\frac{\bH^1_{\GG}(K_\ku,T)}{\bH^1_{\FF}(K_\ku,T)}\right)^+\right)\\
\end{aligned}\]
By grouping the different terms, we obtain that 
\[\det(\LL_{\GG/\FF_0}^+)=\det(\LL_{\GG/\FF}^+)\otimes \det(\LL_{\FF/\FF_0}^+)\]

\textcolor{red}{revise sign conventions with direct sums}
\end{proof}

\begin{proof}[Proof of Proposition \ref{prop:cartesian:map}]
By Lemma \ref{lem:cartesian:det_split}, tensoring the map from Proposition \ref{prop:cartesian_bidual_map} with the identity on $\det(\LL_{\FF/\FF_0}^+)$, we obtain the desired map $\phi_{\GG/\FF}$.
\end{proof}

\begin{proposition}
The assignment $\FF\to \bX_{\FF_0}(\FF)$ forms an inverse system indexed by the set of cartesian Selmer structure.
\label{prop:cartesian:inverse_system}
\end{proposition}

\begin{proof}
\textcolor{red}{do}
\end{proof}

\begin{definition}
We define the set of cartesian systems as the elements in the inverse limit
\[\CART_{\FF_0}=\varprojlim_\FF \bX_{\FF_0}(\FF)\]
where the limit is taken over all cartesian Selmer structures.
\label{def:cartesian_system}
\end{definition}

\subsection{Core cartesian Selmer structures}

\begin{definition}
A cartesian Selmer structure is called a \emph{core structure} when 
\[\bH^1_{\FF^*}(K,T)=0\]
\end{definition}

\begin{proposition}
Under Assumptions \ref{ass:patched:basic}, for every cartesian Selmer structure $\FF$, there is core structure $G$ such that $\FF\leq \GG$.
\label{prop:cartesian:core_upper}
\end{proposition}

\begin{proof}
    \textcolor{red}{complete}
\end{proof}

\begin{proposition}
Let $\FF\leq \GG$ be two core Selmer structures. Then the map $\bX_{\FF_0}(\GG)\to \bX_{\FF_0}(\FF)$ is an isomorphism.
\label{prop:cartesian:core_iso}
\end{proposition}

\begin{proof}
\textcolor{red}{complete}
\end{proof}

Similarly to what happened with Kolyvagin and Stark systems, cartesian systems can be controlled by core Selmer structures.
\begin{proposition}
Let $\FF$ be a core Selmer structure. Then the map
\[\CART_{\FF_0}\to \bX_{\FF_0}(\FF)\]
\label{prop:cartesian_core}
\end{proposition}

\begin{proof}
Let $c\in \bX_{\FF_0}(\FF)$. We will show that there is a unique $\varepsilon\in \CART$ such that $\varepsilon_{\FF}=c$.

Let $\GG$ be a cartesian structure. By Proposition \ref{prop:cartesian:core_upper}, there is a core Selmer structure $\HH$ such that $\FF\leq \HH$ and $\GG\leq \HH$. Since the maps $\phi_{\HH,\FF}:\bX(\HH)\to \bX(\FF)$ is an isomorphism by Proposition \ref{prop:cartesian:core_iso}, the definition of the inverse limit implies that $\varepsilon_\GG$ is the image of the $\varepsilon_{\FF}$ under the map
\[\phi_{\HH,\GG}\circ \phi_{\HH,\FF}^{-1}:\ \bX(\FF)\to \bX(\GG):\ \varepsilon_{\FF}\to \varepsilon_{\GG}\]
Hence, the element is determined by $c$. Morevoer, this definition clearly defines an element of $\CART$, so the projection map is an isomorphism.
\end{proof}

We can now prove that the module of cartesian systems is free of rank one over $R$.

\begin{proposition}
Let $\FF$ be a core Selmer structure. Then $\bX_{\FF_0}(\FF)$ is a free, cyclic $R$-module.
\end{proposition}

\begin{proof}

\end{proof}

\begin{corollary}
The module of cartesian systems $\CART_{\FF_0}$ is a free, cyclic $R$-module.
\end{corollary}

\section{Partial cartesian systems}

\section{Higher dimensional Kolyvagin systems}

In this section, we assume the weaker version of Assumption \ref{ass:basic}.

\begin{assumption}
We assume the following assumptions:
\begin{itemize}
\item \namedlabel{nsTirred}{(T1)} $T/\m T$ is an irreducible $k[[G_K]]$-module.
\item \namedlabel{nsTcoh}{(T3)} $H^1(K(T)_M/K,T)=H^1(K(T)_M/K,T^*)=0$.
\item\namedlabel{Nsd}{(N1)} $T/\m T$ is not isomorphic to $T^*[\m]$ as $k[[G_K]]$-modules.
\item\namedlabel{Nsur}{(N2)} The image of the homomorphism $R\to \textrm{End}(T)$ is contained in the image of $\Z_p[[G_\Q]]\to \textrm{End}(T)$.
\end{itemize}
\label{ass:nonsur}
\end{assumption}

We are not assuming the existence of $\tau$ such that $T/(\tau-1)T$ is a free, cyclic $R$-module. However, there always exist $\tau\in G_K$ such that $T/(\tau-1)T$ is a free $R$-module of rank $d$.

\begin{notation}
Fix some $\tau\in G_{K_M}$ such that $T/(\tau-1)T\cong R^d$, minimizing the exponent $d$.
\end{notation}

Similarly to Definition \ref{def:kolyvagin_primes}, we define the Kolyvagin primes $\ell\in \PP_\tau$ as those not belonging to $\Sigma_\FF$ and whose Frobenius automorphism is conjugate to $\tau$ in $\Gal(K(T)_M/K)$. We also denote by $\NN_\tau$ (resp. by $\NN_\tau^i$) to the set of square-free products of Kolyvagin primes (resp. square-free products of exactly $i$-primes).

Let $\ell\in \PP_\tau$. Since $\Frob_\ell$ is trivial in $\Gal(K_M/K)$, then \textcolor{red}{comment} there is an splitting
\[H^1(K_\ell,T)=H^1_\f(K_\ell,T)\oplus H^1_\tr(K_\ell,T)\]
which is a free $R$ module of rank $d$.

Moreover, there is canonical isomorphisms
\[H^1_\f(K_\ku, T)\cong T/(\Frob_\ku-1) T=T/(\tau-1) T\]

After fixing a generator of $\GG_\ell$, there is another canonical ismorphisms.
\[\phi_\ku^{\fs}:\ H^1_\f(K_\ku, T)\to H^1_\tr(K_\ku, T)\]

\begin{definition}
An element $\kn\in \NN$ is said to be \emph{admissible} if, for every $\ku\mid \kn$, the map 
\[\loc_\ku:\ \bH^1_{\FF^{\kn/\ku}}(K,T)\to \bH^1_\f(K_\ku,T)\]
Denote by $\AA$ the set of admissible $n\in \NN$. For every admissible $\kn\in \AA$, denote by $A_{\kn,\ku}$ the minimal indivisible free $R$-subgroup of $\bH^1_\f(K_\ku,T)$ containing $\Im(\loc_\ku)$.
\end{definition}

The next proposition gives a method to construct admissible products.
\begin{proposition}
Let $\ku_1,\ldots, \ku_s\in \PP$ be a set ok Kolyvagin ultraprimes satisfying that for all $i=1,\ldots, s$, 
\begin{itemize}
    \item The image of $\loc_{\ku_i}:\bH^1_{\FF^{\ku_1\cdots\ku_{i-1}}}(K,T)\to \bH^1_{\f}(K_{\ku_i},T)$ is contained in a free rank $1$ $R$-submodule.
    \item For every $j\in \{1,\ldots, i-1\}$, we have that
    \[\bH^1_{(\FF^*)_{(\ku_1\cdots\ku_{i})/\ku_j}^{\ku_j}}(K,T^*)+\bH^1_{(\FF^*)_{(\ku_1\cdots\ku_{i-1})/\ku_j}}(K,T^*)=\bH^1_{(\FF^*)^{\ku_j}_{(\ku_1\cdots\ku_{i-1})/\ku_j}}(K,T^*)\]
    \[\textcolor{blue}{H^1_{(\FF^*)^{\ell_1\ldots\ell_{i-1}}_{\ell_i}}(K,T^*)+ H^1_{\FF^*_{\ell_1\cdots\ell_{i-1}}}(K,T^*)=H^1_{(\FF^*)^{\ell_1\ldots\ell_{i-1}}}(K,T^*)     }\]
\end{itemize}
Then $\kn:=\ku_1\cdots\ku_s$ is admissible.
\end{proposition}

\begin{proof}
Consider the exact sequence
\[\xymatrix{\bH^1_{\FF^{\kn/\ku_i}_{\ku_i}}(K,T)\ar@{>->}[r] &  \bH^1_{\FF^{\kn/\ku_i}}(K,T)\ar[r] & \bH^1_\f(K_{\ku_i},T) \ar[r] & \bH^1_{(\FF^*)_{\kn/\ku_i}^{\ku_i}}(K,T^*)^\vee \ar@{->>}[r] &\bH^1_{(\FF^*)_{\kn/\ku_i}}(K,T^*)^\vee}\]

Denote $r_i:=\ell_1\cdots \ell_{i-1}$ and $s_i:=\ell_{i+1}\cdots \ell_s$.\textcolor{red}{write}
\end{proof}

\section{Kolyvagin systems}

Definition \ref{def:kol} can be generalised to this setting with the only modification of the original Kolyvagin primes in $\PP$ to rank $d$ Kolyvagin primes in $\PP_\tau$.

\begin{definition}
A \emph{$\tau$-Kolyvagin system} for a Selmer structure $\FF$
$$\kappa=\left\{\kappa_\kn\in H^1_{\mathcal F(\kn)}(K,T):\ \kn\in \AA_\tau\right\}$$
satisfying the following relation for every $\kn\in \AA_\tau$ and $\ku\in \PP_\tau$ not dividing $\kn$. By the definition of Selmer module, we have that 
\[\begin{array}{cc}
\loc_\ku(\kappa_\kn)\in H^1_{\FF(\kn)}(K_\ku,T)=H^1_\f(K_\ku,T),\ &\loc_\ell(\kappa_{\kn\ku})\in H^1_{\FF(\kn\ku)}(K_\ku,T)=H^1_\tr(K_\ku,T)
\end{array}\]
The collection $\kappa$ is a Kolyvagin system if the following is satisfied
\begin{equation}
\loc_\ku(\kappa_{\kn\ku})=\phi_\ku^{\fs}\circ \loc_\ku(\kappa_\kn)
\label{eq:kol_cond}
\end{equation}
for every $\kn\in \AA_\tau$ and $\ku\in \PP_\tau$ not dividing $n$.
\label{def:kol_tau}
\end{definition}

We will aim to construct a partial cartesian system from the classes $\kappa_n$, where $\kappa$ is a $\tau$-Kolyvagin systems and $n\in \NN_\tau$ is admissible.

Recall that, when $n\in\AA_\tau$ is admissible, we had defined a free, cyclic $R$-submodule $A_{\kn,\ku}$ such that 
\[\loc_\ku\Bigl(H^1_{\FF^{\kn/\ku}}(K,T)\Bigr)\subset A_{\kn,\ku}\subset H^1_{\f}(K,T)\]

\begin{proposition}
Assume that $\bH^1_{\FF}(K,T)\neq \bH^1_{\FF_\ku}(K,T)$. Then $A_{\kn,\ku}$ coincides for all admissible $\kn$ dividing $\ku$. 
\end{proposition}

Choose a free $R$-module $B_{n,\ku}$ of rank $d-1$ such that 
\[A_{\kn,\ku}\oplus B_{\kn,\ku}=H^1_{\f}(K,T)\]

We can define, for every admissible $\kn\in \NN$, a Selmer structure $\FF[\kn]$.

\begin{definition}
Let $\kn\in \NN$ be admissible. Define the Selmer structure $\FF[\kn]$ by the local conditions 
\[\left\{\begin{aligned}
&\bH^1_{\FF[n]}(K_\ku,T)=B_n\oplus \phi_\ku^\fs(A_n)&\textrm{ if }\ku\mid \kn\\
&\bH^1_{\FF[n]}(K_\ku,T)=\bH^1_{\FF}(K_\ku,T)&\textrm{ if }\ku\nmid \kn\\
\end{aligned}\right.\]
\end{definition}

\begin{proposition}
Let $\kappa\in \KS_\tau(\FF)$. For every admissible $n\in \NN_\tau$, $\kappa_n\in \bH^1_{\FF[n]}(K,T)$.
\end{proposition}

\begin{proof}
For every $\ku\nmid\kn$, then
\[\loc_\ku(\kappa_n)\in \bH^1_{\FF(n)}(K,T)=\bH^1_{\FF[n]}(K,T)\] 
Alternatively, when $\ku\mid \kn$, we have that 
\[\phi_\ku^\fs\circ \loc_\ku(\kappa_{\kn/\ku})=\loc_\ku(\kappa_kn)\]
Since $\kn$ is admissible, the definition of $A_{\kn,\ku}$ implies that 
\[\loc_\ku(\kappa_{\kn/ku})\in \loc_\ku\Bigl(\bH^1_{\FF(\kn/\ku)}(K,T)\Bigr)\subset \loc_\ku\Bigl(\bH^1_{\FF^{\kn/\ku}}(K,T)\Bigr)\subset A_n\]
Therefore,
\[\loc_\ku(\kappa_{\kn})\in \phi_\ku^{\fs}(A_\kn)\subset \bH^1_{\FF[n]}(K,T)\]
\end{proof}

We can also define partially relaxed and restricted Selmer structures at admissible elements.

\begin{definition}
Let $\ka,\kb,\kn\in \AA$ be admissible elements such that $\ka\mid \kn$, $\kb\mid \kn$ and $\ka$ and $\kb$ are pairwise coprime. The Selmer structure $\FF[\kn_\ka^\kb]$ is defined by the local conditions
\[\left\{\begin{aligned}
&\bH^1_{\FF[n_\ka^\kb]}(K_\ku,T)=B_n&\textrm{ if }\ku\mid \ka\\
&\bH^1_{\FF[n_\ka^\kb]}(K_\ku,T)=\bH^1_\f(K_\ku,T)\oplus \phi_\ku^\fs(A_n)&\textrm{ if }\ku\mid \kb\\
&\bH^1_{\FF[n_\ka^\kb]}(K_\ku,T)=B_n\oplus \phi_\ku^\fs(A_n)&\textrm{ if }\ku\mid \kn/\ka\kb\\
&\bH^1_{\FF[n_\ka^\kb]}(K_\ku,T)=\bH^1_{\FF}(K_\ku,T)&\textrm{ if }\ku\nmid n\\
\end{aligned}\right.\]
\end{definition}












