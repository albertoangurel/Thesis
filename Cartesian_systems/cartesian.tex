\chapter{Cartesian systems}

\section{Cartesian Selmer structures}

This section gives a characterization of cartesian Selmer structures over principal rings, both artinian rings and discrete valuation rings. The definition of cartesian structure is more complicated when $R$ is artinian (see Definition \ref{def:cartesian}) than when $R$ is a discrete valuation ring (\ref{def:patched_cartesian}). However, we will see that both definitions are essentially equivalent.

The first result is a generalisation of Proposition \ref{prop:cartesian_quotient_rk1}.

\begin{proposition}
Let $\FF\leq\GG$ be cartesian Selmer structures. Then
\[\bigoplus_{\ell\in \Sigma_{\FF}\cup \Sigma_{\GG}} H^1_{\GG/\FF}(K,T)\]
is a free $R$-module of rank $\chi(\GG)-\chi(\FF)$.
\label{prop:cartesian_quotient}
\end{proposition}

\begin{proof}

By Proposition \ref{prop:global_duality}, there is an global-duality exact sequence for $\Tbar:=T\otimes k$
\[\xymatrix{H^1_{\FF}(K,\Tbar)\ar@{>->}[r] & H^1_{\GG}(K,\Tbar)\ar[r] & \displaystyle{\bigoplus_{q\in \Sigma_\FF\cup\Sigma_\GG} \frac{H^1_{\GG}(K_q,\Tbar)}{H^1_{\FF}(K_q,\Tbar)}} \ar[r] & H^1_\GG(K,\Tbar^*)^\vee\ar@{->>}[r]& H^1_\FF(K,\Tbar^*)^\vee}\]

Definition \ref{def:core_rank} and dimension counting implies that 
\[\dim_k\left(\displaystyle{\bigoplus_{q\in \Sigma_\FF\cup\Sigma_\GG} \frac{H^1_{\GG}(K_q,\Tbar)}{H^1_{\FF}(K_q,\Tbar)}}\right)=\chi(\GG)-\chi(\FF)\]
\cite[Lemma 1.1.5]{MazurRubin} says that for every pair of cartesian Selmer structures $\FF$ and $\GG$, the quantity
\begin{equation}
\sum_{\ell\in \Sigma_{\FF}\cup \Sigma_{\GG}}\biggl(\length\Bigl(H^1_{\GG}(K_\ell,T\otimes R/\m^i)\Bigr)-\length\Bigl(H^1_{\FF}(K_\ell,T\otimes R/\m^i)\Bigr)\biggr)
\label{eq:length_dif_hr}
\end{equation}
is linearly dependent on $i$. By the computation for $i=1$, we can conclude that
\[\length\bigl(H^1_{\GG/\FF}(K,T)\Bigr)=\length(R)\bigl(\chi(\GG)-\chi(\FF)\bigr)\]



Consider the following composition, which coincides with the multiplication by $\pi^{k-1}$.
\[\xymatrix{ \displaystyle\sum_{\ell\in \Sigma_{\FF}\cup \Sigma_{\GG}}H^1_{\GG/\FF}(K_\ell,T)\ar[r] & \displaystyle\sum_{\ell\in \Sigma_{\FF}\cup \Sigma_{\GG}}H^1_{\GG/\FF}(K_\ell,T\otimes k) \ar[r]^{\pi^{k-1}} & \displaystyle\sum_{\ell\in \Sigma_{\FF}\cup \Sigma_{\GG}} H^1_{\GG/\FF}(K_\ell,T)}\]
The first map is surjective by the propagation of Selmer structures and the second one is injective since $\FF$ is cartesian. By the induction hypothesis, $H^1_{\GG/\FF}(K_\ell,T)$ is an $R$-module whose length is $\length(R)\bigl(\chi(\GG)-\chi(\FF)\bigr)$ and whose $\m^{k-1}$-torsion induces a non-trivial quotient of length $\chi(\GG)-\chi(\FF)$. Hence, the structure theorem implies that 
\[H^1_{\GG/\FF}(K_\ell,T)\cong R^{\chi(\GG)-\chi(\FF)}\qedhere\]
\end{proof}

Conversely, we will prove that we can obtain cartesian Selmer structures by \enquote{adding} or \enquote{removing} free copies of $R$.

\begin{proposition}
Let $R$ be a principal, artinian, local ring and let $\GG$ be a cartesian Selmer structure. Consider another Selmer structure $\FF\leq \GG$ such that there is a prime $\ell$ such that
\[H^1_{\GG}(K_q,T)=H^1_{\FF}(K_q,T)\ \forall q\neq \ell\]
Assume further that 
\[H^1_{\GG/\FF}(K_\ell,T)\cong R\]
Then, $\FF$ is also cartesian.
\label{prop:cartesian_artinian_remove}
\end{proposition}

\begin{proof}
Since all other local conditions coincice, we only need to check the cartesian property at the local condition at $\ell$. Since $R$ is a projective $R$-module, there is a decomposition 
\[H^1_{\GG}(K_\ell,T)=H^1_{\FF}(K_\ell, T)\oplus R\]
We will see that the image of the free module under the projection
\[H^1(K_\ell, T)\to H^1(K_\ell,T\otimes k)\]
is a one-dimensional $k$-vector space. Indeed, the dimension of the image is bounded above by one because $R$ is generated by one element. It is also non-zero because the kernel is contained in $H^1(K_\ell, T)[\m^{k-1}]$, since the following composition is the multiplication by $\pi^{k-1}$, where $\pi$ is a generator of $\m$:
\[\xymatrix{H^1(K_\ell,T)\ar[r] & H^1(K_\ell,T\otimes k) \ar[r]^{\pi^{k-1}} & H^1(K_\ell,T)}\]
Therefore,
\[H^1_{\GG}(K_\ell,\Tbar)=H^1_{\FF}(K_\ell, \Tbar)\oplus k\]
Since $R$ is artinian, the structure theorem gives decompositions,
\[\begin{aligned}
&H^1_{/\FF}(K_\ell,T)=H^1_{/\GG}(K_\ell, T)\oplus R\\
&H^1_{/\FF}(K_\ell,\Tbar)=H^1_{/\GG}(K_\ell, \Tbar)\oplus k
\end{aligned}\]

The map 
\[H^1_{/\FF}(K_\ell, \Tbar)\to H^1_{/\FF}(K_\ell, T)\]
respects the above decomposition, so it is injective because $\GG$ is cartesian. Hence $\FF$ is also cartesian.
\end{proof}

\begin{proposition}
Let $R$ be a principal, artinian, local ring and let $\FF$ be a cartesian Selmer structure. Consider another Selmer structure $\GG\geq \FF$ such that there is a prime $\ell$ such that
\[H^1_{\GG}(K_q,T)=H^1_{\FF}(K_q,T)\ \forall q\neq \ell\]
Assume further that 
\[H^1_{\GG/\FF}(K_\ell,T)\cong R\]
Then, $\GG$ is also cartesian.
\label{prop:cartesian_artinian_add}
\end{proposition}

\begin{proof}
Analogously to the proof of Proposition \ref{prop:cartesian_artinian_remove}, there are decompositions 
\[\begin{aligned}
&H^1_{/\FF}(K_\ell,T)=H^1_{/\GG}(K_\ell,T)\oplus R\\
&H^1_{/\FF}(K_\ell,T)=H^1_{/\GG}(K_\ell,T)\oplus k\\
\end{aligned}\]
Since the map in Definition \ref{def:cartesian} respects this decomposition, the cartesian property of $\FF$ implies that the map
\[H^1_{/\GG}(K_\ell, \Tbar)\to H^1_{/\GG}(K_\ell, T)\]
is injective, so $\GG$ is also cartesian.
\end{proof}

\textcolor{red}{do it with R being DVR}

\section{The graph of cartesian Selmer structures}

\begin{definition}
We consider the directed graph whose vertices are the cartesian Selmer structures on $T$ and there is an arrow $\GG\to \FF$ joining two Selmer structures $\FF$ and $\GG$ whenever $\FF\leq \GG$.
\label{def:cartesian:graph}
\end{definition}

\begin{proposition}
Let $\FF\leq\GG$ be cartesian Selmer structures. Then the module
\[\LL_{\GG/\FF}=\bigoplus_{\ku\in \UU(\Pb)} \bH^1_{\GG/\FF}(K_\ku,T)=\bigoplus_{\ku\in \UU(\Pb)} \frac{\bH^1_{\GG}(K_\ku,T)}{\bH^1_{\FF}(K_\ku,T)}\]
is a free, finitely generated $R$-module of rank $\chi(\GG)-\chi(\FF)$.
\end{proposition}

\begin{proof}
Since $\GG$ is cartesian, the $R$-module
\[\bigoplus_{\ku\in \UU(\Pb)}\bH^1_{/\GG}(K_\ku,T)\] 
is torsion-free. Since $\LL_{\GG/\FF}$ is a finitely-generated submodule, it is free by the structure theorem. By Proposition \ref{prop:patched:global_duality}, there is an exact sequence
\[\xymatrix{H^1_{\FF}(K,T)\ar@{>->}[r] & H^1_{\GG}(K,T)\ar[r] & \LL_{\GG/\FF} \ar[r] & H^1_{\FF^*}(K,T^*)^\vee\ar@{->>}[r]& H^1_{\GG^*}(K,T^*)^\vee}\]
By Proposition \ref{prop:patched:core_rank_limit}, we see that the rank of $\LL_{\GG/\FF}$ is $\chi(\GG)-\chi(\FF)$.
\end{proof}

We want to extend the definition of $\LL_{\GG/\FF}$ for every pair of Selmer structures $\FF$ and $\GG$, not necessarily comparable.
\begin{definition}
Let $\FF$ and $\GG$ be cartesian Selmer structures. We define the \emph{local quotient} as
\[\LL_{\GG/\FF}:=\bigoplus_{\ku\in \Sigma_{\FF< \GG}} \frac{\bH^1_{\GG}(K_\ku,T)}{\bH^1_{\FF}(K_\ku,T)}\oplus \bigoplus_{\ku\in \Sigma_{\GG<\FF}} \left(\frac{\bH^1_{\FF}(K_\ku,T)}{\bH^1_{\GG}(K_\ku,T)}\right)^+\]
where $\Sigma_{\FF<\GG}$ (resp. $\Sigma_{\GG<\FF}$) is the set of ultraprimes $\ku\in \Sigma_{\FF}\cup\Sigma_{\GG}$ such that $\bH^1_{\FF}(K_\ku,T)\subset \bH^1_{\GG}(K_\ku,T)$ (resp. $\bH^1_{\GG}(K_\ku,T)\subset \bH^1_{\FF}(K_\ku,T)$).
\label{def:local_quotient}
\end{definition}

\begin{proposition}
Let $\FF\leq \GG$ be cartesian Selmer structures. There is a canonical homomorphism
\[\phi_{\GG,\FF}:\ \bigcap^{\chi(\GG)} \bH^1_{\GG}(K,T)\otimes \det(\LL^+_{\GG/\FF})\to \bigcap^{\chi(\FF)} \bH^1_{\FF}(K,T)\]
\label{prop:cartesian_bidual_map}
\end{proposition}

\begin{proof}
Since $\LL_{\GG/\FF}$ is a free $R$-module of rank $\chi(\GG)-\chi(\FF)$, Proposition \ref{prop:bidual_map_dvr_free} constructs the map $\phi_{\GG,\FF}$.
\end{proof}

\begin{definition}
Fix a base cartesian Selmer structure $\FF_0$. For every cartesian Selmer structure $\FF$, define
\[\bX(\FF)=\bX_{\FF_0}(\FF)=\bigcap^{\chi(\FF)} \bH^1_{\FF}(K,T)\otimes \det(\LL^+_{\FF/\FF_0})\]
\end{definition}

\begin{proposition}
    Let $\FF\leq \GG$ be two cartesian Selmer structures. Then there is a map
    \[\phi_{\GG/\FF}:\ \bX_{\FF_0}(\GG)\to \bX_{\FF_0}(\FF)\]
    \label{prop:cartesian:map}
\end{proposition}

In order to prove this result, we need the following lemma.
\begin{lemma}
Let $\FF$, $\GG$ and $\FF_0$ be cartesian Selmer structures such that $\FF\leq \GG$. Then there is a canonical isomorphism
\[\det(\LL_{\GG/\FF_0}^+)=\det(\LL_{\GG/\FF}^+)\otimes \det(\LL_{\FF/\FF_0}^+)\]
\label{lem:cartesian:det_split}
\end{lemma}

\begin{proof}
We can split the determinant of $\LL_{\GG/\FF}^+$ as
\[\det\Bigl(\LL_{\GG/\FF_0}^+\Bigr)=\det\left(\bigoplus_{\ku\in \Sigma_{\FF_0< \GG}} \left(\frac{\bH^1_{\GG}(K_\ku,T)}{\bH^1_{\FF_0}(K_\ku,T)}\right)^+\right)\otimes  \det\left(\bigoplus_{\ku\in \Sigma_{\GG<\FF_0}} \frac{\bH^1_{\FF_0}(K_\ku,T)}{\bH^1_{\GG}(K_\ku,T)}\right)\]
Here, all the summands are free $R$-modules, so we can identify them with their biduals. Since $\FF\leq \GG$, there is a partition $\Sigma_{\FF_0<\GG}=\Sigma_{\FF\leq \FF_0<\GG}\sqcup \Sigma_{\FF_0< \FF\leq\GG}$
Hence we can use Propositions \ref{prop:det_quotient} and \ref{prop:det_dual} to split the first determinant as
\[\begin{aligned}
&\det\left(\bigoplus_{\ku\in \Sigma_{\FF_0< \GG}} \left(\frac{\bH^1_{\GG}(K_\ku,T)}{\bH^1_{\FF_0}(K_\ku,T)}\right)^+\right)=\\
&\det\left(\bigoplus_{\ku\in\Sigma_{\FF\leq \FF_0<\GG}} \left(\frac{\bH^1_{\GG}(K_\ku,T)}{\bH^1_{\FF}(K_\ku,T)}\right)^+\right)\otimes\det\left(\bigoplus_{\ku\in\Sigma_{\FF\leq \FF_0<\GG}} \frac{\bH^1_{\FF_0}(K_\ku,T)}{\bH^1_{\FF}(K_\ku,T)}\right)\otimes\\
&\det\left(\bigoplus_{\ku\in\Sigma_{\FF_0< \FF\leq\GG}} \left(\frac{\bH^1_{\GG}(K_\ku,T)}{\bH^1_{\FF}(K_\ku,T)}\right)^+\right)\otimes\det\left(\bigoplus_{\ku\in\Sigma_{\FF_0< \FF\leq\GG}} \left(\frac{\bH^1_{\FF}(K_\ku,T)}{\bH^1_{\FF_0}(K_\ku,T)}\right)^+\right)\\
\end{aligned}\]
Since $\FF\leq \GG$, the sets $\Sigma_{\GG<\FF_0}$ and $\Sigma_{\FF\leq \GG<\FF_0}\sqcup \Sigma_{\GG< \FF\leq\FF_0}$ coincide. Similarly, we can use Propositions \ref{prop:det_quotient} and \ref{prop:det_dual} to split the determinant as
\[\begin{aligned}
&\det\left(\bigoplus_{\ku\in \Sigma_{\GG<\FF_0}} \frac{\bH^1_{\FF_0}(K_\ku,T)}{\bH^1_{\GG}(K_\ku,T)}\right)=\\
&\det\left(\bigoplus_{\ku\in\Sigma_{\FF\leq \GG<\FF_0}} \frac{\bH^1_{\FF_0}(K_\ku,T)}{\bH^1_{\FF}(K_\ku,T)}\right)\otimes\det\left(\bigoplus_{\ku\in\Sigma_{\FF\leq \GG<\FF_0}} \left(\frac{\bH^1_{\GG}(K_\ku,T)}{\bH^1_{\FF}(K_\ku,T)}\right)^+\right)\\
\end{aligned}\]
By grouping the different terms, we obtain that 
\[\det(\LL_{\GG/\FF_0}^+)=\det(\LL_{\GG/\FF}^+)\otimes \det(\LL_{\FF/\FF_0}^+)\]

\textcolor{red}{revise sign conventions with direct sums}
\end{proof}

\begin{proof}[Proof of Proposition \ref{prop:cartesian:map}]
By Lemma \ref{lem:cartesian:det_split}, tensoring the map from Proposition \ref{prop:cartesian_bidual_map} with the identity on $\det(\LL_{\FF/\FF_0}^+)$, we obtain the desired map $\phi_{\GG/\FF}$.
\end{proof}

\begin{proposition}
The assignment $\FF\to \bX_{\FF_0}(\FF)$ forms an inverse system indexed by the set of cartesian Selmer structure.
\label{prop:cartesian:inverse_system}
\end{proposition}

\begin{proof}
\textcolor{red}{do}
\end{proof}

\begin{definition}
We define the set of cartesian systems as the elements in the inverse limit
\[\CART_{\FF_0}=\varprojlim_\FF \bX_{\FF_0}(\FF)\]
where the limit is taken over all cartesian Selmer structures.
\label{def:cartesian_system}
\end{definition}

\subsection{Core cartesian Selmer structures}

\begin{definition}
A cartesian Selmer structure is called a \emph{core structure} when 
\[\bH^1_{\FF^*}(K,T)=0\]
\end{definition}

\begin{proposition}
Under Assumptions \ref{ass:patched:basic}, for every cartesian Selmer structure $\FF$, there is core structure $G$ such that $\FF\leq \GG$.
\label{prop:cartesian:core_upper}
\end{proposition}

\begin{proof}
    \textcolor{red}{complete}
\end{proof}

\begin{proposition}
Let $\FF\leq \GG$ be two core Selmer structures. Then the map $\bX_{\FF_0}(\GG)\to \bX_{\FF_0}(\FF)$ is an isomorphism.
\label{prop:cartesian:core_iso}
\end{proposition}

\begin{proof}
\textcolor{red}{complete}
\end{proof}

Similarly to what happened with Kolyvagin and Stark systems, cartesian systems can be controlled by core Selmer structures.
\begin{proposition}
Let $\FF$ be a core Selmer structure. Then the map
\[\CART_{\FF_0}\to \bX_{\FF_0}(\FF)\]
is an isomorphism.
\label{prop:cartesian_core}
\end{proposition}

\begin{proof}
Let $c\in \bX_{\FF_0}(\FF)$. We will show that there is a unique $\varepsilon\in \CART$ such that $\varepsilon_{\FF}=c$.

Let $\GG$ be a cartesian structure. By Proposition \ref{prop:cartesian:core_upper}, there is a core Selmer structure $\HH$ such that $\FF\leq \HH$ and $\GG\leq \HH$. Since the maps $\phi_{\HH,\FF}:\bX(\HH)\to \bX(\FF)$ is an isomorphism by Proposition \ref{prop:cartesian:core_iso}, the definition of the inverse limit implies that $\varepsilon_\GG$ is the image of the $\varepsilon_{\FF}$ under the map
\[\phi_{\HH,\GG}\circ \phi_{\HH,\FF}^{-1}:\ \bX(\FF)\to \bX(\GG):\ \varepsilon_{\FF}\to \varepsilon_{\GG}\]
Hence, the element is determined by $c$. Morevoer, this definition clearly defines an element of $\CART$, so the projection map is an isomorphism.
\end{proof}

We can now prove that the module of cartesian systems is free of rank one over $R$.

\begin{proposition}
Let $\FF$ be a core Selmer structure. Then $\bX_{\FF_0}(\FF)$ is a free, cyclic $R$-module.
\label{prop:cartesian_cyclic}
\end{proposition}

\begin{proof}
\textcolor{red}{missing}
\end{proof}

\begin{corollary}
The module of cartesian systems $\CART_{\FF_0}$ is a free, cyclic $R$-module.
\end{corollary}

\section{Partial cartesian systems}

In practice, constructing a cartesian system is complicated, and we are only able to construct classes for a subgraph of the cartesian Selmer structures. The goal of this section will be to establish conditions on the subgraph that guarantee that a collection of classes in the subgraph extend (uniquely) to a cartesian system.

\begin{definition}
Let $G$ be a subgraph of the graph of cartesian Selmer structures in definition \ref{def:cartesian:graph}, denoted \emph{partial cartesian graph}. A \emph{partial cartesian system} for $G$ is an element of the inverse limit
\[\CART_{\FF_0}(G)=\varprojlim_{\FF\in G} \bX_{\FF_0}(\FF)\]
Therefore, a partial cartesian system is a collection of classes $\varepsilon_\FF$ for every $\FF\in G$ such that, for every arrow $\GG\to \FF$ in $G$, $\phi_{\GG,\FF}(\varepsilon_{\GG})=\varepsilon_{\FF}$.
\end{definition}



\begin{remark}
For every partial cartesian graph, there is canonical restriction map
\[\CART_{\FF_0}\to\CART_{\FF_0}(G)\]
\label{rem:cartesian:partial_res}
\end{remark}

Kolyvagin and Stark systems are examples of partial cartesian systems.

\textcolor{red}{Explain Kolyvagin and Stark systems}

\begin{notation}
For every subgraph $G$ of the cartesian Selmer structures. We denote by $G_0$ to the subgraph of $G$ whose vertices are the core cartesian Selmer structures in $G$ and the edges are the edges of $G$ connecting two of those structures. We also denote the enhanced graph $\widetilde G$ by adding, whenever there is an arrow in $G$, $\FF\to \GG$, joining two core structures, an arrow in the opposite direction, $\GG\to \FF$. The reason for doing so is Proposition \ref{prop:cartesian:core_iso}.
\end{notation}

\begin{proposition}
Assume we have two paths
\[\begin{aligned}
& \FF\to \HH_1\to \cdots \to\HH_s\to \GG\\
& \FF \to \HH_1'\to \cdots \to \HH'_{s'} \to \GG
\end{aligned}\]
Then the maps $\bX_{\FF_0}(\FF) \to \bX_{\FF_0}(\GG)$ induced by both paths coincide.
\label{prop:cartesian:loops}
\end{proposition}

\begin{proof}
\textcolor{red}{write}
\end{proof}

The next results concern about situations in which the map in Remark \ref{rem:cartesian:partial_res} is surjective.

\begin{proposition}
Let $G=G_0$ be a connected partial cartesian graph whose vertices are core structures. Then the map in Remark \ref{rem:cartesian:partial_res} is an isomorphism.
\label{prop:cartesian:partial_core}
\end{proposition}

\begin{proof}
For any two Selmer structures $\FF$ and $\GG$ in $G_0$, we can choose a path in $\widetilde G$.
\[\FF\to \HH_1\to \cdots\to \HH_s\to \GG\]
By Proposition \ref{prop:cartesian:core_iso}, there are isomorphisms
\[\bX_{\FF_0}(\FF)\cong\bX_{\FF_0}(\HH_1)\cong \cdots \cong \bX_{\FF_0}(\HH_s)\cong \bX_{\FF_0}(\GG)\]
For every $\varepsilon\in \CART_{\FF_0}(G_0)$, $\varepsilon_\FF$ and $\varepsilon_{\GG}$ are identified by this isomorphism. Hence, $\CART_{\FF_0}(G_0)$ is isomorphic to $\bX_{\FF_0}(\FF)$.

Choose a core Selmer structure $\FF$. By Proposition \ref{prop:cartesian_core}, the map in Remark \ref{rem:cartesian:partial_res} is the following isomorphism.
\[\CART_{\FF_0}\cong\bX_{\FF_0}(\FF)\cong \CART_{\FF_0}(G)\]
\end{proof}

\begin{proposition}
Let $G$ be a non-empty partial cartesian graph
\begin{itemize}
    \item The graph $G_0$ is connected.
    \item For every $\FF\in G$, there is some core Selmer structure $\GG\in G_0$ such that $\FF\leq \GG$ and the arrow $\GG\to \FF$ is in $G$.
\end{itemize}
Then the map in Remark \ref{rem:cartesian:partial_res} is an isomorphism.
\label{prop:cartesian:partial_core_upper}
\end{proposition}

\begin{proof}
The second assumption implies that $G_0$ is non-empty. By Proposition \ref{prop:cartesian:partial_core}, there is an isomorphism 
\[\CART_{\FF_0}\to \CART_{\FF_0}(G_0)\]

For every $\FF$ in $G$, choose a core Selmer structure $\GG\in G_0$ with an arrow $\GG\to \FF$ in $G$. If $\varepsilon \in \CART_{\FF_0}(G)$, then $\varepsilon_{\FF}=\phi_{\GG,\FF}(\varepsilon_{\GG})$. By Proposition \ref{prop:cartesian:loops}, for any other such $\GG'\in G_0$, we have that $\phi_{\GG,\FF}(\varepsilon_{\GG})=\phi_{\GG',\FF'}(\varepsilon_{\GG'})$, so this defines a well defined partial cartesian system on $G$.
\end{proof}

\section{Higher dimensional Kolyvagin systems}

\subsection{Higher dimensional Kolyvagin primes}

In this section, we assume the weaker version of Assumption \ref{ass:basic}.

\begin{assumption}
We assume the following assumptions:
\begin{itemize}
\item \namedlabel{nsTirred}{(T1)} $T/\m T$ is an irreducible $k[[G_K]]$-module.
\item \namedlabel{nsTcoh}{(T3)} $H^1(K(T)_M/K,T)=H^1(K(T)_M/K,T^*)=0$.
\item\namedlabel{nsNsd}{(N1)} $T/\m T$ is not isomorphic to $T^*[\m]$ as $k[[G_K]]$-modules.
\item\namedlabel{nsNsur}{(N2)} The image of the homomorphism $R\to \textrm{End}(T)$ is contained in the image of $\Z_p[[G_\Q]]\to \textrm{End}(T)$.
\end{itemize}
\label{ass:nonsur}
\end{assumption}

We are not assuming the existence of $\tau$ such that $T/(\tau-1)T$ is a free, cyclic $R$-module. However, there always exist $\tau\in G_K$ such that $T/(\tau-1)T$ is a free $R$-module of rank $d$.

\begin{notation}
Fix some $\tau\in G_{K_M}$ such that $T/(\tau-1)T\cong R^d$, minimizing the exponent $d$.
\label{not:tau}
\end{notation}

Similarly to Definition \ref{def:kolyvagin_primes}, we define the Kolyvagin primes $\ell\in \PP_\tau$ as those not belonging to $\Sigma_\FF$ and whose Frobenius automorphism is conjugate to $\tau$ in $\Gal(K(T)_M/K)$. We also denote by $\NN_\tau$ (resp. by $\NN_\tau^i$) to the set of square-free products of Kolyvagin primes (resp. square-free products of exactly $i$-primes).

Let $\ell\in \PP_\tau$. Since $\Frob_\ell$ is trivial in $\Gal(K_M/K)$, then \textcolor{red}{comment} there is an splitting
\[H^1(K_\ell,T)=H^1_\f(K_\ell,T)\oplus H^1_\tr(K_\ell,T)\]
which is a free $R$ module of rank $d$.

For every $\ku=(\ell_i)_{i\in \N}\in \PP_{\tau}$, we have some freedom in the choices of the primes at $\overline Q$ above $\ell_i$. We can make this choices so $\Frob_\ku=\tau$ in $\Gal(K(T)_M/K)$, instead of being only conjugates. Then there is an isomorphism
\[H^1_\f(K_\ku, T)\cong T/(\Frob_\ku-1) T=T/(\tau-1) T,\ c\mapsto c(\Frob_\ku)+(\tau-1)T\]

After fixing a generator of $\GG_\ell$, there is another canonical ismorphisms.
\[\phi_\ku^{\fs}:\ H^1_\f(K_\ku, T)\to H^1_\tr(K_\ku, T)\]

\subsection{Admissible elements}

\begin{definition}
An element $\kn\in \NN$ is said to be \emph{admissible} if, for every $\ku\mid \kn$, the map 
\[\loc_\ku:\ \bH^1_{\FF^{\kn/\ku}}(K,T)\to \bH^1_\f(K_\ku,T)\]
Denote by $\AA$ the set of admissible $n\in \NN$. For every admissible $\kn\in \AA$, denote by $A_{\kn,\ku}$ the minimal indivisible free $R$-subgroup of $\bH^1_\f(K_\ku,T)$ containing $\Im(\loc_\ku)$.
\end{definition}

The next proposition gives a method to construct admissible products.
\begin{proposition}
Let $\ku_1,\ldots, \ku_s\in \PP$ be a set of Kolyvagin ultraprimes satisfying that for all $i=1,\ldots, s$, 
\begin{itemize}
    \item The image of $\loc_{\ku_i}:\bH^1_{\FF^{\ku_1\cdots\ku_{i-1}}}(K,T)\to \bH^1_{\f}(K_{\ku_i},T)$ is a free rank $1$ $R$-submodule.
    \item For every $j\in \{1,\ldots, i-1\}$, we have that
    \[\bH^1_{(\FF^*)^{\ku_1\ldots\ku_{i-1}}_{\ku_i}}(K,T^*)+ \bH^1_{(\FF^*)_{\ku_1\cdots\ku_{i-1}}}(K,T^*)=\bH^1_{(\FF^*)^{\ku_1\ldots\ku_{i-1}}}(K,T^*)     \]
\end{itemize}
Then $\kn:=\ku_1\cdots\ku_s$ is admissible.
\label{prop:admissible_condition}
\end{proposition}

\begin{proof}
Consider the exact sequence
\[\xymatrix{\bH^1_{\FF^{\kn/\ku_i}_{\ku_i}}(K,T)\ar@{>->}[r] &  \bH^1_{\FF^{\kn/\ku_i}}(K,T)\ar[r] & \bH^1_\f(K_{\ku_i},T) \ar[r] & \bH^1_{(\FF^*)_{\kn/\ku_i}^{\ku_i}}(K,T^*)^\vee \ar@{->>}[r] &\bH^1_{(\FF^*)_{\kn/\ku_i}}(K,T^*)^\vee}\]

Denote $r_i:=\ku_1\cdots \ku_{i-1}$ and $s_i:=\ku_{i+1}\cdots \ku_s$. We claim that, for every $j=i+1,\ldots,s$, the images of the maps
\[\begin{aligned}
&\loc_{\ku_i}^*:\ \bH^1_{(\FF^*)_{r_i\ku_{i+1}\cdots \ku_{j-1}}^{\ku_i}}(K,T^*)&\to \bH^1_\s(K_{\ku_i},T^*)\\
&\loc_{\ku_i}^*:\ \bH^1_{(\FF^*)_{r_i\ku_{i+1}\cdots \ku_{j-1}\ku_j}^{\ku_i}}(K,T^*)&\to \bH^1_\s(K_{\ku_i},T^*)
\end{aligned}\]

By the second assumption on the choice of the prime $\ku_j$ implies that 
\[\bH^1_{(\FF^*)_{r_i\ku_{i+1}\cdots \ku_{j-1}\ku_j}^{\ku_i}}(K,T^*)+ H^1_{(\FF^*)_{\ku_1\cdots{\ku_{j-1}}}}(K,T^*)=\bH^1_{(\FF^*)_{r_i\ku_{i+1}\cdots \ku_{j-1}}^{\ku_i}}(K,T^*)\]
Since the second summand is contained in the kernel of $\loc_{\ku_i}$, the images of both maps coincide. By induction, we can construct an exact sequence
\[\xymatrix{\bH^1_{\FF^{\kn/\ku_i}_{\ku_i}}(K,T)\ar@{>->}[r] &  \bH^1_{\FF^{\kn/\ku_i}}(K,T)\ar[r] & \bH^1_\f(K_{\ku_i},T) \ar[r] & \bH^1_{(\FF^*)_{r_i}^{\ku_i}}(K,T^*)^\vee \ar@{->>}[r] &\bH^1_{(\FF^*)_{r_i}}(K,T^*)^\vee}\]
Comparing it with the exact sequence
\[\xymatrix{\bH^1_{\FF^{r_i}_{\ku_i}}(K,T)\ar@{>->}[r] &  \bH^1_{\FF^{r_i}}(K,T)\ar[r] & \bH^1_\f(K_{\ku_i},T) \ar[r] & \bH^1_{(\FF^*)_{r_i}^{\ku_i}}(K,T^*)^\vee \ar@{->>}[r] &\bH^1_{(\FF^*)_{r_i}}(K,T^*)^\vee}\]
we can see that the images of the following maps coincide:
\[\begin{aligned}
&\loc_{\ku_i}:\ \bH^1_{\FF^{\kn/\ku_i}}(K,T)&\to \bH^1_{\f}(K_{\ku_i},T)\\
&\loc_{\ku_i}:\ \bH^1_{\FF^{r_i}}(K,T)&\to \bH^1_{\f}(K_{\ku_i},T)
\end{aligned}\]
The image of the second map is contained one-dimensional, free $R$-module by the first assumption on the choice of the prime $\ku_i$. Hence the image of the first map is also contained in a one-dimensional, free $R$-module. Since that holds for every prime divisor of $\kn$, then $\kn$ is admissible.
\end{proof}

\subsection{Suitable higher dimensional Kolyvagin primes}

The goal of this section is to adapt Proposition \ref{prop:cheb_nd} to higher dimensional Kolyvagin primes. We adapt the proof of \cite[Proposition 3.6.2]{MazurRubin} to this situation.

\begin{proposition}
Assume Assumptions \ref{ass:nonsur}. Let $C\subset \bH^1(K,T)$ be a finitely-generated $R$-module and let $D\subset \bH^1(K,T^*)$ be a cofinitely generated $R$-module. Assume $\tau\in G_{K_M}$ is as in Notation \ref{not:tau}. Assume we have maps
\[\begin{array}{cc}
\phi:\ C\to T/(\tau-1)T,\ \psi:\ D\to T^*/(\tau-1)T^*
\end{array}\]
Then there is a $\tau$-Kolyvagin ultraprime $\ku\in \PP_\tau$ such that the maps
\[\begin{array}{cc}
\loc_\ku:\  C\to T/(\tau-1)T,\ \ \loc_\ku^*:\ D\to T^*/(\tau-1)T^*
\end{array}\]
coincide with $\phi$ and $\psi$, respectively.
\label{prop:cheb_nd_ns}    
\end{proposition}

\begin{proof}
Since $C$ is finitely generated, there exists a square-free product of ultraprimes $\kn$ such that $C\subset \bH^1(K^\kn/K,T)$.

Denote by $C_k$ the image of $C$ under the projection to $\bH^1(K,T/\m^k)$ and by $D_k$ the intersection of $D$ with $\bH^1(K,T^*)[\m^k]$. The assumptions on $C$ and $D$ imply that both $C_k$ and $D_k$ are finite. Since $C$ is finitely generated, it is compact, to it equal to 
\[C=\varprojlim_k C_k\subset \varprojlim_k \bH^1(K^\kn/K,T/\m^k)\]
From $\phi$ and $\psi$, we obtain the maps
\[\begin{array}{cc}
    \phi_k:\ \bH^1(K^{\kn}/K,T/\m^k)\to T/(\m^k,\tau-1)T,& \psi_k:\ \bH^1(K,T^*[\m^n])\to T^*[\m^k]/(\tau-1)T^*[\m^k]
\end{array}\]

Let $(n_i)_{i\in \N}$ be a sequence of square-free integers representing $\kn$. Then 
\[\bH^1(K^{\kn}/K,T/\m^k)=\UU_i\Bigl(H^1(K^{n_i}/K,T/\m^k)\Bigr)\]

\textcolor{red}{result needed}, there are subgroups $C_k^{(i)}\subset H^1(K^{n_i}/K,T/\m^k)$ such that 
\[C_k=\UU_i\Bigl(C_k^{(i)}\Bigr)\]
Since $C_k$ is finite, then $C_k\cong C_k^{(i)}$ for $\UU$-many $i$. Let $S_k$ be the set of indices for which this isomorphism holds. Then we have a decreasing sequence of sets $(S_k)\subset \UU$. \textcolor{red}{more comment}

For every $i\in S_k$, the isomorphism $ C_k^{(i)}\cong C_k$ induces a map
\[\phi_k^{i}:\ C_k^{(i)}\subset H^1(K^{n_i}/K,T/\m^k)\to T/(\m^k,\tau-1)T\]

\textbf{Claim}: There exists a $\tau$-Kolyvagin prime $\ell_k^{i}\in \PP_{\tau}(T/\m^k)$ not dividing $n_i$ such that 
\[\loc_{\ell_k^{i}}:\ C_{k}^{(i)}\hookrightarrow H^1(K^{n_i}/K,T/\m^k)\to H^1_{\f}(K_{\ell_k^{(i)}},T/m^k)=T/(\m^k,\tau-1)T\] coincides with $\phi_k$. Note that it implies that the map \textcolor{red}{proof required}
\[\loc_{\ell_k^i}:\ C_{k'}^{(i)}\to T/(\m^{k'},\tau-1)\]
coincides with $\phi_{k'}$.

Consider the ultraprime $\ku$ represented by the sequence $(\ell_a^a)_{a\in \N}$. For every $a\in S_a\cap \N_{\geq a}$, the map
\[\loc_{\ell_a^a}:\ C_k^{(a)}\subset H^1(K^{n_a}/K,T/\m^k)\to \bH^1(K_{\ell_a^a},T/\m^k)\]
coincides with $\phi_k^{(a)}$.

Therefore, the map
\[\loc_\ku:\ \bH^1(K^{\kn}/K,T/\m^k)\to \bH^1(K_\ku,T/\m^k)\cong T/(m^k,\tau-1) T\]
coincides with $\phi_{k}$. By taking limits, the map
\[\loc_{\ku}:\ \bH^1(K^{\kn}/K,T)\to  \bH^1(K_\ku,T)\cong T/(\tau-1) T\]

\textcolor{red}{complete the dual side}
\end{proof}



\section{Kolyvagin systems}

Definition \ref{def:kol} can be generalised to this setting with the only modification of the original Kolyvagin primes in $\PP$ to rank $d$ Kolyvagin primes in $\PP_\tau$.

\begin{definition}
A \emph{$\tau$-Kolyvagin system} for a Selmer structure $\FF$
$$\kappa=\left\{\kappa_\kn\in H^1_{\mathcal F(\kn)}(K,T):\ \kn\in \AA_\tau\right\}$$
satisfying the following relation for every $\kn\in \AA_\tau$ and $\ku\in \PP_\tau$ not dividing $\kn$. By the definition of Selmer module, we have that 
\[\begin{array}{cc}
\loc_\ku(\kappa_\kn)\in H^1_{\FF(\kn)}(K_\ku,T)=H^1_\f(K_\ku,T),\ &\loc_\ell(\kappa_{\kn\ku})\in H^1_{\FF(\kn\ku)}(K_\ku,T)=H^1_\tr(K_\ku,T)
\end{array}\]
The collection $\kappa$ is a Kolyvagin system if the following is satisfied
\begin{equation}
\loc_\ku(\kappa_{\kn\ku})=\phi_\ku^{\fs}\circ \loc_\ku(\kappa_\kn)
\label{eq:kol_cond}
\end{equation}
for every $\kn\in \AA_\tau$ and $\ku\in \PP_\tau$ not dividing $n$.
\label{def:kol_tau}
\end{definition}

We will aim to construct a partial cartesian system from the classes $\kappa_n$, where $\kappa$ is a $\tau$-Kolyvagin systems and $n\in \NN_\tau$ is admissible.

Recall that, when $n\in\AA_\tau$ is admissible, we had defined a free, cyclic $R$-submodule $A_{\kn,\ku}$ such that 
\[\loc_\ku\Bigl(H^1_{\FF^{\kn/\ku}}(K,T)\Bigr)\subset A_{\kn,\ku}\subset H^1_{\f}(K,T)\]

\begin{proposition}
Assume that $\bH^1_{\FF}(K,T)\neq \bH^1_{\FF_\ku}(K,T)$. Then $A_{\kn,\ku}$ coincides for all admissible $\kn$ dividing $\ku$. 
\end{proposition}

Choose a free $R$-module $B_{n,\ku}$ of rank $d-1$ such that 
\[A_{\kn,\ku}\oplus B_{\kn,\ku}=H^1_{\f}(K,T)\]

We can define, for every admissible $\kn\in \NN$, a Selmer structure $\FF[\kn]$.

\begin{definition}
Let $\kn\in \NN$ be admissible. Define the Selmer structure $\FF[\kn]$ by the local conditions 
\[\left\{\begin{aligned}
&\bH^1_{\FF[n]}(K_\ku,T)=B_n\oplus \phi_\ku^\fs(A_n)&\textrm{ if }\ku\mid \kn\\
&\bH^1_{\FF[n]}(K_\ku,T)=\bH^1_{\FF}(K_\ku,T)&\textrm{ if }\ku\nmid \kn\\
\end{aligned}\right.\]
\end{definition}

\begin{proposition}
Let $\kappa\in \KS_\tau(\FF)$. For every admissible $n\in \NN_\tau$, $\kappa_n\in \bH^1_{\FF[n]}(K,T)$.
\end{proposition}

\begin{proof}
For every $\ku\nmid\kn$, then
\[\loc_\ku(\kappa_n)\in \bH^1_{\FF(n)}(K,T)=\bH^1_{\FF[n]}(K,T)\] 
Alternatively, when $\ku\mid \kn$, we have that 
\[\phi_\ku^\fs\circ \loc_\ku(\kappa_{\kn/\ku})=\loc_\ku(\kappa_kn)\]
Since $\kn$ is admissible, the definition of $A_{\kn,\ku}$ implies that 
\[\loc_\ku(\kappa_{\kn/\ku})\in \loc_\ku\Bigl(\bH^1_{\FF(\kn/\ku)}(K,T)\Bigr)\subset \loc_\ku\Bigl(\bH^1_{\FF^{\kn/\ku}}(K,T)\Bigr)\subset A_n\]
Therefore,
\[\loc_\ku(\kappa_{\kn})\in \phi_\ku^{\fs}(A_\kn)\subset \bH^1_{\FF[n]}(K,T)\]
\end{proof}

\begin{proposition}
Let $\kappa \in \KS_\tau(\FF)$. The set 
\[\Bigl\{\kappa_\kn^{++}\in \bigcap^{1} \bH^1_{\FF[n]}(K,T): \kn\in \AA\Bigr\}\]
induces a partial cartesian system on a graph with vertices
\[\{\FF[\kn]:\kn\in \AA\}\cup \{\FF[\kn_\ku]:\kn\ku\in \AA\}\]
\end{proposition}

\begin{proof}
\textcolor{red}{do}
\end{proof}

We can also define partially relaxed and restricted Selmer structures at admissible elements.

\begin{definition}
Let $\ka,\kb,\kn\in \AA$ be admissible elements such that $\ka\mid \kn$, $\kb\mid \kn$ and $\ka$ and $\kb$ are pairwise coprime. The Selmer structure $\FF[\kn_\ka^\kb]$ is defined by the local conditions
\[\left\{\begin{aligned}
&\bH^1_{\FF[n_\ka^\kb]}(K_\ku,T)=B_n&\textrm{ if }\ku\mid \ka\\
&\bH^1_{\FF[n_\ka^\kb]}(K_\ku,T)=\bH^1_\f(K_\ku,T)\oplus \phi_\ku^\fs(A_n)&\textrm{ if }\ku\mid \kb\\
&\bH^1_{\FF[n_\ka^\kb]}(K_\ku,T)=B_n\oplus \phi_\ku^\fs(A_n)&\textrm{ if }\ku\mid \kn/\ka\kb\\
&\bH^1_{\FF[n_\ka^\kb]}(K_\ku,T)=\bH^1_{\FF}(K_\ku,T)&\textrm{ if }\ku\nmid n\\
\end{aligned}\right.\]
\end{definition}

\textcolor{red}{check subspaces when increasing $n$}

\subsection{Paths of admissible elements}

\begin{lemma}
Let $\kc,\kn\in \AA$ be admissible elements such that $\kc\mid \kn$, and let $\ku\in \PP_{\tau}$ be a $\tau$-Kolyvagin prime. Assume that the image of the of the map
\[\loc_{\ku}:\ \bH^1_{\FF[\kc]}(K,T)\to H^1(K_\ku,T)\]
is a maximal one-dimensional subspace. Then 
\[\bH^1_{\FF[\kc\ku]^*}(K,T^*)= \bH^1_{\FF[\kc^{\ku}]^*}(K,T^*)\]
\label{lem:hd:tr_res}
\end{lemma}

\begin{proof}
The above condition is equivalent to $\Im(\loc_\ku)=A_{\kn,\ku}$. There is an exact sequence
\[\xymatrix{\bH^1_{\FF[\kc_\ku]}(K,T)\ar@{>->}[r] & \bH^1_{\FF[\kc]}(K,T)\ar[r] &A_{n,\ku} \ar[r] & \bH^1_{\FF[\kc_{\ku}]^*}(K,T^*)\ar@{->>}[r] &\bH^1_{\FF[\kc]^*}(K,T^*)}\]
This implies that $\bH^1_{\FF[\kc_{\ku}]^*}(K,T^*)=\bH^1_{\FF[\kc]^*}(K,T^*)$. Note that 
\[\bH^1_{\FF[\kc^{\ku}]^*}(K,T^*)=\bH^1_{\FF[\kc]^*}(K,T^*)\cap \bH^1_{\FF[\kc\ku]^*}(K,T^*)\]
Since $\bH^1_{\FF[\kc\ku]^*}(K,T^*)\subset \bH^1_{\FF[\kc_{\ku}]^*}(K,T^*)=\bH^1_{\FF[\kc]^*}(K,T^*)$, we have that 
\[\bH^1_{\FF[\kc\ku]^*}(K,T^*)= \bH^1_{\FF[\kc^{\ku}]^*}(K,T^*)\qedhere\]
\end{proof}

\begin{proposition}
Let $\FF$ be a cartesian Selmer structure such that 
\[\bH^1_{\FF^*}(K,T^*)^\vee\cong R^r\times R/\m^{e_1}\times \cdots\times R/\m^{e_s}\]
For every $i=1,\ldots, r+s$, there is an admissible $\kn_i\in \AA_i$ such that 
\[\begin{aligned}
&\bH^1_{\FF[\kn_i]}(K,T^*)^\vee\cong R^{r-i}\times R/\m^{e_1}\times \cdots\times R/\m^{e_s}\ &\forall i\in\{1,\ldots,r\}\\
&\bH^1_{\FF[\kn_i]}(K,T^*)^\vee\cong  R/\m^{e_{j+1-r}}\times \cdots\times R/\m^{e_s}\ &\forall i\in\{j+1,\ldots,r+s\}\\
\end{aligned}\]
\end{proposition}

\begin{proof}
Choose a maximal, one-dimensional, free $R$ submodule of $T/(\tau-1)T$, denoted by $A$.

Choose also a divisible, of corank one, $R$ submodule of $T^*/(\tau-1)$, denoted by $A^*$ such that 
\[A^*\cap \Ann(\varphi^\fs(A))=0\]

We proceed by induction on $i$. Assume we have constructed $\kn_i=\ku_1\ldots\ku_j\in \AA_i$ satisfying the condition of the proposition, and that the choices of the ultraprimes $\ku_j$ satisfy the conditions in Proposition \ref{prop:admissible_condition}. Assume also that the quotient
\[\frac{\bH^1_{(\FF^*)^{\ku_1\cdots\ku_{i-1}}}(K,T^*)}{\bH^1_{\FF^*}(K,T^*)}\]
is $\m$-divisible. 

Since the core rank of $\FF^{\ku_1\cdots \ku_{i-1}}$ is positive, $\bH^1_{\FF^{\ku_1\cdots \ku_{i-1}}}(K,T^*)$ contains a free submodule. Therefore, there exists a map
\[\phi:\ \bH^1_{\FF^{\ku_1\cdots \ku_{i-1}}}(K,T)\to T/(\tau-1)T\]
whose image is exactly $A$. The assumptions on the dual Selmer group imply the existence of the of a map
\[\psi:\ \bH^1_{(\FF^*)^{\ku_1\cdots\ku_{i-1}}}(K,T^*)\to T^*/(\tau-1)T^*\]

\textcolor{red}{missing}

\textcolor{red}{complete}
\end{proof}













