
\chapter{Kolyvagin systems}



\section{Kolyvagin systems and theta ideals}
\label{sec:core_rank_one}


In this section, we outline the classical theory of Kolyvagin systems, as described in \cite{MazurRubin}. This theory is limited to principal coefficient rings $R$ and core rank being equal to one.

\begin{assumption}
Assume, in addition to Assumption \ref{ass:R} that $R$ is a principal ring, with $\pi$ denoting a generator of the maximal ideal. Moreover, consider a cartesian Selmer structure of core rank $\chi(\FF)=1$.
\label{ass:Rprin_chi1}
\end{assumption}

\begin{notation}
In order to simplify the notation, we fix a generator $\tau_\ell$ of $\GG_\ell=\Gal(K(\ell)/K)$ for every Kolyvagin prime $\ell \in \PP$. This choice fixes, by Proposition \ref{prop:finite-singular}, an isomorphism
\[\varphi_\ell^\fs: H^1_\f(K_\ell,T)\to H^1_\s(K_\ell, T)\]
\end{notation}

\begin{definition}
A \emph{Kolyvagin system} for a Selmer structure $\FF$
$$\kappa=\left\{\kappa_n\in H^1_{\mathcal F(n)}(K,T):\ n\in \NN\right\}$$
satisfying the following relation for every $n\in \NN$ and $\ell\in \PP$ not dividing $n$. By the definition of Selmer module, we have that 
\[\begin{array}{cc}
\loc_\ell(\kappa_n)\in H^1_{\FF(n)}(K_\ell,T)=H^1_\f(K_\ell,T),\ &\loc_\ell(\kappa_{n\ell})\in H^1_{\FF(n\ell)}(K_\ell,T)=H^1_\tr(K_\ell,T)
\end{array}\]
The collection $\kappa$ is a Kolyvagin system if the following is satisfied
\begin{equation}
\loc_\ell(\kappa_{n\ell})=\phi_\ell^{\fs}\circ \loc_\ell(\kappa_n)
\label{eq:kol_cond}
\end{equation}
for every $n\in \NN$ and $\ell\in \PP$ not dividing $n$.
\label{def:kol}
\end{definition}



\begin{remark}
The set of Kolyvagin systems has a natural structure of $R$-module. It will be denoted by $\KS(\FF)$.
\end{remark}



Kolyvagin systems carry information about the structure of the Selmer group. The key idea is to look at the classes $\kappa_n$, where $n\in \NN_i$ for the different non-negative integers $i$. The information carried by a single class $\kappa_n$ is seen in its index.



\begin{definition}
Let $M$ be an $R$-module and let $a\in M$. Consider the canonical map into the bidual module
\[\Phi:\ M\to M^{++}: a\in M\mapsto \Bigl[\varphi\in \Hom(M,R)\mapsto \varphi(a)\Bigr]\]
The \emph{index} of $a$ is defined as
\[\ind(a,M)=\Im(\Phi(a))\]
\label{def:index}
\end{definition}

\begin{remark}
When $R$ is a principal, local, artinian ring with maximal ideal $\m$, the index of an element $a\in M$ coincides
\[\ind(a)=\m^{\max\{j\in \N:\ a\in \m^jM\}}\]
\end{remark}

\begin{notation}
When there is no risk of confussion, we will denote $\ind(a)$ instead of $\ind(a,M)$.
\end{notation}

We can now define the ideals $\Theta_i$ as the ideals in $R$ generated by the indices of all $\kappa_n$ where $n\in \NN_i$.

\begin{definition}
Let $\kappa\in \KS(\FF)$. The theta ideals of $\kappa$ are defined as
\[\Theta_i(\kappa):=\sum_{n\in \NN_i} \ind\biggl(\kappa_n, H^1_{\FF(n)}(K,T)\biggr)\]
\end{definition}




\begin{theorem} (\cite[Theorem 4.3.3]{MazurRubin})
Under Assumption \ref{ass:Rprin_chi1}, $\KS(\FF)$ is a free, cyclic $R$-module.
\label{th:KS_chi1}
\end{theorem}

\section{Selmer structures of core rank 1}

The generators of $\KS(\FF)$ are the Kolyvagin systems carrying information about the Selmer group.

\begin{definition}
A Kolyvagin system is said to be \emph{primitive} if it generates $\textrm{KS}(\FF)$ as an $R$-module.
\end{definition}

We can now state the main theorem of loc. cit., which relates the theta ideals of a primitive Kolyvagin systems with the (higher) Fitting ideals of the Selmer group.

\begin{theorem}(\cite[Theorem 4.5.9]{MazurRubin})
Let $R$ be a principal, artinian, local ring with finite residue field, and let $T$ be an $R[[G_K]]$-module, which is free and finitely generated as $R$-module and unramified only at finitely many places. Let $\FF$ be a cartesian Selmer structure on $T$ of core rank $\chi(\FF)=1$. If $\kappa\in \KS(\FF)$ is a primitive Kolyvagin system, then 
\[\Theta_i(\kappa)=\Fitt^R_i(H^1_{\FF^*}(\Q,T^*))\]
\label{th:str_chi1}
\end{theorem}



The proof of Theorem \ref{th:str_chi1} is divided in the following two lemmas:

\begin{lemma}(\cite[Corollary 4.5.4.]{MazurRubin})
Under Assumption \ref{ass:Rprin_chi1}, if $\kappa\in \KS(T)$ is a Kolyvagin system and $n\in \NN$, then
\[\ind(\kappa_n)=\Fitt^0\Bigl(H^1_{\FF^*(n)}(K,T^*)\Bigr)\]
\label{lem:kn_trn}
\end{lemma}

\begin{lemma}
Under Assumption \ref{ass:Rprin_chi1}, then 
\[\Fitt_i^R\Bigl(H^1_{\FF^*}(K,T^*)\Bigr)=\sum_{n\in \NN_i}\Fitt^0\Bigl(H^1_{(\FF^*)(n)}(K,T^*)\Bigr)\]
\label{lem:fitti_fitt0trn}
\end{lemma}

\subsection{Proof of Lemma \ref{lem:fitti_fitt0trn}}

In order to prove Lemma \ref{lem:fitti_fitt0trn}, we start by showing, for any $n\in \NN_i$, the inclusion
\[\Fitt_0^R\Bigl(H^1_{\FF^*(n)}(K,T^*)\Bigr)\subset \Fitt_i^R\Bigl(H^1_{\FF^*}(K,T^*)\Bigr)\]

Consider the exact sequence
\[\xymatrix{0\ar[r] &  H^1_{\FF_n^*}(K,T)\ar[r] & H^1_{\FF^*}(K,T)\ar[r] & \prod_{\ell\mid n} H^1_{\f}{\FF^*}(K,T)  }\]
Since all the prime divisors of $n$ are Kolyvagin primes, the last term is isomorphic to $R^{\nu(n)}$. The description of Fitting ideals over principal rings in Proposition \ref{prop:fitting_dvr} implies that 
\[\Fitt_0^R\Bigl(H^1_{\FF_n^*}(K,T)\Bigr)\subset \Fitt_{\nu(n)}^R\Bigl(H^1_{\FF^*}(K,T)\Bigr)\]
Since $H^1_{\FF_n^*}(K,T)\subset H^1_{\FF^*(n)}(K,T)$, then 
\[\Fitt_0^R\Bigl(H^1_{\FF^*(n)}(K,T)\Bigr)\subset \Fitt_0^R\Bigl(H^1_{\FF_n^*}(K,T)\Bigr)\]
Combining both inclusions, we conclude the proof of this inclusion.

In order to deal with the other inclusion, we need to construct, for each $i\in \Z_{\geq 0}$, a vertex $n_i\in \NN_i$ such that 
\[\Fitt^0\Bigl(H^1_{(\FF^*)(n_i)}(K,T^*)\Bigr)=\Fitt_i^R\Bigl(H^1_{\FF^*}(K,T^*)\Bigr)\]

Assume we we have a structural homomorphism
\[H^1_{\FF}(K,T^*)\approx R^r\times R/\m^{e_1}\times \cdots R/\m^{e_s}\]

An inductive application of Lemma \ref{lem:transverse_reduction_free} (see also Remark \ref{rem:rk1_free}) construct vertices $n_i\in \NN_i$ such that 
\[\begin{aligned}
H^1_{\FF(n_i)}(K,T)\cong R^{r-i} \times R/\m^{e_1}\times \cdots R/\m^{e_s}\ \textrm{for }i\leq r\\
H^1_{\FF(n_{r+j})}(K,T)\cong R/\m^{e_{j+1}}\times \cdots R/\m^{e_s}\ \textrm{for }j\geq 1
\end{aligned}\]
This proves the other inclusion, what concludes the proof of this lemma.












\section{Selmer structures of rank 0}

When $\FF$ is a cartesian Selmer structure of core rank $0$, we cannot apply the argument above since the only Kolyvagin system is the trivial one.

\begin{theorem}(\cite[Theorem 4.2.2]{MazurRubin})
Let $\FF$ be a cartesian Selmer structure such that $\chi(\FF)=0$. Then $\KS(\FF)=0$.
\end{theorem}


The method we will use for the computation of the Selmer module $H^1_{\FF}(K,T)$ involves considering an auxiliary Selmer structure $\GG\geq \FF$, also cartesian, such that $\chi(\GG)=1$. One can show that $\FF$ and $\GG$ only differ in one local condition.

\begin{proposition}
There exists a unique prime $\ell$ such that $H^1_{\FF}(K_q,T)\subsetneq H^1_{\GG}(K_q,T)$. Moreover, there is a non-canonical homomorphism
\[H^1_{\GG/\FF}(K_q,T):=H^1_{\GG}(K_q,T)\Bigm/ H^1_{\FF}(K_q,T)\approx R\]
\end{proposition}

\begin{proof}
By Proposition \ref{prop:global_duality}, there is an global-duality exact sequence for $\Tbar:=T\otimes k$
\[\xymatrix{H^1_{\FF}(K,\Tbar)\ar@{>->}[r] & H^1_{\GG}(K,\Tbar)\ar[r] & \displaystyle{\bigoplus_{q\in \Sigma_\FF\cup\Sigma_\GG} \frac{H^1_{\GG}(K_q,\Tbar)}{H^1_{\FF}(K_q,\Tbar)}} \ar[r] & H^1_\GG(K,\Tbar^*)^\vee\ar@{->>}[r]& H^1_\FF(K,\Tbar^*)^\vee}\]

Since $\chi(\FF)=0$ and $\chi(\GG)=1$, Definition \ref{def:core_rank} and dimension counting implies that 
\[\dim_k\left(\displaystyle{\bigoplus_{q\in \Sigma_\FF\cup\Sigma_\GG} \frac{H^1_{\GG}(K_q,\Tbar)}{H^1_{\FF}(K_q,\Tbar)}}\right)=1\]
Therefore, there exists a unique prime $\ell$ such that ${H^1_{\FF}(K_\ell,\Tbar)}\subsetneq {H^1_{\GG}(K_\ell,\Tbar)}$.
Hence ${H^1_{\FF}(K_\ell,T)}\subsetneq {H^1_{\GG}(K_\ell,T)}$.

For all other primes $q\neq \ell$, we can apply \cite[Lemma 1.1.5]{MazurRubin}, which says that for every pair of cartesian Selmer structures $\FF$ and $\GG$, the quantity
\begin{equation}
\length\Bigl(H^1_{\GG}(K_q,T\otimes R/\m^i)\Bigr)-\length\Bigl(H^1_{\FF}(K_q,T\otimes R/\m^i)\Bigr)
\label{eq:length_dif}
\end{equation}
is linearly dependent on $i$. Since it vanishes for $i=1$, then $H^1_{\FF}(K_q,T)=H^1_{\GG}(K_q,T)$.

For the prime $\ell$, \cite[Lemma 1.1.5]{MazurRubin} implies that 
\begin{equation}
\length\Bigl(H^1_{\GG/\FF}(K_q,T)\Bigr)=\length(R)
\end{equation}

Consider he following composition, which coincides with the multiplication by $\pi^{k-1}$.
\[\xymatrix{H^1_{\GG/\FF}(K_\ell,T)\ar[r] & H^1_{\GG/\FF}(K_\ell,T\otimes k) \ar[r]^{\pi^{k-1}} & H^1_{\GG/\FF}(K_\ell,T)}\]
The first map is surjective by the propagation of Selmer structures and the second one is injective since $\FF$ is cartesian. By the induction hypothesis, $H^1_{\GG/\FF}(K_\ell,T)$ is an $R$ of length $k$ whose $\m^{k-1}$-torsion induces a non-trivial quptient. Hence, the structure theorem implies that 
\[H^1_{\GG/\FF}(K_\ell,T)\cong R\qedhere\]
\end{proof}


\begin{remark}
If we choose a Kolyvagin prime, or any other prime $\ell$ such that $H^1_{/\FF}(K_\ell,T)\cong R$, the Selmer structure $\GG=\FF^\ell$ is cartesian with $\chi(\FF^\ell)=1$.
\end{remark}


Now, we describe a process in which Kolyvagin systems for $\GG$ describe the Selmer module $H^1_{\FF}(K,T)$. In order to do that, we need to localise the Kolyvagin systems at the prime at which $\FF$ and $\GG$ differ.

\begin{definition}
Let $\FF\leq \GG$ be two Selmer structures with $\chi(\FF)=0$ and $\chi(\GG)=1$ differing at the prime $\ell$ and let $\kappa\in \KS(\GG)$. Define the quantities $\delta$ associated to $\kappa$ by
\[\delta_n(\kappa,\FF):=\loc_\ell(\kappa_n)\in H^1_{\GG}(K_\ell, T)\bigm/H^1_{\FF}(K_\ell, T)\ \forall n\in \NN\]
\label{def:delta}
\end{definition}

The quantities $\delta_n$ can be used to define the $\Theta$ ideals of rank $0$.


\begin{definition}
Let $\FF\leq \GG$ be two Selmer structures such that $\chi(\FF)=0$ and $\chi(\GG)=1$ differing at the prime $\ell$ and let $\kappa\in \KS(\GG)$. We can define the ideals of $R$
\[\Theta_i^{(0)}(\kappa,\FF):=\sum_{n\in \NN_i}\ind\biggl(\delta_n(\kappa,\FF),H^1_{\GG/\FF}(K_\ell, T)\biggr)\]
\end{definition}

The comparison between the ideals $\Theta_i^{(0)}(\kappa,\FF)$ and the Fitting ideals of $H^1_{\FF}(K,T)$ leads to the first main result of this thesis.

\begin{theorem}
Let $R$ be a principal, artinian, local ring with finite residue field, let $T$ be an $R[[G_K]]$-module, unramified only at finitely many places which is free and finitely generated as an $R$-module. Assume $\FF\leq \GG$ are cartesian Selmer structures on $T$ satisfying that $\chi(\FF)=0$ and $\chi(\GG)=1$. If $\kappa\in \KS(\GG)$ is a primitive Kolyvagin system, then 
\begin{equation}
\Theta_i^{(0)}(\kappa,\FF)\subset\Fitt_i^R\Bigl(H^1_{\mathcal F}(K, T)\Bigr)\ \forall i\in \Z_{\geq 0}
\label{eq:theta}
\end{equation}
 Moreover, if one of the following conditions is satisfied
\begin{enumerate}[(i)]
\item $i\leq\dim_k\biggl(H^1_{\FF}(K,T)\biggm/H^1_{\FF}(K,T)[\m^{k-1}]\biggr)=:r$
\item $\Theta_{i-1}^{(0)}(\kappa,\FF)\subsetneq \Fitt_{i-1}^R\left(H^1_{\mathcal F}(K, T)\right)$
\item There is some $k\in \N$ and some $n\in\mathcal N$ such that $\nu(n)=i-1$, $\Theta_{i-1}(\kappa)=\delta_n R$ and 
$$H^1_{\mathcal F(n)}(\Q,T)\approx R/\m^{e_1}\times \cdots \times R/\m^{e_s}$$
for some $e_1>e_2\geq \cdots\geq e_s$.
\end{enumerate}
then the equality $\Theta_i^{(0)}(\kappa,\FF)=\Fitt_i^R\Bigl(H^1_{\mathcal F}(\mathbb Q, T)\Bigr)$ holds.
\label{th:str_chi0}
\end{theorem}

Similarly to the results in \textsection \ref{sec:core_rank_one}, the proof is divided in the following lemmas, which will be proven in the next subsections.


\begin{lemma}
If $\kappa\in \KS(\GG)$ is a primitive Kolyvagin system and $n\in \NN$, then
\[\ind(\delta_n)=\Fitt_0^R\Bigl(H^1_{\FF(n)}(K,T)\Bigr)\]
\label{lem:deltan_trn}
\end{lemma}

\begin{lemma}
If $\FF$ is a cartesian 
\[\sum_{n\in \NN_i}\Fitt_0^R\Bigl(H^1_{\FF^*(n)}(K,T^*)\Bigr)\subset \Fitt_i^R\Bigl(H^1_{\FF^*}(K,T^*)\Bigr)\]
\label{lem:fitti_sub_fitt0trn}
\end{lemma}

\begin{proof}
Analogous to the similar inclusion in Lemma \ref{lem:fitti_fitt0trn}.
\end{proof}

In order to prove the other inclusion, whenever it holds, we will construct a vertex $n_i\in \NN_i$ such that 
\[\Fitt_0^R\Bigl(H^1_{\FF^*(n_i)}(K,T^*)\Bigr)= \Fitt_i^R\Bigl(H^1_{\FF^*}(K,T^*)\Bigr)\]

Note that the equality holds trivially when $i<r$, since $\Fitt_i^R\Bigl(H^1_{\FF^*}(K,T^*)\Bigr)$ vanishes. For $i=r$, the equality is proven in the following lemma.

\begin{lemma}
There exists some vertex $n_r\in \NN_r$ such that 
\[\Fitt_0^R\Bigl(H^1_{\FF^*(n_r)}(K,T)\Bigr)=\Fitt_i^R\Bigl(H^1_{\FF}(K,T)\Bigr)\]
\label{lem:fitti_eq_fitt0_rank}
\end{lemma}

Lemma \ref{lem:fitti_eq_fitt0_rank} completes the proof of the equality $\Theta_i^{(0)}(\kappa,\FF)=\Fitt_i^R\Bigl(H^1_{\mathcal F}(\mathbb Q, T)\Bigr)$ under assumption (i).

Note that Lemma \ref{lem:fitti_eq_fitt0_rank} determines the structure of the modified Selmer group. Indeed, assume there is a structural homomorphism
\[H^1_{\FF}(K,T)\approx R^r\times R/\m^{e_1}\times \cdots \times R/\m^{e_s}\]
for some exponents $e_1\geq \cdots \geq e_s$, all being at most $k-1$. Then there is an homomorphism 
\[H^1_{\FF(n_r)}(K,T)\approx  R/\m^{e_1}\times \cdots \times R/\m^{e_s}\]
We can extend this construction to higher values of $i$.

\begin{lemma}
For every $j\geq 0$, we can either construct a vertex
\begin{itemize}
\item $n_{r+j}\in \NN_{r+j}$ such that $H^1_{\FF(n_{r+j})}(K,T)\cong R/\m^{e_{j+1}}\times \cdots \times R/\m^{e_s}$.
\item $n_{r+j+1}\in \NN_{r+j+1}$ such that $H^1_{\FF(n_{r+j+1})}(K,T)\cong R/\m^{e_{j+1}}\times \cdots \times R/\m^{e_s}$.
\end{itemize}
\label{lem:fitti_eq_fitt0_higher}
\end{lemma}

Note that such vertices guarantee the equality in Theorem \ref{th:str_chi0} for their respective indices.

\begin{corollary}
For the indices $k\geq 0$ such that there exists an element $n_r\in\NN_r$ as in Lemma \ref{lem:fitti_eq_fitt0_higher}, there is an equality
\[\sum_{n\in \NN_{r+k}}\Fitt_0^R\Bigl(H^1_{\FF^*(n)}(K,T^*)\Bigr)\subset \Fitt_{r+k}^R\Bigl(H^1_{\FF^*}(K,T^*)\Bigr)\]
\label{cor:fitti_eq_fitt0_higher}
\end{corollary}

Corollary \ref{cor:fitti_eq_fitt0_higher} proves the equality $\Theta_i^{(0)}(\kappa,\FF)=\Fitt_i^R\Bigl(H^1_{\mathcal F}(\mathbb Q, T)\Bigr)$ under assumption (i). Indeed, if such equality does not hold for some index $i-1=r+j-1$, the vertex $n_{r+j-1}$ in Lemma \ref{lem:fitti_eq_fitt0_higher} cannot be constructed. Hence, Lemma \ref{lem:fitti_eq_fitt0_higher} guarantees the existence of the vertex $n_{r+j}$, which proves the equality for the index $i=r+j$ by Corollary \ref{cor:fitti_eq_fitt0_higher}.

This concludes the proof of Theorem \ref{th:str_chi0}. The inclusion is a direct consequence of Lemmas \ref{lem:deltan_trn} and \ref{lem:fitti_sub_fitt0trn}. Lemma \ref{lem:fitti_eq_fitt0_rank} and Corollary \ref{cor:fitti_eq_fitt0_higher} proves the equality under the first or second assumption. If we are in the situation of the third assumption for some index $i-1=r+j-1$, the proof of Lemma \ref{lem:fitti_eq_fitt0_higher} shows that we can construct the vertex $n_{r+j}\in \NN_{r+j}$ satisfying that 
\[H^1_{\FF(n_{r+j})}(K,T)\cong R/\m^{e_{j+1}}\times \cdots \times R/\m^{e_s}\]
This construction proves the equality for the index $i=r+j$.

Theorem \ref{th:str_chi0} requires $\GG$ to be a primitive Kolyvagin system. This fact can also be checked from the quantities $\delta_n$.

\begin{theorem}
Let $R$ be a principal, artinian, local ring with finite residue field, let $T$ be an $R[[G_K]]$-module, unramified only at finitely many places which is free and finitely generated as an $R$-module. Assume $\FF\leq \GG$ are cartesian Selmer structures on $T$ satisfying that $\chi(\FF)=0$ and $\chi(\GG)=1$. If $\kappa\in KS(\GG)$, then $\kappa$ is primitive if and only if there is $n\in \NN$ such that 
\[\ind\Bigl(\delta_n(\kappa,\FF),H^1_{\GG/\FF}(K,T)\Bigr)=R\]
\label{th:delta_primitive}
\end{theorem}

\begin{proof}
If $\kappa$ is not primitive, there is a primitive Kolyvagin systems $\kappa'$ such that $\kappa=a \kappa'$ and $a\in R\setminus R^\times$. Then 
\[\ind\Bigl(\delta_n(\kappa,\FF)\Bigr)=(a)\ind\Bigl(\delta_n(\kappa',\FF)\Bigr)\subset a R\]

Conversely, if $\kappa$ is primitive, Lemma \ref{lem:deltan_trn} reduces the proof to finding some $n\in \NN$ such that 
\[H^1_{\FF(n)}(K,T)=0\]
By Proposition \ref{prop:selmer_quotient}, this is equivalent to 
\[H^1_{\FF(n)}(K,T\otimes k)=0\]
Such $n\in \N$ can be obtained by an inductive application of Lemma \ref{lem:transverse_reduction_free} on the $k$ representation $T\otimes k$.
\end{proof}


\subsection{Proof of Lemma \ref{lem:deltan_trn}}

The proof of Lemma \ref{lem:deltan_trn} involves comparing the indices of $\kappa_n$ and $\delta_n$, so we can then apply Lemma \ref{lem:kn_trn}. 


\begin{lemma}
For every $n\in\mathcal N$, let 
\[C_n:=\coker\left(\loc_\ell:\ H^1_{\GG}(K,T)\to H^1_{\GG/\FF}(K_\ell, T)\right)\]
Then $\ind(\delta_n)=\ind(\kappa_n)\cdot \Fitt^{(0)}(C_n)$.
\label{lem:delta_kappa_C}
\end{lemma}

\begin{proof}
Note that Proposition \ref{prop:core_rank_str} and Proposition \ref{prop:rank_modified} implies the existence of a non-canonical isomorphism
\begin{equation}
H^1_{\GG(n)}(K,T)\approx R\oplus H^1_{\GG^*(n)}(K,T^*)
\label{eq:selmerG_decomp}
\end{equation}
Let $(x_n,y_n)$ be the components of $\kappa_n$ under this identification. By Lemma \ref{lem:kn_trn}, $\ind(\kappa_n)=\Fitt_0^R\Bigl( H^1_{\FF^*}(K,T^*)\Bigr)$, then $y_n=0$ and $x_n$ is a generator of $\ind(\kappa_n)$. The decomposition in \eqref{eq:selmerG_decomp} induces a map in $R^+$ defined by
\[\xymatrix{R\ar[r] & H^1_{\GG}(K,T)\ar[r]^{\loc_\ell} & \ar[r] H^1_{\GG/\FF}(K_\ell,T) \ar[r]&R}\]
The composite map is the multiplication by some $a\in R$, which is also a generator of \(\Fitt^0(C_n)\). Therefore,
\[\ind(\delta_n)=\ind(\kappa_n) \Fitt_0^R(C_n)\]

\end{proof}

\begin{proof}[Proof of Lemma \ref{lem:deltan_trn}]
The exact sequence in Proposition \ref{prop:global_duality} induces a short exact sequence
\[\xymatrix{0\ar[r] & C_n \ar[r]  & H^1_{\FF^*(n)}(K,T)^\vee\ar[r] & H^1_{\GG^*(n)}(K,T)^\vee\ar[r] & 0}\]
We then have the identity of Fitting ideals
\[\Fitt_0^R\Bigl(H^1_{\FF^*(n)}(K,T)\Bigr)=\Fitt_0^R\Bigl(C_n\Bigr)\Fitt_0^R\Bigl(H^1_{\GG^*(n)}(K,T)\Bigr)=\Fitt_0^R\Bigl(C_n\Bigr)\ind(\kappa_n)=\ind(\delta_n)\]
where the second inequality follows from Lemma \ref{lem:kn_trn} and the last one from Lemma \ref{lem:delta_kappa_C}.
\end{proof}





\subsection{Proof of Lemma \ref{lem:fitti_eq_fitt0_rank}}

The proof is obtained as an inductive application of Lemma \ref{lem:transverse_reduction_free}. By the structure theorem of finitely generated $R$-modules, and the definition of core rank, there are non-canonical homomorphisms
\[H^1_{\FF}(K,T)=H^1_{\FF^*}(K,T^*)=R^r\times R/\m^{e_1}\times \cdots \times R/\m^{e_s}\]

We will construct inductively a vertex $n_i\in \NN_i$, where $i\leq r$, such that 
\[H^1_{\FF(n_i)}(K,T)\approx H^1_{\FF^(n_i)}(K,T^*)\approx R^{r-i}\times R/\m^{e_1}\times \cdots \times R/\m^{e_s}\]

Indeed, assume we have constructed $n_i\in \NN_i$ for some $i\leq r-1$. Clearly, $H^1_{\FF(n_i)}(K,T)$ contains a submodule isomorphic to $R$, so Lemma \ref{lem:transverse_reduction_free} implies the existence of a prime $\ell_{i+1}$ such that for $n_{i+1}=n_i\ell_{i+1}$, we get that 
\[H^1_{\FF^(n_{i+1})}(K,T^*)\approx R^{r-(i+1)}\times R/\m^{e_1}\times \cdots \times R/\m^{e_s}\]
Since $\chi(\FF)=0$, there is a non-canonical homomorphism
\[H^1_{\FF(n_{i+1})}(K,T)\approx H^1_{\FF^*(n_{i+1})}(K,T^*)\approx R^{r-(i+1)}\times R/\m^{e_1}\times \cdots \times R/\m^{e_s}\]

\subsection{Proof of Lemma \ref{lem:fitti_eq_fitt0_higher}}

Note that the element $n_r\in \NN_r$ was already constructed in Lemma \ref{lem:fitti_eq_fitt0_rank}. An inductive application of Lemma \ref{lem:transverse_reduction_torsion}, implies the existence of an element $n_{r+j}\in \NN_{r+j}$ such that 
\[H^1_{\FF(n_{r+j})}(K,T)\approx H^1_{\FF^*(n_{r+j})}(K,T^*)\approx R/\m^{t_{j+1}}\times R/\m^{e_{j+2}}\times \cdots \times R/\m^{e_s}\]
where $t_{j+1}\geq e_{j+1}$. We assume that we construct this elements minimising the exponents $t_{j+1}$ in every step.

For an index $j$, if $t_{j+1}=e_{j+1}$, then the element $n_{r+j}\in \NN_{r+j}$ satisfy the hypothesis of Lemma \ref{lem:fitti_eq_fitt0_higher}. 

Otherwise, if $t_{j+1}>e_{j+1}$, then $t_{j+1}>e_{j+2}$ as well , so the last remark in Lemma \ref{lem:transverse_reduction_torsion} guarantees the existence of a prime $\ell_{r+j+1}$ such that, when $n_{r+j+1}=n_{r+j}\ell_{r+j+1}$, the exponent $t_{j+2}$ coincides with $e_{j+2}$. By the minimality assumption on $t_{j+2}$, we know that 
\[H^1_{\FF(n_{r+j+1})}(K,T)\approx H^1_{\FF^*(n_{r+j+1})}(K,T^*)\approx R/\m^{e_{j+2}}\times \cdots \times R/\m^{e_s}\]

\section{Non-self dual Galois representations of rank $0$}

This section is devoted to the computation of the Fitting ideals of a Selmer module $H^1_{\FF}(K,T)$ of a cartesian Selmer structure of core rank $\chi(\FF)=0$, defined over a Galois representation $T$ satisfying Assumptions \ref{ass:nd}.

\begin{theorem}
Let $R$ be a principal, artinian, local ring with finite residue field, let $T$ be an $R[[G_K]]$-module unramified only at finitely many places, and let $\FF\leq \GG$ be a cartesian Selmer structures on $T$ satisfying Assumptions \ref{ass:nd} and $\chi(\GG)=1$. If $\kappa\in \KS(\GG)$ is a primitive Kolyvagin system, then 
\[\Theta_i^{(0)}(\kappa,\FF)=\Fitt_i^R(H^1_\FF(K,T))\]
\label{th:str_chi0_nd}
\end{theorem}

Note that the inclusion 
\[\Theta_i^{(0)}(\kappa,\FF)\subset \Fitt_i^R(H^1_\FF(K,T))\]
is proven by theorem \ref{th:str_chi0}. In addition, Lemma \ref{lem:deltan_trn} reduce the proof of the other inclusion to the following lemma.

\begin{lemma}
For every $i\in \Z_{\geq 0}$, there exists some vertex $n_i\in \NN_i$ such that 
\[\Fitt_0^R\Bigl(H^1_{\FF^*(n_i)}(K,T)\Bigr)=\Fitt_i^R\Bigl(H^1_{\FF}(K,T)\Bigr)\]
\label{lem:fitti_eq_fitt0_nd}
\end{lemma}

\begin{proof}
Assume we we have a structural homomorphism
\[H^1_{\FF}(K,T^*)\approx R^r\times R/\m^{e_1}\times \cdots R/\m^{e_s}\]

An inductive application of Lemma \ref{lem:transverse_reduction_nd} (see also Remark \ref{rem:rk1_free}) construct vertices $n_i\in \NN_i$ such that 
\[\begin{aligned}
H^1_{\FF(n_i)}(K,T)\cong R^{r-i} \times R/\m^{e_1}\times \cdots R/\m^{e_s}\ \textrm{for }i\leq r\\
H^1_{\FF(n_{r+j})}(K,T)\cong R/\m^{e_{j+1}}\times \cdots R/\m^{e_s}\ \textrm{for }j\geq 1
\end{aligned}\]
This proves the other inclusion, what concludes the proof of this lemma.

\end{proof}



\section{Selmer groups over discrete valuation rings}


\textcolor{red}{this section has not been written yet. These are materials from the paper}



\begin{remark}
Assume either
\begin{itemize}
\item $R$ is a discrete valuation ring.
\item $\length(R)\geq \length\left(H^1_\FF(\Q,T)_{\tors}\right)$.
\end{itemize}
Then $\rank_R(H^1_{\FF^*}(\Q,T^*)\du)$ is the minimal $i$ such that $\Theta_i\neq 0$.
\end{remark}

Provided that we know the ideals $\Theta_i(\kappa)$, theorem \ref{th:kur_par} determines the Fitting ideals of the dual Selmer group
\begin{corollary}
Assume $R$ is a discrete valuation ring and write $\Theta_i(\kappa)=\mathfrak m^{n_i}$. Then
$$\Fitt_i^R\left(H^1_{\mathcal F^*}(\mathbb Q, T^*)\du\right)=\mathfrak m^{\min\left\{n_i,\frac{n_{i+1}+n_{i-1}}{2}\right\}}$$
\label{cor:kur_par}
\end{corollary}

\begin{proof}
Write $\Fitt_i^R\left(H^1_{\mathcal F^*}(\mathbb Q, T^*)\du\right)=\m^{m_i}$, so the inequality in theorem \ref{th:kur_par} implies that $n_i\geq m_i$. By the structure theorem of finitely generated modules over principal ideal domains, the following inequality holds for $i\in \N$:

\[m_{i+1}-m_i\geq m_i-m_{i-1}\]
Hence the index $m_i$ can be upper bounded using $m_{i-1}$ and $m_{i+1}$:
\begin{equation}
m_i\leq \frac{m_{i+1}+m_{i-1}}{2}
\label{eq:Fitting_ineq}
\end{equation}


Assume that $n_i=m_i$. Then 
\[m_i\leq \frac{m_{i+1}+m_{i-1}}{2}\leq \frac{n_{i+1}+n_{i-1}}{2}\Rightarrow m_i=\min\left\{n_i,\frac{n_{i+1}+n_{i-1}}{2}\right\}\]

Assume that $n_i> m_i$. Theorem \ref{th:kur_par} can be applied to $i+1$. Since condition (ii) in this theorem holds by our assumption, we obtain that $m_{i+1}=n_{i+1}$. On the other hang, assume by contradiction that $n_{i-1}>m_{i-1}$. In this case, theorem \ref{th:kur_par}, would imply that $n_i=m_i$, contradicting our assumption. Therefore, $n_{i-1}=m_{i-1}$. Moreover, condition (iii) in theorem \ref{th:kur_par} cannot be satisfied since, otherwise, our assumption would not be true. Hence, the equality holds in equation \eqref{eq:Fitting_ineq} and we obtain
\[m_i=\frac{m_{i+1}+m_{i-1}}{2}=\frac{n_{i+1}+n_{i-1}}{2}=\min\left\{n_i,\frac{n_{i+1}+n_{i-1}}{2}\right\}\]
\end{proof}









Under those assumptions, the following improvement of theorem \ref{th:kur_par} is true. Sakamoto proves the equality \eqref{eq:theta_nd} below under stronger assumptions when the coefficient ring $R$ is a Gorenstein ring of dimension zero. In particular, R. Sakamoto's result only worked when $H^1_{\FF}(\Q,T)=0$. However, when $R$ is a principal ring, we can weaken the assumptions to obtain the following result.

\begin{theorem}
Let $(T,\FF,\PP)$ be a Selmer triple satisfying \ref{Hffr}-\ref{Hprimes} and \ref{Nsd}-\ref{Nsur}. Then for every $i\in \Z_{\geq 0}$, the following equality is satisfied:
\begin{equation}
\Theta_i(\kappa)=\Fitt_i^R\left(H^1_{\mathcal F^*}(\mathbb Q, T^*)\du\right)
\label{eq:theta_nd}
\end{equation}
\label{th:kur}
\end{theorem}



