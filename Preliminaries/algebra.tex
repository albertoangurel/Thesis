\chapter{Algebraic Preliminaries}


\section{Fitting ideals}


\begin{definition}

Let $M$ be a finitely presented $R$-module. Choose a resolution
\[\xymatrix{R^n\ar[r]^{A} & R^m\ar[r] & M \ar[r] & 0}\]
where the map $R^n\to R^m$ is represented by the matrix $A$. For every $i\geq 0$, we define the $i^{\textrm{th}}$ Fitting ideal $\Fitt_i^R(M)$ is the ideal in $R$ generated by the minors of size $(m-i)$ of $A$.
\end{definition}

\begin{remark}
The $i^{\textrm{th}}$ Fitting ideal coincides with the image of the following map, induced by the matrix $A$.
\[\Fitt_i^R(M)=\textrm{Im}\Biggl(\bigwedge^{m-i} R^n\to \bigwedge^{m-i} R^m\Biggr)\]
\end{remark}

It can be shown that Fitting ideals are well defined.
\begin{proposition}(\cite[Corollary 20.4]{Eisenbud})
The Fitting ideals $\Fitt_i^R(M)$ are independent of the chosen resolution.
\end{proposition}

When the coefficient ring $R$ is simple enough, the sequence of Fitting ideals can recover $M$ up to pseudo-isomorphism.
\begin{proposition}
Let $R$ be a either a discrete valuation ring or a quotient of it, and let $M$ be a finitely generated $R$-module. Then the Fitting ideals determine $M$ up to isomorphism.
\label{prop:fitting_dvr}
\end{proposition}

\begin{proof}
Let $\m$ be the maximal ideal of $R$. The structure theorem of finitely generated modules over principal ideal domains implies that there are integers $r$, $s$ and $\alpha_1\geq\ldots\geq \alpha_s$ such that 
\[M\approx R^r \times R/\m^{\alpha_1}\times\cdots \times R/\m^{\alpha_s}\]

Then $M$ admits a resolution 
\[\xymatrix{R^{r+s}\ar[r]^{A} & R^{r+s}\ar[r] & M \ar[r] & 0}\]
where the matrix $A$ is given by 
\[A=\begin{pmatrix}\begin{array}{c;{2pt/2pt}c} \textbf{0}_{r\times r} & \textbf{0}_{s\times r} \\\hdashline[2pt/2pt]\textbf{0}_{r\times s} & \begin{matrix} \pi^{\alpha_1}\\ & \ddots \\ & & \pi^{\alpha_s} \end{matrix}\end{array}\end{pmatrix}\]
where $\pi$ is a generator of $\m$. We can then compute,
\begin{itemize}
\item $i\in \{0,\ldots, r-1\}\Rightarrow \Fitt_i(M)=(0)$
\item $j\in \{0,\ldots,s-1\}\Rightarrow \Fitt_{r+j}=\prod_{k=j+1}^s \m^{i_k} =\m^{\sum_{k=j+1}^s i_k}$
\item $i\geq r+s\Rightarrow \Fitt_i(M)=(1)$.
\end{itemize}

The Fitting ideals determine $M$ up to isomorphism, since
\begin{itemize}
\item $r$ is the minimum $i$ such that $\Fitt_i(M)\neq 0$.
\item For $i\geq 0$, $\alpha_i=\Fitt_{r+i+1}(M)\Fitt_{r+i}(M)^{-1}$. 
\end{itemize}
\end{proof}




\section{Preliminaries on exterior powers}

The goal of this section is to introduce the necessary theory of exterior biduals.

\begin{definition}
Let $R$ be a ring, let $M$ be an $R$-module and let $r\in M$. The $r^{\textrm{th}}$ \emph{exterior power} 
\[\bigwedge^r M\]
is defined as the quotient of the tensor product $M^{\otimes r}$ by the submodule generated by the elements of the form
\[m_1\otimes \cdots \otimes m_i\otimes \cdots\otimes m_j\otimes \cdots\otimes m_r+m_1\otimes \cdots \otimes m_j\otimes \cdots\otimes m_i\otimes \cdots\otimes m_r\]
\label{def:exterior_power}
\end{definition}

\begin{definition}
If $F$ is a free $R$-module of rank $s$, the determinant of $F$ is defined as 
\[\det(F)=\bigwedge^s F\]
\label{def:determinant}
\end{definition}

We now state two basic properties of the determinant.
\begin{proposition}
Let $A\subset B$ be free $R$-modules such that $B/A$ is also free. Then there is a canonical isomorphism
\[\det(A)\otimes \det(B/A)\cong \det(B)\]
\label{prop:det_quotient}
\end{proposition}

\begin{proposition}
Let $A$ be a free $R$-module. There is a canonical isomorphism
\[\det(A)\otimes \det(A^+)\cong R\]
\label{prop:det_dual}
\end{proposition}

\begin{definition}
Let $R$ be a ring, let $M$ be an $R$-module and let $r\in M$. The $r^{\textrm{th}}$ \emph{exterior bidual} is defined as
\[\bigcap^r M=\biggl(\bigwedge^r M^+\biggr)^+\]
\label{def:exterior_bidual}
\end{definition}

The properties of exterior biduals will be deduced assuming certain properties of the ring $R$.

\subsection{Exterior powers over self-injective rings}

The goal of this section is to introduce the necessary theory of exterior biduals over self injective ring.



\textcolor{red}{properties of self-injective rings}

\begin{proposition}
Let $R$ be a self-injective ring. The dual functor $M\mapsto M^+$ is exact.
\label{prop:selfinjective_dual_exact}
\end{proposition}

\begin{proposition}
If we have an exact sequence of $R$-modules over a self-injective ring $R$
\[\xymatrix{0\ar[r] & M\ar[r]^\mu & N\ar[r]^\varepsilon & R^s}\]
for some natural number $s$. If $r\geq s$ is another natural number, there is a canonical map
\[\phi_{N,M}:\ \bigcap^{r} N \to \bigcap^{r-s} M\]
\label{prop:bidual_map_selfinjective}
\end{proposition}

\begin{proof}
By Proposition \ref{prop:selfinjective_dual_exact}, there is another exact sequence
\[\xymatrix{R^s\ar[r]^{\varepsilon^+} & N^+\ar[r]^{\mu^+} & M+\ar[r] & 0}\]
For every $m\in M^+$, choose a lift $n\in N^+$. With this choices, there is a lifting map of sets 
\[\widetilde{\mu^+}:\  M^+\to N^+\]
Denote by $\{e_1,\ldots,e_s\}$ the canonical basis of $R_s$. Then define the map
\[\phi_{N,M}^*:\ \bigwedge^{r-s} M^+\to \bigwedge^r N^+:\ m_1\wedge\cdots \wedge m_{r-s} \mapsto \widetilde{\mu^+}(m_1)\wedge\cdots \wedge \widetilde{\mu^+}(m_{r-s})\wedge \varepsilon^+(e_1)\wedge\cdots\wedge\varepsilon^+(e_s)\]
One can chech that this definition is independend of the lifting choices made and an $R$-homomorphism.

The dual of this map induces a map
\[\phi_{N,M}:\ \bigcap^r N\to \bigcap^{r-s} M\]
\end{proof}

When the last term of the previous exact sequence is free, but not $R^s$, the map $\phi_{N,M}$ is no longer canonical unless we tensor with the determinant of the dual module.
\begin{proposition}
If we have an exact sequence of $R$-modules over a self-injective ring $R$
\[\xymatrix{0\ar[r] & M\ar[r]^\mu & N\ar[r]^\varepsilon & F}\]
where $F$ is a free $R$-module of rank $s$. If $r\geq s$ is a natural number, there is a canonical map
\[\phi_{N,M}:\ \bigcap^{r} N\otimes \det(F^+) \to \bigcap^{r-s} M\]
\label{prop:bidual_map_selfinjective_free}
\end{proposition}

\begin{proof}
As in Proposition \ref{prop:bidual_map_selfinjective}, we can construct a lifting map of sets
\[\widetilde{\mu^+}:\  \bigwedge^{r-s} M^+\to \bigwedge^{r-s} N^+\]
We can use it to construct a well-defined $R$-homomorphism
\[\phi_{N,M}^*:\ \bigwedge^{r-s} M^+\otimes \det(F^+)\to \bigwedge^{r-s} N^+,\ m\otimes f\mapsto \widetilde{\mu^+}(m)\otimes \varepsilon^+(f)\]
The dual map can be expressed as
\[\phi_{N,M}:\ \bigcap^{r-s} N\otimes \det(F^+)\to \bigcap^r M\]
\end{proof}


\subsection{Exterior biduals over discrete valuation rings}

When $R$ is a discrete valuation ring, we can prove an analogue of Propositions \ref{prop:bidual_map_selfinjective_free} and \ref{prop:bidual_map_selfinjective_free}.


\begin{proposition}
If we have an exact sequence of $R$-modules over a discrete valuation ring $R$
\[\xymatrix{0\ar[r] & M\ar[r]^\mu & N\ar[r]^\varepsilon & R^s}\]
for some natural number $s$. If $r\geq s$ is another natural number, there is a canonical map
\[\phi_{N,M}:\ \bigcap^{r} N \to \bigcap^{r-s} M\]
\label{prop:bidual_map_dvr}
\end{proposition}

\begin{proof}
The image of $\varepsilon$ is a submodule of $R^s$ so, in particular, it is torsion-free. Therefore, it is a free $R$-module, so $\Ext(\Im(\varepsilon),R)=0$. Hence there is an exact sequence
\[\xymatrix{R^{s}\ar[r] & N^+\ar[r] & M^+\ar[r] & 0}\]
We can use this exact sequence to construct the map $\phi_{N,M}$ as in Proposition \ref{prop:bidual_map_selfinjective}.
\end{proof}


Similarly, we can adapt the previous result to the case when the last term is a free $R$-module, non-canonically isomorphic to $R^s$.
\begin{proposition}
If we have an exact sequence of $R$-modules over a discrete valuation ring $R$
\[\xymatrix{0\ar[r] & M\ar[r]^\mu & N\ar[r]^\varepsilon & F}\]
where $F$ is a free $R$-module of rank $s$. If $r\geq s$ is a natural number, there is a canonical map
\[\phi_{N,M}:\ \bigcap^{r} M\otimes \det(F^+) \to \bigcap^{r-s} N\]
\label{prop:bidual_map_dvr_free}
\end{proposition}


\subsection{Exterior biduals over the Iwasawa algebra}
\begin{proposition}
Consider the exact sequence of $\Lambda$-modules
\[\xymatrix{0\ar[r] & N\ar[r] &M \ar[r] & \Lambda^s }\]
Then there is a canonical map 
\[\Phi: \bigcap^{r+s} M\to \bigcap^{r} N\]
\label{prop:bidual_map_iwasawa}
\end{proposition}

Since $\Lambda$ is not a self-injective ring, we need to prove some basic facts about the extension groups of $\Lambda$-modules before addressing the proof of Proposition \ref{prop:bidual_map}.


\begin{lemma}
Let $M$ be a submodule of $\Lambda^s$. Then $\Ext^1(M,\Lambda)$ is finite.
\label{lem:ext_finite}
\end{lemma}

\begin{proof}
For every $\Lambda$-module $M$, recall that $M^+:=\Hom(M,\Lambda)$. There is a canonical map
\[\Phi:\ M\to M^{++},\ m\mapsto \bigl(\varphi\in M^+\mapsto \varphi (a)\bigr)\]
Let $T_1(M)=\ker\Phi$. Since $M$ is contained in $\Lambda^s$, we claim that $T_1(M)=0$. Indeed, fix an inclusion $\iota:\ M\hookrightarrow  \Lambda^s$ and denote by $\pi^i:\ \Lambda^s\to \Lambda$ the projection at the $i^{\textrm{th}}$ coordinate. Let $m\in M\setminus\{0\}$. Then there is some $j\in \{1,\ldots,s\}$ such that the $j^{\textrm{th}}$ coordinate of $\iota(m)$ is non-zero. Then 
\[\Phi(m)(\pi^j\circ \iota)=\pi_j(\iota(m))\neq 0\]
Thus $m\notin T_1(M)$ and the proof of the claim is complete.

By \cite[Corollary 5.5.9]{NSW}, $\Ext^1(M,\Lambda)$ is finite.
\end{proof}

\begin{lemma}
Let $N\subset M$ be $\Lambda$ modules such that $N$ has finite index in $M$. Then $N^*=M^*$.
\label{lem:dual_fin_index}
\end{lemma}

\begin{proof}
There is an exact sequence 
\[\xymatrix{(M/N)^*\ar[r] & M^* \ar[r] & N^*\ar[r] & \Ext^1(M/N,\Lambda)}\]
Since $M/N$ is finite, then both $(M/N)^*$ and $\Ext^1(M/N,\Lambda)$ vanish. Indeed, $(M/N)^*=0$ since $\Lambda$ does not contain elements of finite order and $\Ext^1(M/N,\Lambda)=0$ by \cite[Corollary 5.5.4]{NSW}
\end{proof}



\begin{proof}[Proof of proposition \ref{prop:bidual_map}]
Let $I$ be the image of the map $N\to M$ inside $\Lambda^s$. By lemma \ref{lem:ext_finite}, $\Ext^1(I,\Lambda)$ is finite. There is an exact sequence
\[\xymatrix{\Lambda^s\ar[r] & M^* \ar[r] & N^*\ar[r] & \Ext^1(I,\Lambda)}\]
Call $J$ to the image of $M^*$ inside $N^*$, which has finite index in $N^*$ because of the finiteness of $\Ext^1(I,\Lambda)$. The exact sequence
\[\xymatrix{\Lambda^s\ar[r] & M^* \ar[r] &J\ar[r] & 0}\] 
induces a canonical map
\[\bigwedge^r J\to \bigwedge^{r+s} M^*\]
The dual of this map is
\[\bigcap^{r+s} M^*\to \Hom\left(\bigwedge^r J,\Lambda\right)\]
Since $\bigwedge^r J$ has finite index in $\bigwedge^r N^*$, lemma \ref{lem:dual_fin_index} implies that their duals are equal, so the above map can be rewritten as
\[\bigcap^{r+s} M^*\to \bigcap^{r} N^*\]
\end{proof}

\begin{proposition}
Consider the exact sequence of $\Lambda$-modules
\[\xymatrix{0\ar[r] & N\ar[r] &\Lambda^{r+s}\ar[r] & \Lambda^s \ar[r] & M\ar[r] & 0}\]
Let $\varphi$ be a generator of $\displaystyle \bigcap^{r+s} \Lambda^{r+s}$ and let
\[\Phi:\ \bigcap^{r+s} \Lambda^{r+s} \to \bigcap^{r} N\] 
be the map constructed in proposition \ref{prop:bidual_map}. Then the image of $\displaystyle \Phi(\varphi)\in \Hom\left(\bigwedge^{r} N^*,\Lambda\right)$ contains the $0^{th}$ Fitting ideal of $M$.
\label{prop:bidual_fitting}
\end{proposition}

\begin{proof}
Let $\Psi$ be the composition of the following maps
\[\xymatrix{\bigwedge^{r} (\Lambda^{r+s})^*\ar[r]& \bigwedge^{r} N^* \ar[r] & \bigwedge^{s+t} (\Lambda^{r+s})^* \ar[r]^{\varphi} &\Lambda}\]
where the first map is induced by the homomorphism $(\Lambda^{r+s})^*\to N^*$ and has finite cokernel, the second map is the one constructed in the proof of \ref{prop:bidual_map}. Since $\Phi(\varphi)$ is the composition of the last two maps, its image contains the image of $\Psi$ with finite index. The proof concludes by noticing the image of $\Psi$ coindides with the $0^{\textrm{th}}$ Fitting ideal of $M$.
\end{proof}  

