\chapter{Kolyvagin systems over the Iwasawa algebra}

%The results on the previous chapter were limited in two different directions: the coefficient ring $R$ was required to be principal and the core rank was at most one. These two assumptions were relaxed in the work of D.~Burns, R.~Sakamoto and T.~Sano. In this work, Kolyvagin systems are redefined as collections of classes in the bidual exterior powers of Selmer groups and can be used to bound the Fitting ideals of the Selmer group. However, Kolyvagin systems in this setting do not determine all Fitting ideals of the Selmer group, and one need to introduce the notion of Stark system in order to do that.

The results in previous chapters were limited to principal rings. Here, we generalise previous results to compute the Fitting ideal over the Iwasawa algebra $\Lambda:=\Z_p[[T]]$. For that we need to consider Kolyvagin systems consisting of collections of classes in exterior biduals of Selmer groups, as defined in \cite{BurnsSakamotoSano2}. In this chaper, we combine this setting with the patched Selmer groups defined in Chapter \ref{ch:patched} following \cite{Sweeting}. This new setting presents some technical complications, since the coefficient ring $\Lambda$ is no longer self-injective, a condition required in the construction in \cite{BurnsSakamotoSano2}.

\section{Preliminaries on exterior powers}

The goal of this section is to introduce the necessary background to construct systems in the exterior biduals of the Selmer groups. The all have in common that they are constructed from different versions of maps of the following kind.

\begin{proposition}
Consider the exact sequence of $\Lambda$-modules
\[\xymatrix{0\ar[r] & N\ar[r] &M \ar[r] & \Lambda^s }\]
Then there is a canonical map 
\[\Phi: \bigcap^{r+s} M\to \bigcap^{r} N\]
\label{prop:bidual_map}
\end{proposition}

Since $\Lambda$ is not a self-injective ring, we need to prove some basic facts about the extension groups of $\Lambda$-modules before addressing the proof of Proposition \ref{prop:bidual_map}.


\begin{lemma}
Let $M$ be a submodule of $\Lambda^s$. Then $\Ext^1(M,\Lambda)$ is finite.
\label{lem:ext_finite}
\end{lemma}

\begin{proof}
For every $\Lambda$-module $M$, recall that $M^+:=\Hom(M,\Lambda)$. There is a canonical map
\[\Phi:\ M\to M^{++},\ m\mapsto \bigl(\varphi\in M^+\mapsto \varphi (a)\bigr)\]
Let $T_1(M)=\ker\Phi$. Since $M$ is contained in $\Lambda^s$, we claim that $T_1(M)=0$. Indeed, fix an inclusion $\iota:\ M\hookrightarrow  \Lambda^s$ and denote by $\pi^i:\ \Lambda^s\to \Lambda$ the projection at the $i^{\textrm{th}}$ coordinate. Let $m\in M\setminus\{0\}$. Then there is some $j\in \{1,\ldots,s\}$ such that the $j^{\textrm{th}}$ coordinate of $\iota(m)$ is non-zero. Then 
\[\Phi(m)(\pi^j\circ \iota)=\pi_j(\iota(m))\neq 0\]
Thus $m\notin T_1(M)$ and the proof of the claim is complete.

By \cite[Corollary 5.5.9]{NSW}, $\Ext^1(M,\Lambda)$ is finite.
\end{proof}

\begin{lemma}
Let $N\subset M$ be $\Lambda$ modules such that $N$ has finite index in $M$. Then $N^*=M^*$.
\label{lem:dual_fin_index}
\end{lemma}

\begin{proof}
There is an exact sequence 
\[\xymatrix{(M/N)^*\ar[r] & M^* \ar[r] & N^*\ar[r] & \Ext^1(M/N,\Lambda)}\]
Since $M/N$ is finite, then both $(M/N)^*$ and $\Ext^1(M/N,\Lambda)$ vanish. Indeed, $(M/N)^*=0$ since $\Lambda$ does not contain elements of finite order and $\Ext^1(M/N,\Lambda)=0$ by \cite[Corollary 5.5.4]{NSW}
\end{proof}



\begin{proof}[Proof of proposition \ref{prop:bidual_map}]
Let $I$ be the image of the map $N\to M$ inside $\Lambda^s$. By lemma \ref{lem:ext_finite}, $\Ext^1(I,\Lambda)$ is finite. There is an exact sequence
\[\xymatrix{\Lambda^s\ar[r] & M^* \ar[r] & N^*\ar[r] & \Ext^1(I,\Lambda)}\]
Call $J$ to the image of $M^*$ inside $N^*$, which has finite index in $N^*$ because of the finiteness of $\Ext^1(I,\Lambda)$. The exact sequence
\[\xymatrix{\Lambda^s\ar[r] & M^* \ar[r] &J\ar[r] & 0}\] 
induces a canonical map
\[\bigwedge^r J\to \bigwedge^{r+s} M^*\]
The dual of this map is
\[\bigcap^{r+s} M^*\to \Hom\left(\bigwedge^r J,\Lambda\right)\]
Since $\bigwedge^r J$ has finite index in $\bigwedge^r N^*$, lemma \ref{lem:dual_fin_index} implies that their duals are equal, so the above map can be rewritten as
\[\bigcap^{r+s} M^*\to \bigcap^{r} N^*\]
\end{proof}

\begin{proposition}
Consider the exact sequence of $\Lambda$-modules
\[\xymatrix{0\ar[r] & N\ar[r] &\Lambda^{r+s}\ar[r] & \Lambda^s \ar[r] & M\ar[r] & 0}\]
Let $\varphi$ be a generator of $\displaystyle \bigcap^{r+s} \Lambda^{r+s}$ and let
\[\Phi:\ \bigcap^{r+s} \Lambda^{r+s} \to \bigcap^{r} N\] 
be the map constructed in proposition \ref{prop:bidual_map}. Then the image of $\displaystyle \Phi(\varphi)\in \Hom\left(\bigwedge^{r} N^*,\Lambda\right)$ contains the $0^{th}$ Fitting ideal of $M$.
\label{prop:bidual_fitting}
\end{proposition}

\begin{proof}
Let $\Psi$ be the composition of the following maps
\[\xymatrix{\bigwedge^{r} (\Lambda^{r+s})^*\ar[r]& \bigwedge^{r} N^* \ar[r] & \bigwedge^{s+t} (\Lambda^{r+s})^* \ar[r]^{\varphi} &\Lambda}\]
where the first map is induced by the homomorphism $(\Lambda^{r+s})^*\to N^*$ and has finite cokernel, the second map is the one constructed in the proof of \ref{prop:bidual_map}. Since $\Phi(\varphi)$ is the composition of the last two maps, its image contains the image of $\Psi$ with finite index. The proof concludes by noticing the image of $\Psi$ coindides with the $0^{\textrm{th}}$ Fitting ideal of $M$.
\end{proof}

\section{Stark Systems}

Let $m,n\in \NN$ be square-free products of Kolyvagin ultraprimes such that $m\mid n$. There is an exact sequence
\[\xymatrix{0\ar[r] & \bH^1_{\FF^m}(\Q,T)\ar[r] & \bH^1_{\FF^n}(\Q,T)\ar[r] & \prod_{\ell\mid \frac{n}{m}} \bH^1_{s}(\Q,T)}\]
By proposition \ref{prop:bidual_map}, this exact sequence induces a map
\[\Phi_{n,m}:\ \bigcap^{r+\nu(n)} \bH^1_{\FF^n}(\Q,T)\to \bigcap^{r+\nu(m)} \bH^1_{\FF^m}(\Q,T)\]

\begin{lemma}
Let $\m,n,r\in \NN$ be square-free products of Kolyvagin ultraprimes such that $m\mid n\mid r$. Then
\[\phi_{r,m}=\phi_{r,n}\circ \phi_{n,m}\]
\label{lem:com} 
\end{lemma}

\begin{definition}
The set of Stark systems of $\FF$ is defined as the inverse limit
\[\bSS(\FF):=\varprojlim_{n\in \NN} \bigcap^{r+\nu(n)} \bH^1_{\FF^n}(\Q,T)\]
\end{definition}


\begin{theorem}
Let $n\in \NN$ be a core vertex. Then the projection map
\[\bSS(\FF)\to \bigcap^{r+\nu(n)} \bH^1_{\FF^n}(\Q,T)\]
is an isomorphism.
\label{th:stark_core_projection}
\end{theorem}

\begin{proof}
We only need to prove that if $n\in \NN$ is a core vertex and $\ell\in \PP$ does not divide $n$, the map
\[\bigcap^{r+\nu(n\ell)} \bH^1_{\FF^{n\ell}} (\Q,\bT)\to \bigcap^{r+\nu(n)} \bH^1_{\FF^n} (\Q,\bT)\]
is an isomorphism. This map is induced by the exact sequence
\[\xymatrix{0\ar[r] & \bH^1_{\FF^n}(\Q,\bT)\ar[r]&\bH^1_{\FF^{n\ell}}(\Q,\bT)\ar[r] & \bH^1_s(\Q_\ell, \bT) \ar[r] &0}\]
Since $\Ext^1(\Lambda,\Lambda)=0$, the dual map $\bH^1_{\FF^{n\ell}}(\Q,\bT)^+\to \bH^1_{\FF^n}(\Q,\bT)^+$ is surjective. Hence we can construct an injective map 
\[\bigwedge^{r+\nu(n)} \bH^1_{\FF^n}(\Q,\bT)^+ \to \bigwedge^{r+\nu(n\ell)} \bH^1_{\FF^{n\ell}}(\Q,\bT)^+\]
which turns out to be an isomorphism since both are free $\Lambda$-modules of rank $1$. Therefore, its dual map is also an isomorphism.

\textcolor{blue}{perhaps comment that $n\ell$ is also a core vertex}
\end{proof}

\subsection{Core vertex}

\begin{definition}
A \emph{core vertex} of rank $r$ is a square-free product of ultraprimes $n\in \NN$ such that $\bH^1_{\FF^*_n}(\Q,\bT^*)=0$ and $\bH^1_{\FF}(\Q,\bT)$ is a free $\Lambda$-module of rank $r+\nu(n)$.
\end{definition}

\begin{proposition}
Let $n\in \NN$ be a core vertex and let $m\in \NN$ be such that $n\mid m$. Then $m$ is also a core vertex.
\label{prop:core_vertex_div}
\end{proposition}

\begin{proof}
Since $n\mid m$, then $\bH^1_{\FF_m}(\Q,T)$ is contained in $\bH^1_{\FF_n}(\Q,T)$, so it also vanishes. The exact sequence
\[\xymatrix{0\ar[r] & \bH^1_{\FF^n}(\Q,\bT) \ar[r] & \bH^1_{\FF^m}(\Q,\bT) \ar[r] & \bigoplus_{\ku\mid \frac{m}{n}}\bH^1_s(\Q_{\ku},\bT)\ar[r] & 0}\]
The first and third terms of this exact sequence are free modules of ranks $r+\nu(n)$ and $\nu(m)-\nu(n)$. Hence $\bH^1_{\FF^m}(\Q,\bT)$ is free of rank $r+\nu(n)$.
\end{proof}

\begin{assumption}
    There exist an integer $r$ and $n\in \NN$ such that $\bH^1_{\FF^*}(\Q,\bT^*)=0$ and $\bH^1_{\FF^n}(\Q,\bT)$ is a free $\Lambda$-module of rank $r+\nu(n)$.
    \label{ass:core_vertex}
\end{assumption}

\begin{corollary}
The module of Stark systems $\bSS(\FF)$ is a free $\Lambda$-module of rank one.
\end{corollary}

\begin{theorem}
Let $\varepsilon=(\varepsilon)_{n\in \NN}$ be a generator of $\bSS(\FF)$. For every $m\in \NN$, the image of $\varepsilon_m\in \Hom\left(\bigwedge^{r+\nu(m)} \bH^1_{\FF^m}(K,\bT)^+,\Lambda\right)$ contains the $0^{\textrm{th}}$ Fitting ideal of $\bH^1_{\FF_m^*}(\Q,\bT^*)$ with finite index.
\end{theorem}

\begin{proof}
By assumption \ref{ass:core_vertex} and proposition \ref{prop:core_vertex_div}, there exists a core vertex $n$ such that $m\mid n$, which leads to the following exact sequence
\[\xymatrix{0\ar[r] & \bH^1_{\FF^n}(K,\bT) \ar[r] & \bH^1_{\FF^m}(K,\bT)\ar[r] & \prod_{\ku\mid n/m} \bH^1_\s(K_\ku,\bT) \ar[r] & \bH^1_{\FF^*_m}(K,\bT^*)\ar[r] & 0}\]
which induces a map 
\[\phi_{n,m}:\ \bigcap^{r+\nu(n)} \bH^1_{\FF^n}(K,\bT)\to \bigcap^{r+\nu(m)} \bH^1_{\FF^m}(K,\bT)\]

Since $\varepsilon$ generates $\bSS(\FF)$, Theorem \ref{th:stark_core_projection} implies that $\varepsilon_n$ generates $\bigcap^{r+\nu(n)}$. Since $\varepsilon_m=\phi_{n,m}(\varepsilon_n)$, then proposition \ref{prop:bidual_fitting} implies that the image of $\varepsilon_n$ contains $\Fitt^0(\bH^1_{\FF_m}(\Q,\bT^*))$ with finite index. 
\end{proof}

