\chapter{Kolyvagin systems over the Iwasawa algebra}

%The results on the previous chapter were limited in two different directions: the coefficient ring $R$ was required to be principal and the core rank was at most one. These two assumptions were relaxed in the work of D.~Burns, R.~Sakamoto and T.~Sano. In this work, Kolyvagin systems are redefined as collections of classes in the bidual exterior powers of Selmer groups and can be used to bound the Fitting ideals of the Selmer group. However, Kolyvagin systems in this setting do not determine all Fitting ideals of the Selmer group, and one need to introduce the notion of Stark system in order to do that.

The results in previous chapters were limited to principal rings. Here, we generalise previous results to compute the Fitting ideal over the Iwasawa algebra $\Lambda:=\Z_p[[T]]$. For that we need to consider Kolyvagin systems consisting of collections of classes in exterior biduals of Selmer groups, as defined in \cite{BurnsSakamotoSano2}. In this chaper, we combine this setting with the patched Selmer groups defined in Chapter \ref{ch:patched} following \cite{Sweeting}. This new setting presents some technical complications, since the coefficient ring $\Lambda$ is no longer self-injective, a condition required in the construction in \cite{BurnsSakamotoSano2}.



\section{Stark Systems}

Let $m,n\in \NN$ be square-free products of Kolyvagin ultraprimes such that $m\mid n$. There is an exact sequence
\[\xymatrix{0\ar[r] & \bH^1_{\FF^m}(K,T)\ar[r] & \bH^1_{\FF^n}(K,T)\ar[r] & \prod_{\ell\mid \frac{n}{m}} \bH^1_{\s}(K_\ell,T)}\]
By proposition \ref{prop:bidual_map}, this exact sequence induces a map
\[\Phi_{n,m}:\ \bigcap^{r+\nu(n)} \bH^1_{\FF^n}(K,T)\to \bigcap^{r+\nu(m)} \bH^1_{\FF^m}(K,T)\]

\begin{remark}
Note that the map $\Phi_{n,m}$ is dependent on a choice of an isomorphism $\bH^1_{\s}(\Q,T)\cong R$, or equivalently, an element in $\bH^1_{\s}(K_\ell,T)^\times$. From now on, we assume we have fixed such isomorphism for every $\ell\in \PP$.
\end{remark}

\begin{lemma}
Let $\m,n,r\in \NN$ be square-free products of Kolyvagin ultraprimes such that $m\mid n\mid r$. Then
\[\phi_{r,m}=\phi_{r,n}\circ \phi_{n,m}\]
\label{lem:com} 
\end{lemma}

\begin{proof}
\textcolor{red}{do}
\end{proof}

Therefore, the set of maps $\phi_{n,m}$ forms an inverse system, so it makes sense to consider the elements in the inverse limit.



\begin{definition}
The set of Stark systems of $\FF$ is defined as the inverse limit
\[\bSS(\FF):=\varprojlim_{n\in \NN} \bigcap^{r+\nu(n)} \bH^1_{\FF^n}(\Q,T)\]
\label{def:stark_systems}
\end{definition}







\subsection{Weak core vertices}

Definition \ref{def:stark_systems} might seem abstract, but the Stark systems can be controlled by their values at some particular $n\in \NN$.

\begin{definition}
A \emph{weak core vertex} of rank $r$ is a square-free product of ultraprimes $n\in \NN$ such that $\bH^1_{\FF^*_n}(\Q,\bT^*)=0$ and $\bH^1_{\FF}(\Q,\bT)$ is a free $\Lambda$-module of rank $r+\nu(n)$.
\end{definition}

\begin{proposition}
Let $n\in \NN$ be a weak core vertex and let $m\in \NN$ be such that $n\mid m$. Then $m$ is also a weak core vertex.
\label{prop:core_vertex_div}
\end{proposition}

\begin{proof}
Since $n\mid m$, then $\bH^1_{\FF_m}(\Q,T)$ is contained in $\bH^1_{\FF_n}(\Q,T)$, so it also vanishes. The exact sequence
\[\xymatrix{0\ar[r] & \bH^1_{\FF^n}(\Q,\bT) \ar[r] & \bH^1_{\FF^m}(\Q,\bT) \ar[r] & \bigoplus_{\ku\mid \frac{m}{n}}\bH^1_s(\Q_{\ku},\bT)\ar[r] & 0}\]
The first and third terms of this exact sequence are free modules of ranks $r+\nu(n)$ and $\nu(m)-\nu(n)$. Hence $\bH^1_{\FF^m}(\Q,\bT)$ is free of rank $r+\nu(n)$.
\end{proof}

Weak core vertices control the Stark systems.

\begin{theorem}
Let $n\in \NN$ be a core vertex. Then the projection map
\[\bSS(\FF)\to \bigcap^{r+\nu(n)} \bH^1_{\FF^n}(\Q,T)\]
is an isomorphism.
\label{th:stark_core_projection}
\end{theorem}

\begin{proof}
We only need to prove that if $n\in \NN$ is a core vertex and $\ell\in \PP$ does not divide $n$, the map
\[\bigcap^{r+\nu(n\ell)} \bH^1_{\FF^{n\ell}} (\Q,\bT)\to \bigcap^{r+\nu(n)} \bH^1_{\FF^n} (\Q,\bT)\]
is an isomorphism. This map is induced by the exact sequence
\[\xymatrix{0\ar[r] & \bH^1_{\FF^n}(\Q,\bT)\ar[r]&\bH^1_{\FF^{n\ell}}(\Q,\bT)\ar[r] & \bH^1_s(\Q_\ell, \bT) \ar[r] &0}\]
Since $\Ext^1(\Lambda,\Lambda)=0$, the dual map $\bH^1_{\FF^{n\ell}}(\Q,\bT)^+\to \bH^1_{\FF^n}(\Q,\bT)^+$ is surjective. Hence we can construct an injective map 
\[\bigwedge^{r+\nu(n)} \bH^1_{\FF^n}(\Q,\bT)^+ \to \bigwedge^{r+\nu(n\ell)} \bH^1_{\FF^{n\ell}}(\Q,\bT)^+\]
which turns out to be an isomorphism since both are free $\Lambda$-modules of rank $1$. Therefore, its dual map is also an isomorphism.
\end{proof}


If we assume the existence of core vertices, we know that the module of Stark systems is free of rank one.

\begin{assumption}
    There exist an integer $r$ and $n\in \NN$ such that $\bH^1_{\FF^*}(\Q,\bT^*)=0$ and $\bH^1_{\FF^n}(\Q,\bT)$ is a free $\Lambda$-module of rank $r+\nu(n)$.
    \label{ass:core_vertex}
\end{assumption}

\begin{corollary}
Under assumption \ref{ass:core_vertex}, the module of Stark systems $\bSS(\FF)$ is a free $\Lambda$-module of rank one. The generators of $\bSS(\FF)$ are called \emph{primitive} Stark systems.
\end{corollary}

\begin{theorem}
Let $\varepsilon=(\varepsilon)_{n\in \NN}$ be a generator of $\bSS(\FF)$. For every $m\in \NN$, the image of $\varepsilon_m\in \Hom\left(\bigwedge^{r+\nu(m)} \bH^1_{\FF^m}(K,\bT)^+,\Lambda\right)$ contains the $0^{\textrm{th}}$ Fitting ideal of $\bH^1_{\FF_m^*}(\Q,\bT^*)$ with finite index.
\end{theorem}

\begin{proof}
By assumption \ref{ass:core_vertex} and proposition \ref{prop:core_vertex_div}, there exists a core vertex $n$ such that $m\mid n$, which leads to the following exact sequence
\[\xymatrix{0\ar[r] & \bH^1_{\FF^n}(K,\bT) \ar[r] & \bH^1_{\FF^m}(K,\bT)\ar[r] & \prod_{\ku\mid n/m} \bH^1_\s(K_\ku,\bT) \ar[r] & \bH^1_{\FF^*_m}(K,\bT^*)\ar[r] & 0}\]
which induces a map 
\[\phi_{n,m}:\ \bigcap^{r+\nu(n)} \bH^1_{\FF^n}(K,\bT)\to \bigcap^{r+\nu(m)} \bH^1_{\FF^m}(K,\bT)\]

Since $\varepsilon$ generates $\bSS(\FF)$, Theorem \ref{th:stark_core_projection} implies that $\varepsilon_n$ generates $\bigcap^{r+\nu(n)}$. Since $\varepsilon_m=\phi_{n,m}(\varepsilon_n)$, then proposition \ref{prop:bidual_fitting} implies that the image of $\varepsilon_n$ contains $\Fitt^0(\bH^1_{\FF_m}(\Q,\bT^*))$ with finite index. 
\end{proof}

