\chapter{Kolyvagin systems over the Iwasawa algebra}

%The results on the previous chapter were limited in two different directions: the coefficient ring $R$ was required to be principal and the core rank was at most one. These two assumptions were relaxed in the work of D.~Burns, R.~Sakamoto and T.~Sano. In this work, Kolyvagin systems are redefined as collections of classes in the bidual exterior powers of Selmer groups and can be used to bound the Fitting ideals of the Selmer group. However, Kolyvagin systems in this setting do not determine all Fitting ideals of the Selmer group, and one need to introduce the notion of Stark system in order to do that.

The results in previous chapters were limited to principal rings. Here, we generalise previous results to compute the Fitting ideal over the Iwasawa algebra $\Lambda:=\Z_p[[T]]$. For that we need to consider Kolyvagin systems consisting of collections of classes in exterior biduals of Selmer groups, as defined in \cite{BurnsSakamotoSano2}. In this chaper, we combine this setting with the patched Selmer groups defined in Chapter \ref{ch:patched} following \cite{Sweeting}. This new setting presents some technical complications, since the coefficient ring $\Lambda$ is no longer self-injective, a condition required in the construction in \cite{BurnsSakamotoSano2}.

\section{Iwasawa Selmer modules}

In this chapter, we will assume that we are working with a Galois representation $\bT$ over the Iwasawa algebra. 

\begin{assumption}
Let $\Lambda=\Z_p[[X]]$ be the classical Iwasawa algebra and let $\bT$ be a $\Lambda[[G_K]]$-module that is free and finitely generated as a $\Lambda$-module.
\end{assumption}

We will aim to develop a theory of Kolyvagin systems to study Selmer groups defined on $\bT$. We will use the patched cohomology setting, as in Definition \ref{def:patched:selmer}.

\begin{notation}
We denote $\bT_{n,m}=\bT/(p^n,X^m)\bT$.
\end{notation}

The definition of patched Selmer structures for these Galois representations is included in Definition \ref{def:patched:selmer}. But the definition of patched cartesian Selmer structures, established in Definition \ref{def:patched_cartesian}, was exclusive for discrete valuation rings. However, a similar definition applies to Iwasawa Selmer modules.

\begin{definition}(Cartesian local condition)
A local condition $\bH^1_{\FF}(K_\ku,T)\subset \bH^1(K_\ku,T)$ is called \emph{cartesian} if $H^1_{/\FF}(K_\ku,T)$ is a torsion-free $R$-module. A Selmer structure is said to be cartesian is all of its local conditions are cartesian.
\label{def:iwasawa:cartesian}
\end{definition}

\begin{proposition}
Let $\FF$ be a cartesian Selmer structure and let $f\in \Lambda$. Then there is an injection 
\[\bH^1_{\FF}(K,\bT)/f\bH^1_{\FF}(K,\bT)\hookrightarrow \bH^1_{\FF}(K,\bT/f\bT)\]
\label{prop:iwasawa:quotient_injection}
\end{proposition}

\begin{proof}
Let $\kn$ be the product of the ultraprimes in $\Sigma_{\FF}$. Consider the exact sequence 
\[\xymatrix{0\ar[r] & \bT \ar[r] & \bT \ar[r] & \bT/f\bT\ar[r] & 0}\]
\textcolor{red}{long cohomological exact sequence} induces an injection
\[\bH^1(K^\kn/K,\bT)/f \bH^1(K^\kn/K,\bT) \hookrightarrow \bH^1(K^\kn/K,\bT/f\bT)\]

Since $\prod_{\ku\in \Sigma_\FF} \bH^1(K_\ku,\bT)$ contains no $f$-torsion, there is another injection 
\[\bH_{\FF}^1(K^\kn/K,\bT)/f \bH^1_{\FF}(K^\kn/K,\bT)\hookrightarrow \bH^1(K^\kn/K,\bT)/f \bH^1(K^\kn/K,\bT)\]
which completes the proof.
\end{proof}

We want to prove an analogue with quotients by the elements divisible by the maximal ideal $\m$. Since the maximal ideal is not principal, we need to impose an extra assumption on $H^2(K,\bT)$ containing no finite $\Lambda$-submodules.

\begin{assumption}
The cohomology group $\bH^2(K,\bT)$ contains no finite $\Lambda$-submodules.
\label{ass:H2}
\end{assumption}

\begin{proposition}
Let $\FF$ be a cartesian Selmer structure and recall $\m$ is the maximal ideal of $\Lambda$. Then there is an injection 
\[\bH^1_{\FF}(K,\bT)/\m\hookrightarrow \bH^1_{\FF}(K,\bT/\m)\]
\textcolor{red}{comment of unramified patched cohomology}
\label{prop:iwasawa:quotient_maximal}
\end{proposition}

\begin{proof}
Recall that $\kn$ is the product of all the primes in $\Sigma_\FF$. By Proposition \ref{prop:iwasawa:quotient_injection},
\[\bH^1(K^{\kn}/K,\bT)/X\hookrightarrow \bH^1(K^{\kn}/K,\bT/X\bT)\twoheadrightarrow \bH^2(K^{\kn}/K,\bT)[X] \]
Snake's lemma induces another exact sequence
\[\xymatrix{\bH^2(K^{\kn}/K,\bT)[\m]\ar[r]   & \bH^1(K^{\kn}/K,\bT)/\m\ar[r]   & \bH^1(K^{\kn}/K,\bT/X\bT)/p}\]
By Assumption \ref{ass:H2}, the first term vanishes, so the second map is injective. Now consider the exact sequence
\[\xymatrix{0\ar[r] & \bT/X\bT\ar[r]^p & \bT/X\bT\ar[r] & \bT/\m\bT\ar[r] & 0}\]
It induces an injection
\[\bH^1(K^{\kn}/K,\bT/X\bT)/p\hookrightarrow \bH^1(K^{\kn}/K,\bT/\m\bT)\]
Combining it with the injection above, we obtain an injection 
\[\bH^1_{\FF}(K,\bT)/\m\hookrightarrow \bH^1_{\FF}(K,\bT/\m)\]
\end{proof}





\subsection{Suitable ultraprimes}

This section is devoted to proving analogues of Propositions \ref{prop:cheb} and \ref{prop:cheb_nd} in the setting of patched Iwasawa Selmer groups. The method we will use involves a recursive application of Proposition \ref{prop:cheb} to construct certain ultraprimes satisfying certain properties.


\begin{proposition}
Let $\bT$ be a Galois representation over the Iwasawa algebra satisfying Assumptions \ref{ass:patched:basic} and \ref{ass:H2}. Assume that $\bH^1_{\FF}(K,\bT)$ does not vanish. Then there is an ultraprime $\ku$ such that $\loc_\ku\neq 0$, whose construction involves two choices.
\label{prop:iwasawa:cheb:surjective}
\end{proposition}

\begin{proof}
By Nakayama's lemma, there exists $\alpha\in \bH^1(K,\bT)\setminus \m \bH^1(K,\bT)$. Write
\[\alpha=(\alpha_{n,m})\in \varprojlim_{n,m} \bH^1_{\FF}(K,\bT_{n,m})=\bH^1_{\FF}(K,\bT_{n,m})\]
Since $\alpha\notin \m\bH^1(K,\bT)$, Proposition \ref{prop:iwasawa:quotient_maximal} implies that $\alpha_{1,1}\neq 0$. For every $a\in \N$, $\alpha_{a,a}$ projects to $\alpha_{1,1}$ under the map
\[\bH^1_{\FF}(K,\bT_{a,a})\to \bH^1_{\FF}(K,\bT_{1,1})\]
Hence $\alpha_{a,a}$ does not belong to the kernel of this projection, i.e., 
\[p^{a-1}X^{a-1}\alpha_{a,a}\neq 0\]


For every $a\in \N$, choose a sequence $\Bigl(\alpha_{a,a}^{(i)}\Bigr)$ representing $\alpha_{a,a}$. By the definition of ultraproduct and the closedness of ultrafilters under intersections, there is a set $S_a\in \UU$ such that
\[p^{a-1} X^{a-1} \alpha_{a,a}^{(i)}\neq 0\ \forall i\in S_a\]
Denote $\iota_a:\ \N\to S_a$ the unique increasing bijection.

For every $a$, we can choose, by Proposition \ref{prop:cheb}, a Kolyvagin prime $\ell_a\in \PP(T_{a,a})$ such that 
\[\loc_{\ell_a}\Bigl(p^{a-1} X^{a-1} \alpha_{a,a}^{\iota(a)}\Bigr)\neq 0\]
This implies that the map
\[\loc_{\ell_a}:\ \bH^1_{\FF}(K,\bT_{a,a})\to \bH^1_{\f}(K,\bT_{a,a})\]
is surjective.

Construct the ultraprime $\ku$ represented by the sequence $(\ell_a)_{a\in \N}$. Consider the maps
\[\loc_\ku:\ \bH^1_{\FF}(K,\bT_{a,a})\to \bH^1_\f(K_\ku,\bT_{a,a})\]

Since $\ell_a\in \PP(T_{a,a})$ is a Kolyvagin ultraprime, $\bH^1_\f(K_\ku,T_{a,a})$ is a free, cyclic $\Lambda_{a,a}$-module.

For every $b\geq a$, the construction implies that 
\[\loc_{\ell_b}\Bigl(p^{b-1} X^{b-1} \alpha_{b,b}^{\iota(b)}\Bigr)\neq 0\]
It implies that the top horizontal arrow of the following diagram is surjective.
\[\xymatrix{
    \bH^1_{\FF}(K,\bT_{b,b})\ar@{->>}[r]^{\loc_{\ell_b}} \ar[d]  &  \bH^1_\f(K_{\ell_b},\bT_{b,b}) \ar@{->>}[d]\\
    \bH^1_{\FF}(K,\bT_{a,a})\ar@{->>}[r]^{\loc_{\ell_b}}  &  \bH^1_\f(K_{\ell_b},\bT_{a,a})
}\]
The rightmost vertical arrow is surjective because $\ell_b\in \PP(T_{b,b})$. Therefore, the bottom horizontal arrow is also surjective.

By \ref{prop:ultrafilter_exact}, the patched map
\[\loc_\ku:\ \bH^1_{\FF}(K,\bT_{a,a})\to \bH^1(K_\ku,\bT_{a,a})\]
is also surjective for all $a\in \NN$. Since the inverse limit is right exact, the following map is also surjective
\[\loc_\ku:\ \bH^1_{\FF}(K,\bT)\to \bH^1(K_\ku,\bT)\]
\end{proof}

On the dual side, we can construct suitable Kolyvagin ultraprimes controlling the Selmer group.

\begin{proposition}
Let $\bT$ be a Galois representation over the Iwasawa algebra satisfying Assumptions \ref{ass:patched:basic}. Then there is some $\kn\in \NN$ such that
\[\bH^1_{(\FF^*)_\kn}(K,\bT^*)=0\]
\label{prop:iwasawa:cheb_dual_res}
\end{proposition}

\begin{proof}
Let $\{r^1,\ldots,r^s\}$ be a basis of $\bH^1_{\FF^*}(K,\bT^*[\m])$ as a $k$-vector space. By Lemmas \ref{lem:structure_patching} and \ref{lem:patching_structures_dual}, there is a sequence of Selmer structures $\FF_i$ such that 
\[\bH^1_{\FF^*}(K,\bT^*[\m])=\UU_i\Bigl(\bH^1_{(\FF_i)^*}(K,\bT^*[\m])\Bigr)\]
For every $j\in \{1,\ldots, s\}$ choose a sequence $(r_i^j)_{i\in \N}$, where $r_i^j\in H^1_{(\FF_i)^*}(K,\bT^*[\m])$, representing $r^j$.
There is a set of indices $S\in \UU$ such that, for all $i\in S$, the set $\{r_i^1,\ldots, r_i^s\}$ is a basis of $H^1_{(\FF_i)^*}(K,\bT^*[\m])$.

For every $a\in S$, by Proposition \ref{prop:patched:selmer_torsion}, there is an injection
\[\bH^1(K,\bT^*[m])\hookrightarrow \bH^1(K,\bT_{a,a}^*)\]

For every $j\in\{1,\ldots, s\}$ and $a\in \N$, we can choose a Kolyvagin prime in $\ell_a^j\in \PP(T_{a,a})$ satisfying that 
\[\loc_{\ell_a^j}(r_{a^j})\neq 0\]

Choose ultraprimes $\ku^1=(\ku^1_i)_{i\in \N},\ldots, \ku^s=(\ku^s_i)_{i\in \N}$ satisfying that $\ku^j_a=\ell_a^j$ for each $j=1,\ldots, s$ and $a\in S$.

Define $n_{a}:=\ku_a^1\cdots\ku_a^s$ for each $a\in \N$. Then, we have that
\[H^1_{(\FF_{a})^*_{n_a}}(K,\bT^*[\m])=0\ forall a\in S\]
If $\kn=\ku^1\cdots\ku^s$, Lemma \ref{prop:patched:selmer_torsion} implies that 
\[\bH^1_{(\FF^*)_\kn}(K,\bT^*)[\m]=\bH^1_{(\FF^*)_\kn}(K,\bT^*[\m])=\UU_a\Bigl(H^1_{(\FF_{a})^*_{n_a}}(K,\bT^*[\m])=0\Bigr)=0\]
\end{proof}

By combining choices of primes in the proofs of Propositions \ref{prop:iwasawa:cheb:surjective} and \ref{prop:iwasawa:cheb_dual_res}, we can obtain the following improved version of the last proposition.

\begin{corollary}
Let $\bT$ be a Galois representation over the Iwasawa algebra satisfying Assumptions \ref{ass:patched:basic} and let $\FF$ be a cartesian Selmer structure of positive core rank. Then there is some $\kn\in \NN$ such that
\[\bH^1_{(\FF^*)_{(\kn)}}(K,\bT^*)=0\]
\label{prop:iwasawa:cheb_dual_tr}
\end{corollary}

\begin{proof}
By a combination of the proofs of Propositions \ref{prop:iwasawa:cheb:surjective} and \ref{prop:iwasawa:cheb_dual_res}, we can construct ultraprimes $\ku^1,\ldots, \ku^j$ such that 
\[\loc_{\ku^j}:\ \bH^1_{\FF(\ku^1\cdots\ku^{j-1})}(K,\bT)\to \bH^1_\f(K_{\ku^j},\bT)\]
is surjective for each $j=1,\ldots,s$ and 
\[\bH^1_{(\FF^*)_\kn}(K,\bT^*)=0\]
Indeed, we can construct the ultraprimes in Proposition \ref{prop:iwasawa:cheb_dual_res} individually satisfying Proposition \ref{prop:iwasawa:cheb:surjective}, since $\bH^1_{\FF(\ku^1\ldots\ku^{j-1})}(K,\bT)\neq 0$ since the core rank of $\FF(\ku^1\ldots\ku^{j-1})$ is positive. \textcolor{red}{complete}
\end{proof}







\section{Stark Systems}

Let $m,n\in \NN$ be square-free products of Kolyvagin ultraprimes such that $m\mid n$. There is an exact sequence
\[\xymatrix{0\ar[r] & \bH^1_{\FF^m}(K,T)\ar[r] & \bH^1_{\FF^n}(K,T)\ar[r] & \prod_{\ell\mid \frac{n}{m}} \bH^1_{\s}(K_\ell,T)}\]
By proposition \ref{prop:bidual_map}, this exact sequence induces a map
\[\Phi_{n,m}:\ \bigcap^{r+\nu(n)} \bH^1_{\FF^n}(K,T)\to \bigcap^{r+\nu(m)} \bH^1_{\FF^m}(K,T)\]

\begin{remark}
Note that the map $\Phi_{n,m}$ is dependent on a choice of an isomorphism $\bH^1_{\s}(\Q,T)\cong R$, or equivalently, an element in $\bH^1_{\s}(K_\ell,T)^\times$. From now on, we assume we have fixed such isomorphism for every $\ell\in \PP$.
\end{remark}

\begin{lemma}
Let $\m,n,r\in \NN$ be square-free products of Kolyvagin ultraprimes such that $m\mid n\mid r$. Then
\[\phi_{r,m}=\phi_{r,n}\circ \phi_{n,m}\]
\label{lem:com} 
\end{lemma}

\begin{proof}
\textcolor{red}{do}
\end{proof}

Therefore, the set of maps $\phi_{n,m}$ forms an inverse system, so it makes sense to consider the elements in the inverse limit.



\begin{definition}
The set of Stark systems of $\FF$ is defined as the inverse limit
\[\bSS(\FF):=\varprojlim_{n\in \NN} \bigcap^{r+\nu(n)} \bH^1_{\FF^n}(\Q,T)\]
\label{def:stark_systems}
\end{definition}







\subsection{The module of Stark systems}

Definition \ref{def:stark_systems} might seem abstract, but the Stark systems can be controlled by their values at some particular $n\in \NN$.

\begin{definition}
A \emph{weak core vertex} of rank $r$ is a square-free product of ultraprimes $n\in \NN$ such that $\bH^1_{\FF^*_n}(\Q,\bT^*)=0$ and $\bH^1_{\FF^n}(\Q,\bT)$ is a free $\Lambda$-module of rank $r+\nu(n)$.
\end{definition}

\textcolor{red}{prove existence of weak core vertices}

\textcolor{red}{comments on the rank}

The main theorem of this section shows that the Starks systems are controlled by their values at weak core vertices.

\begin{theorem}
Let $n\in \NN$ be a core vertex. Then the projection map
\[\bSS(\FF)\to \bigcap^{r+\nu(n)} \bH^1_{\FF^n}(\Q,\bT)\]
is an isomorphism.
\label{th:stark_core_projection}
\end{theorem}


\begin{lemma}
Let $n\in \NN$ be a weak core vertex and let $m\in \NN$ be such that $n\mid m$. Then $m$ is also a weak core vertex.
\label{lem:core_vertex_div}
\end{lemma}

\begin{proof}
Since $n\mid m$, then $\bH^1_{\FF_m}(\Q,T)$ is contained in $\bH^1_{\FF_n}(\Q,T)$, so it also vanishes. The exact sequence
\[\xymatrix{0\ar[r] & \bH^1_{\FF^n}(\Q,\bT) \ar[r] & \bH^1_{\FF^m}(\Q,\bT) \ar[r] & \bigoplus_{\ku\mid \frac{m}{n}}\bH^1_s(\Q_{\ku},\bT)\ar[r] & 0}\]
The first and third terms of this exact sequence are free modules of ranks $r+\nu(n)$ and $\nu(m)-\nu(n)$. Hence $\bH^1_{\FF^m}(\Q,\bT)$ is free of rank $r+\nu(n)$.
\end{proof}





\begin{proof}[Proof of Theorem \ref{th:stark_core_projection}]
We only need to prove that if $n\in \NN$ is a core vertex and $\ell\in \PP$ does not divide $n$, the map
\[\bigcap^{r+\nu(n\ell)} \bH^1_{\FF^{n\ell}} (\Q,\bT)\to \bigcap^{r+\nu(n)} \bH^1_{\FF^n} (\Q,\bT)\]
is an isomorphism. This map is induced by the exact sequence
\[\xymatrix{0\ar[r] & \bH^1_{\FF^n}(\Q,\bT)\ar[r]&\bH^1_{\FF^{n\ell}}(\Q,\bT)\ar[r] & \bH^1_s(\Q_\ell, \bT) \ar[r] &0}\]
Since $\Ext^1(\Lambda,\Lambda)=0$, the dual map $\bH^1_{\FF^{n\ell}}(\Q,\bT)^+\to \bH^1_{\FF^n}(\Q,\bT)^+$ is surjective. Hence we can construct an injective map 
\[\bigwedge^{r+\nu(n)} \bH^1_{\FF^n}(\Q,\bT)^+ \to \bigwedge^{r+\nu(n\ell)} \bH^1_{\FF^{n\ell}}(\Q,\bT)^+\]
which turns out to be an isomorphism since both are free $\Lambda$-modules of rank $1$. Therefore, its dual map is also an isomorphism.
\end{proof}


If we assume the existence of core vertices, we know that the module of Stark systems is free of rank one.

\begin{assumption}
    There exist an integer $r$ and $n\in \NN$ such that $\bH^1_{\FF^*}(\Q,\bT^*)=0$ and $\bH^1_{\FF^n}(\Q,\bT)$ is a free $\Lambda$-module of rank $r+\nu(n)$.
    \label{ass:core_vertex}
\end{assumption}

\begin{corollary}
Under assumption \ref{ass:core_vertex}, the module of Stark systems $\bSS(\FF)$ is a free $\Lambda$-module of rank one. The generators of $\bSS(\FF)$ are called \emph{primitive} Stark systems.
\end{corollary}

\begin{theorem}
Let $\varepsilon=(\varepsilon)_{n\in \NN}$ be a generator of $\bSS(\FF)$. For every $m\in \NN$, the image of $\varepsilon_m\in \Hom\left(\bigwedge^{r+\nu(m)} \bH^1_{\FF^m}(K,\bT)^+,\Lambda\right)$ contains the $0^{\textrm{th}}$ Fitting ideal of $\bH^1_{\FF_m^*}(\Q,\bT^*)$ with finite index.
\end{theorem}

\begin{proof}
By Assumption \ref{ass:core_vertex} and Lemma \ref{lem:core_vertex_div}, there exists a core vertex $n$ such that $m\mid n$, which leads to the following exact sequence
\[\xymatrix{0\ar[r] & \bH^1_{\FF^n}(K,\bT) \ar[r] & \bH^1_{\FF^m}(K,\bT)\ar[r] & \prod_{\ku\mid n/m} \bH^1_\s(K_\ku,\bT) \ar[r] & \bH^1_{\FF^*_m}(K,\bT^*)\ar[r] & 0}\]
which induces a map 
\[\phi_{n,m}:\ \bigcap^{r+\nu(n)} \bH^1_{\FF^n}(K,\bT)\to \bigcap^{r+\nu(m)} \bH^1_{\FF^m}(K,\bT)\]

Since $\varepsilon$ generates $\bSS(\FF)$, Theorem \ref{th:stark_core_projection} implies that $\varepsilon_n$ generates $\bigcap^{r+\nu(n)}$. Since $\varepsilon_m=\phi_{n,m}(\varepsilon_n)$, then proposition \ref{prop:bidual_fitting} implies that the image of $\varepsilon_n$ contains $\Fitt^0(\bH^1_{\FF_m}(\Q,\bT^*))$ with finite index. 
\end{proof}

\subsection{Higher Fitting ideals of the Selmer group}

Stark systems also determine the higher Fitting ideals of Selmer group. For that, we need to define a sequence of theta ideals associated to a Stark system, similarly to the ones defined previously to Kolyvagin systems.
\begin{definition}
Let $\varepsilon=(\varepsilon_n)_{n\in \NN}\in \SS(\FF)$ be an Stark system. For each non-negative integer $i$, we define the $i^{\textrm{th}}$ theta ideal by
\[\Theta_i(\varepsilon)=\sum_{n\in \NN_i} \Im(\varepsilon_n)\]
where $\varepsilon_n$ is understood as an element of $\Hom\left(\bigwedge^{r+\nu(m)} \bH^1_{\FF^m}(K,\bT)^+,\Lambda\right)$.
\end{definition}

\section{Kolyvagin systems}

\subsection{Definition of Kolyvagin systems}

The exact sequence
\[\xymatrix{0\ar[r] & \bH^1_{{\FF(n)_{\ell}}}(\Q,T)\ar[r] & \bH^1_{\FF(n)}(\Q,T)\ar[r]& \bH^1_{\f}(\Q_\ell, T)\cong \Lambda}\] 
induces a map, using proposition \ref{prop:bidual_map}
\[\nu_\ell:\ \bigcap^r \bH^1_{\FF(n)}(\Q,T) \to \bigcap^{r-1} \bH^1_{\FF(n)_\ell}(\Q,T)\] 

On the other hand, the exact sequence
\[\xymatrix{0\ar[r] & \bH^1_{{\FF(n)_{\ell}}}(\Q,T)\ar[r] & \bH^1_{\FF(n\ell)}(\Q,T)\ar[r]& \bH^1_{\tr}(\Q_\ell, T)\cong \Lambda}\] 
induces a map
\[\varphi_\ell^\fs:\ \bigcap^r \bH^1_{\FF(n\ell)}(\Q,T) \to \bigcap^{r-1} \bH^1_{\FF(n)_\ell}(\Q,T)\]

\begin{definition}
A \emph{Kolyvagin system} of rank $r$ is an element
\[(\kappa_n)_{n\in \NN} \in \prod_{n\in \NN} \bigcap_r \bH^1_{\FF(n)}(\Q,T)\]
satisfying for all $n\in \NN$ and $\ell\in \PP$ not dividing $n$ that 
\[\varphi_\ell^\fs (\kappa_{n\ell})=\nu_\ell(\kappa_n)\]
\end{definition}

\subsection{Regulator map}

The exact sequence 
\[\xymatrix{0\ar[r] & \bH^1_{\FF(n)}(\Q,T)\ar[r] &\bH^1_{\FF^n}(\Q,T) \ar[r] & \prod_{\ell\mid n} \bH^1_{\f}(\Q_\ell, T)}\cong \Lambda^{\nu(n)}\] 
induces, by propositon \ref{prop:bidual_map}, a map
\[\Reg_n:\ \bigcap^{r+\nu(n)} \bH^1_{\FF^n}(\Q,T)\to \bigcap^r \bH^1_{\FF(n)}(\Q,T)\]

\begin{theorem}
Combined for all $n\in \NN$, the maps $\Reg_n$ induce a regulator map
\[\Reg:\ \bSS(\FF)\to \bKS(\FF)\]
\label{th:regulator_bijective}
\end{theorem}

\begin{proof}

\end{proof}

\begin{theorem}
The regulator map $\Reg:\ \bSS(\FF)\to \bKS(\FF)$ is an isomorphism.
\end{theorem}

\begin{theorem}
Let $\kappa$ be a primitive Kolyvagin system. Then $\textrm{Im}(\kappa_n)$ contains the Fitting ideal $\Fitt_0\left(\bH^1_{\FF(n)^*}(\Q,T^*)^\vee\right)$ with finite index.
\label{th:kol_fitting0}
\end{theorem}

\begin{proof}
Since $\kappa$ is a primitive Kolyvagin system, theorem \ref{th:regulator_bijective} implies the existence of a primitive Stark system $\varepsilon=(\varepsilon_n)_{n\in \NN}$ such that $\Reg(\epsilon)=\kappa$.


Let $m$ be a core vertex such that $n\mid m$. Consider the exact sequence
\[\xymatrix{0 \ar[r] & \bH^1_{\FF(n)} (\Q,T)\ar[r] & \bH^1_{\FF^m}(\Q,T) \ar[r] & \bigoplus_{\ell\mid n} \bH^1_\f(\Q_\ell, T)\oplus_{\ell\mid {m/n}}\bH^1_{\tr}(\Q_\ell, T) \\ & \bH^1_{\FF(n)^*}(\Q,T^*)^\vee\ar[r] & 0}\]

The map induced by proposition \ref{prop:bidual_map} is $\Reg_n\circ \phi_{m,n}$ and sends $\varepsilon_m$ to $\kappa_n$. Hence, by proposition \ref{prop:bidual_fitting}, $\textrm{Im}(\kappa_n)$ contains $\Fitt_0\left(\bH^1_{\FF}(\Q,T^*)^\vee\right)$ with finite index.
\end{proof}

\begin{corollary}
For every $\kappa\in \KS(\bT)$ and every height $1$ prime ideal of $\Lambda$, the localization $\Im(\kappa_n)$ is contained in $\Fitt_0\left(\bH^1_{\FF(n)^*}(\Q,T^*)^\vee\right)_\beta$.
\end{corollary}

\begin{proof}
Then $\kappa$ is primitive, the corollary follows from Theorem \ref{th:kol_fitting0} since the localization at $\beta$ of every finite $\Lambda$-module vanishes.

When $\kappa$ is not primitive, then $\kappa=f \widetilde \kappa$ for some $f\in \Lambda$ and some primitive Kolyvagin system $\widetilde \kappa$. Since the result is true for $\widetilde \kappa$ and $\textrm{Im}(\kappa_n)\subset \textrm{Im}(\widetilde \kappa_n)$, the corollary also holds for $\kappa$.
\label{cor:kol_fitting_0}
\end{proof}



\section{Structure of the Selmer group}


\subsection{Fitting ideal and $\Lambda$-modules up to pseudo-isomorphism}

Let $M$ be a $\Lambda$ module which admits a pseudo-isomorphism
\[M\cong \Lambda^r \times \prod_{i=1}^s {\Lambda}^s\]

\subsection{Fitting ideals of the Selmer groups}
\begin{lemma}
Let $n$ be a core vertex. Then
\[\Fitt^i(\bH^1_{\FF^*}(K,\bT))=\sum_{\ku\mid n} \Fitt^{i-1}(\bH^1_{(\FF^*)_\ku}(K,\bT))\]
\end{lemma}

\begin{corollary}
\[\Fitt^i(\bH^1_{\FF^*}(K,\bT)^\vee)=\sum_{\ku\in \PP} \Fitt^{i-1}(\bH^1_{(\FF^*)_\ku}(K,\bT)^\vee)\]
\end{corollary}

\begin{corollary}
\[\Fitt^i(\bH^1_{\FF^*}(K,\bT^*)^\vee)=\sum_{n\in \NN^i(\PP)} \Fitt^{0}(\bH^1_{(\FF^*)_n}(K,\bT^*)^\vee)\]
\label{cor:fitt_str}
\end{corollary}

\begin{corollary}
\[\sum_{n\in \NN^i(\PP)} \Fitt^{0}(\bH^1_{(\FF^*)(n)}(K,\bT^*)^\vee)\subset \Fitt^i(\bH^1_{\FF^*}(K,\bT^*)^\vee)\]
\textcolor{red}{with finite index.}
\label{cor:fitt_tw}
\end{corollary}

\begin{proof}
Since $\Fitt^{0}(\bH^1_{(\FF^*)_n}(K,\bT^*)^\vee)$ is contained in $\bH^1_{(\FF^*)(n)}(K,\bT^*)^\vee$, then 
\[\Fitt^{0}(\bH^1_{(\FF^*)(n)}(K,\bT^*)^\vee)\subset \Fitt^{0}(\bH^1_{(\FF^*)_n}(K,\bT^*)^\vee)\]
Hence the proposition follows from 
\end{proof}


Let $\kappa$ be a primitive Kolyvagin system.

\begin{definition}
Let $\kappa\in \bKS(\bT)$ be a Kolyvagin system. The $i^{\textrm{th}}$ Theta ideal of $\kappa$ is 
\[\Theta_i(\kappa):=\sum_{n\in \NN^i(\PP)}\textrm{Im}(\kappa_n)\]
\end{definition}

\begin{proposition}
Let $\kappa\in \bKS(\bT)$. Then 
\[\Theta_i(\kappa)\subset_\f \Fitt^i(\bH^1_{\FF^*}(K,\bT^*))\] 
\end{proposition}

\begin{proof}

By Theorem \ref{th:kol_fitting0}, 
\[\textrm{Im}(\kappa_n)\subset_\f\Fitt^0(\bH^1_{\FF(n)^*}(\Q,\bT^*)^\vee)\] 
Equivalently, for every height $1$ prime ideal,
\[\textrm{Im}(\kappa_n)_\beta\subset\Fitt^0(\bH^1_{\FF(n)^*}(\Q,\bT^*)^\vee)_\beta\]
Hence, by corollary \ref{cor:fitt_tw},
\[\Theta_i(\kappa)_\beta=\sum_{n\in \NN^i(\PP)}\textrm{Im}(\kappa_n)_\beta \subset \sum_{n\in \NN^i(\PP)}\Fitt^0(\bH^1_{\FF(n)^*}(\Q,\bT^*)^\vee)_\beta \subset\Fitt^i(\bH^1_{\FF^*}(K,\bT^*)^\vee)_\beta\]
Since it holds for all height $1$ prime ideals $\beta$,
\[\Theta_i(\kappa) \subset_\f \Fitt^i(\bH^1_{\FF^*}(K,\bT^*)^\vee)\]
\end{proof}

\begin{theorem}
Let $\kappa\in \bKS(\bT)$ be a primitive Kolyvagin system. Then 
\[ \Fitt^i(\bH^1_{\FF^*}(K,\bT^*)^\vee)\subset_\f\Theta_i(\kappa)\]
\end{theorem}

\begin{proof}
$\bH^1_{\FF^*}(K,\bT^*)^\vee$ is a finitely generated $\Lambda$-module, so it admits a pseudo-isomorphism 
\[\bH^1_{\FF^*}(K,\bT^*)^\vee \approx \Lambda^r\times \prod_{i=1}^s \prod_{j=1}^{k_i} \Lambda/(f_i^{\alpha_{i,j}})\]
where $r$ is a non-negative integer and $f_i$ are either the prime $p$ or irreducible distinguished polynomials.

An inductive application of Corollaries \ref{cor:cheb_sel_rk} and \ref{cor:cheb_sel_tors} proves the existence of ultraprimes $\ku_1,\ldots,\ku_r,\kv_1,\ldots \kv_t$, where $t$ is the maximum of $k_1,\ldots, k_s$ such that 
\[\begin{aligned}
& \bH^1_{(\FF^*)(\ku_1\cdots\ku_\alpha)}(\Q,\bT)\approx \Lambda^{r-i}\times  \prod_{i=1}^s \prod_{j=1}^{k_i} \Lambda/(f_i^{\alpha_{i,j}})\ \ \forall \alpha=0,\ldots,r\\
&\bH^1_{(\FF^*)(\ku_1\cdots\ku_r\kv_1\cdots\kv_j)}(K,\bT^*)\approx \prod_{i=1}^s \prod_{j=\beta+1}^{k_i} \Lambda/(f_i^{\alpha_{i,j}})\ \ \forall \beta=0,\ldots,t
\end{aligned}\]
If we define $kn_i$ as the formal square-free product of the first $i$ primes of $(\ku_1,\ldots,\ku_r,\kv_1,\ldots,kv_s)$, we obtain that
\[\Fitt^i(\bH^1_{\FF^*}(K,\bT^*)^\vee)\subset_\f \Fitt^0(\bH^1_{\FF^*(\kn_i)})\]
By Theorem \ref{th:kol_fitting0}, the latter is contained up to finite index in $\Theta_i$, so this proof is concluded.
\end{proof}


