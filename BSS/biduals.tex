\chapter{Kolyvagin systems over the Iwasawa algebra}

%The results on the previous chapter were limited in two different directions: the coefficient ring $R$ was required to be principal and the core rank was at most one. These two assumptions were relaxed in the work of D.~Burns, R.~Sakamoto and T.~Sano. In this work, Kolyvagin systems are redefined as collections of classes in the bidual exterior powers of Selmer groups and can be used to bound the Fitting ideals of the Selmer group. However, Kolyvagin systems in this setting do not determine all Fitting ideals of the Selmer group, and one need to introduce the notion of Stark system in order to do that.

The results in previous chapters were limited to principal rings. Here, we generalise previous results to compute the Fitting ideal over the Iwasawa algebra $\Lambda:=\Z_p[[T]]$. For that we need to consider Kolyvagin systems consisting of collections of classes in exterior biduals of Selmer groups, as defined in \cite{BurnsSakamotoSano2}. In this chaper, we combine this setting with the patched Selmer groups defined in Chapter \ref{ch:patched} following \cite{Sweeting}. This new setting presents some technical complications, since the coefficient ring $\Lambda$ is no longer self-injective, a condition required in the construction in \cite{BurnsSakamotoSano2}.

\section{Iwasawa Selmer modules}

In this chapter, we will assume that we are working with a Galois representation $\bT$ over the Iwasawa algebra. 

\begin{assumption}
Let $\Lambda=\Z_p[[X]]$ be the classical Iwasawa algebra and let $\bT$ be a $\Lambda[[G_K]]$-module that is free and finitely generated as a $\Lambda$-module.
\end{assumption}

We will aim to develop a theory of Kolyvagin systems to study Selmer groups defined on $\bT$. We will use the patched cohomology setting, as in Definition \ref{def:patched:selmer}.

The definition of patched Selmer structures for these Galois representations is included in Definition \ref{def:patched:selmer}. But the definition of patched cartesian Selmer structures, established in Definition \ref{def:patched_cartesian}, was exclusive for discrete valuation rings. However, a similar definition applies to Iwasawa Selmer modules.

\begin{definition}(Cartesian local condition)
A local condition $\bH^1_{\FF}(K_\ku,T)\subset \bH^1(K_\ku,T)$ is called \emph{cartesian} if $H^1_{/\FF}(K_\ku,T)$ is a torsion-free $R$-module. A Selmer structure is said to be cartesian is all of its local conditions are cartesian.
\label{def:iwasawa:cartesian}
\end{definition}

\subsection{Suitable ultraprimes}

This section is devoted to proving analogues of Propositions \ref{prop:cheb} and \ref{prop:cheb_nd} in the setting of patched Iwasawa Selmer groups.

\begin{proposition}
Assume $R$ is a zero-dimensional Gorenstein ring and $T$ is an $R[[G_K]]$-module. Let $c_1,\ldots,c_s\in H^1(K,T)$ and let $c_1^*,\ldots,c_t^*\in H^1(K,T^*)$. Assume that $s+t\leq p$. Let $x_i\subset\Ann(\Ann(c_i))$ and $x_i^*\subset\Ann(\Ann(c_j^*))$ be non-zero elements.

Then there exists a set $\QQ\subset \PP$ such that for all $q\in \QQ$, the localizations $x_i\in R\loc_q(c_i)$ and $y_i\in R \loc_q(c_j^*)$ for all $i\leq s$ and $j\leq t$.

%$H^1_\f(K_q,T)[\Ann(c_i)]$ and $H^1_\f(K_q,T^*)[\Ann(c_j^*)]$, respectively, for all $i\leq s$ and $j\leq t$.
\end{proposition}

\begin{proof}
By \cite[Lemma 3.9]{BurnsSakamotoSano2}, the restriction induces an injection
\[H^1(K,T)\hookrightarrow \Hom(K(T)_M,T/(\tau-1)T)\]
Let $F_i$ be the field of definition of $c_i$, let $\widetilde c_i:\ G_K\to T$ be a cocycle representing $c_i$, and define $a_i:=-\widetilde c_i(\tau)\in T/(\tau-1)T$. The definition of $a_i$ is independent of the choice of the cocycle, which implies that $a_i\in T/(\tau-1)T[\Ann(c_i)]$.

The image of $f(c_i)$ coincides $\Bigl(T/(\tau-1)T\Bigr)[\Ann(c_i)]$. Indeed, the injectivity of $f$ implies that $\Ann(f(c_i))=\Ann(c_i)$ and, since $R$ is self-injective and $R\cong T/(\tau-1)T$, $\Im(c_i)=\Ann(\Ann(c_i))$.

Define
\[H_i=f(c_i)^{-1} (a_i+\m \Im(c_i))\subset G_{K(T)_M}\]
which is a coset of the subgroup $f(c_i)^{-1}(\m \Im(c_i))$. By Nakayama's lemma and the assumption on the principality of $\Im(c_i)=\Ann(\Ann(c_i))$, the index of this subgroup is $p$.

Similarly, define $F_j^*$, $a_j^*$ and $H_j^*$ similarly for the class $c_j^*$.

By the assumption $s+t<p$, we can find
\[\gamma\in G_{K(T)_M}\setminus (H_1\cup\cdots \cup H_s\cup H_1^*\cup\cdots \cup H_t^*)\]

Set $F=K(T)_MF_1\cdots F_sF_1^*\cdots F_t^*$, and define $\QQ$ be the set of primes $q\notin S$ such that $q$ is unramified in $F/K$ and $\Frob_q$ is conjugate to $\tau\gamma$ in $\Gal(F/K)$. Then Chebotarev density theorem implies that $\QQ$ has positive density, and it is contained in $\PP$ by construction. For every $q\in\QQ$, we have that
\[\loc_q(c_i)=c_i(\tau\gamma)=\tau c_i(\gamma)+c_i(\tau)=f(c_i)(\gamma)-a_i\in \Im(f(c_i))\setminus \m\,\Im(f(c_i))\subset T/(\tau-1)T\]
under the identification $H^1_\f(K_q,T)\cong T/(\Frob_q-1) T=T/(\tau-1)T$. Since $\Im(c_i)$ is principal, Nakayama's lemma implies that it is generated by $\loc_q(c_i)$. An analogous argument works for $\loc_q(c_i^*)$.
\end{proof}

\begin{proposition}
Let $c\in \bH^1_{\FF}(\Q,T)$ be an element such that its projection to $\bH^1_{\FF}(\Q,T/(p^n,X^m)T)$ is nonzero. Then there exists a Kolyvagin ultraprime $\ku$ such that 
\[(p^{n-1}X^{m-1})\bH^1(\Q_\ku,T)\subset \loc_\ku(\Lambda c)\]
\label{prop:cheb_blocks}
\end{proposition}

\begin{proof}
Write $c_{a,b}$ to the projection of $c$ to $\bH^1_{\FF}(\Q,T/(p^a,t^b)T)$. 

Note that, since $c_{n,m}\neq 0$, then $p^rT^s c_{n+r,m+s}\neq 0$. \textcolor{red}{prove this}

For every $(n+r,m+s)\in \N^2$, let $\ell_{n,m}$ be a Kolyagin prime for $T_{n,m}$ such that $\loc_{\ell_{n+r,m+s}}(p^rT^s c_{n+r,m+s})\neq 0$. Use an ordering of $\N^2$ to construct an ultraprime $\ku$ as the sequence of all the ${n,m}$.

It is easily seen that $\loc$
\end{proof}

\begin{assumption} 
Let $I=f\Lambda$ be a principal ideal. The canonical map
\[\bH^1_{\FF}(K,\bT)/I\bH_{\FF}^1(K,\bT)\to \bH^1_{\FF}(K,T/IT)\]
is an injection.
\label{ass:cartesian}
\end{assumption}

\begin{assumption}
If $\bT'$ is a quotient of $\bT$, then $H^0(K,\bT')=0$.
\label{ass:sur}
\end{assumption}



\begin{lemma}

Assume that $\bH^1(\Q,\bT)$ is non-zero and that there is a map
\[\varphi:\ \Lambda \to \bH^1_{\FF^*}(\Q,\bT^*)^\vee\]
with kernel $f\Lambda$., where $f$ is either $0$ or a product of a distinguished polynomial with a power of $p$. Then there exists a Kolyvagin ultraprime $\ku$ such that 
\[\loc_\ku:\ \bH^1_{\FF}(K,\bT)\to \bH^1_{\f}(K_\ku,\bT)\]
is surjective and 
\[\loc_\ku^*:\ \bH^1_\f(K_\ku,\bT^*)^\vee\to \bH^1_{\FF^*}(K,\bT^*)^\vee\]
has kernel contained in $f\bH^1_\f(K_\ku,\bT^*)^\vee$.

\label{lem:chebotarev}
\end{lemma}

\begin{proof}
By Nakayama's lemma, there exists $\alpha\in \bH^1(\Q,\bT)\setminus \m \bH^1(\Q,\bT)$. Write
\[\alpha=(\alpha_{n,m})\in \varprojlim_{n,m} \bH^1_{\FF}(K,\bT_{n,m})=\bH^1_{\FF}(K,\bT_{n,m})\]
Since $\alpha\notin p\bH^1(K,\bT)$, assumption \ref{ass:cartesian} implies that $\alpha$ projects to a non-zero element in $\bH^1(\Q,\bT/p\bT)$. Since $\Lambda/p\Lambda\cong \F_p[[X]]$ is a discrete valuation ring, then 
\[\bH^1(\Q,\bT/p\bT)=\varprojlim_m \bH^1(\Q,\bT_{1,m})\] 
Therefore, there exists some $m_0$ such that $\alpha_{1,m_0}\neq 0$. It implies that $\alpha_{n,m}\neq 0$ for all $m\geq m_0$. Moreover, the composition of maps
\[\xymatrix{\bH_{\FF}^1(K,\bT_{n,m})\ar[r] & \bH^1_{\FF}(K,\bT_{1,m_0})\ar@{>->}[r]^{\cdot p^{n-1} X^{m-m_0}} & \bH^1_{\FF}(K,\bT_{n,m})}\]
is the multiplication by $p^{n-1} X^{m-m_0}$. Since $\alpha_{1,m_0}$ is the image of $\alpha_{n,m}$ under the first map and the second map is injective by assumption \ref{ass:sur}, we get that $p^{n-1} X^{m-m_0} \alpha_{n,m}\neq 0$.


Similarly, since $\alpha\notin X\bH^1(\Q,T)$, an analogous argument swapping the roles of $p$ and $X$ implies that $\alpha_{n_0,1}\neq 0$. From this, it follows that $p^{n-n_0} X^{m-1} \alpha_{n,m}\neq 0$ for all $n\geq n_0$.

On the dual side, the map $\varphi$ induces a surjective map in the dual spaces:
\[\varphi^*:\ \bH^1_{\FF^*}(\Q,\bT^*)\to \Lambda^\vee[f^*]\]
where $f^*$ is the dual polynomial of $f$, i.e., the image of $f$ under the $\Lambda$ automorphism that sends $(1+X)$ to $(1+X)^{-1}$. We need to consider the cases where $f=0$ and $f\neq 0$ separately.

\textcolor{red}{f=0}

When $f\neq 0$, we can express it as $f=p^\alpha g$, for some non-negative integer $\alpha$ and some distinguished polynomial $g$ of degree $d$. Since $\Lambda$ is a unique factorisation domain, there is an injective homomorphism
\[\Lambda/(f)\hookrightarrow \Lambda/(p^\alpha)\times \Lambda/(g)\]
It induces a surjection on the Pontryagin duals
\[\Lambda^\vee[p^\alpha]\times \Lambda^\vee[g^*]\twoheadrightarrow \Lambda^\vee[f^*]\]
Note that this map is induced by the canonical inclusion at every factor.



For every $n\in \N$, the module $\Lambda^\vee[g^*,p^n]$ is the Pontryagin dual of $\Lambda/(g,p^n)$, so it is a free $\Z/p^n$ module of rank $d$. Since
\[\dim_{\Lambda/\m} \Lambda^\vee[g^*,p^n]/\m=\dim_{\Lambda/\m} \Lambda/(g,p^n)[\m]=1,\]
then $\Lambda^\vee[g^*,p^n]$ is a cyclic $\Lambda$-module. Note that, if $x_n$ is a generator of $\Lambda^\vee[g^*,p^n]$, then $p^{n-1} X^{d-1} x_n$ is a non-trivial element in $\Lambda^\vee[\m]$.

Since the map $\varphi^*$ is surjective, there exists some element $\gamma_{n}\in \bH^1_{\FF^*}(\Q,\bT^*)$ such that $\varphi^*(\gamma_n)=x_n$. Since
\[\varphi^*(p^{n-1} X^{d-1} \gamma_n)=p^{n-1} X^{d-1}  x_n\neq 0\]
Therefore, $p^{n-1} X^{d-1}  \gamma_n\neq 0$. 

When $\alpha\geq 1$, for every $n\in \N$, the module $\Lambda^\vee[p^\alpha,X^n]$ is the dual of $\Lambda/(p^\alpha,X_n)$. Since the $\m$-torsion of $\Lambda/(p^\alpha,X_m)$ is a one-dimensional $\Lambda/\m$-vector space, then $\Lambda^\vee[p^\alpha,X^n]$ is a cyclic $\Lambda$-module. Let $y_n$ be a generator of $\Lambda^\vee[p^\alpha,X^n]$. Since $\varphi^*$ is a surjective map and $y_n\in \Lambda[f^*]$, there is some element $\beta_n\in \bH^1_{\FF^*}(\Q,\bT^*)$ such that $\varphi^*(\beta_n)=y_n$. Since $\varphi^*(p^{\alpha-1} X^{n-1}\beta_m)\neq 0$, then $p^{\alpha-1} X^{n-1}\beta_n\neq 0$ as well.


Since
\[\bH^1_{\FF^*}\Bigl(\Q,\bT^*\Bigr)=\varinjlim_{n,m} \bH^1_{\FF^*}\Bigl(\Q,(\bT_{n,m})^*\Bigr)\]
there is a pair $(r(n),s(n))$ for every $n\in \N$ such that both
\[\beta_n,\gamma_n\in \textrm{Im}\biggl( \bH^1_{\FF^*}\Bigl(\Q,(\bT_{r(n),s(n)})^*\Bigr) \hookrightarrow \bH^1_{\FF^*}\Bigl(\Q,\bT^*\Bigr)\biggr)\]
When there is no risk of confussion, we will denote the preimages of $\beta_n$ and $\gamma_n$ indistinctly. We can assume without loss of generality that $r(n),s(n)\geq n,\alpha,n_0,m_0$.

 
 
For any $n\in \N$, fix a representative $\Bigl(c_{r(n),s(n)}^{(i)}\Bigr)_{i\in \N}$ of $c_{r(n),s(n)}$ in the patched cohomology group $\bH^1_{\FF}\Bigl(\Q,\bT_{r(n),s(n)}\Bigr)$. Since $p^{r(n)-1} X^{s(n)-m_0} c_{r(n),s(n)}\neq 0$ and $p^{r(n)-n_0} X^{s(n)-1} c_{r(n),s(n)}\neq 0$, we can guarantee that $p^{r(n)-1} X^{s(n)-m_0} c_{r(n),s(n)}^{(i)}\neq 0$ and $p^{r(n)-n_0} X^{s(n)-1} c_{r(n),s(n)}^{(i)}\neq 0$ for $\UU$-many $i$.

Similarly, we can find representatives $\beta_n^{(i)}$ and $\gamma_n^{(i)}$ of $\beta_n$ and $\gamma_n$, respectively, in $\bH^1_{\FF}\Bigl(\Q,(\bT_{r(n),s(n)})^*\Bigr)$. Similarly, $p^{\alpha-1} X^{n-1} \beta_n^{(i)}\neq 0$ and $\lambda_n\gamma_n^{(i)}\neq 0$ for $\UU$-many $i$.

Since the ultrafilter is closed by finite intersections, the set $S$ of indices $i$ such that all of the conditions 
\[\begin{array}{cc}
p^{r(n)-1} X^{s(n)-m_0} c_{r(n),s(n)}^{(i)}\neq 0,& p^{r(n)-n_0} X^{s(n)-1} c_{r(n),s(n)}^{(i)}\neq 0, \\
p^{\alpha-1} X^{n-1} \beta_n^{(i)}\neq 0, & p^{n-1} X^{d-1}\gamma_n^{(i)}\neq 0
\end{array}\]
hold is in the ultrafilter. In particular, $S$ is infinite, so there is a unique increasing bijection $\iota:\ \N\to S$.

Note that $c_{r(n),s(n)}^{(i)}$ and $c_{r(n),s(n)}^{(i)}$ belong to $H^1\Bigl(K,\bT_{r(n),s(n)}\Bigr)$, and $\beta_n^{(i)}$ and $\gamma_n^{(i)}$ belong to $H^1\Bigl(K,(\bT_{r(n),s(n)})^*\Bigr)$. By \cite[proposition 3.6.1]{MazurRubin} (see also \cite{BurnsSakamotoSano2}), for all $n$ there exists a Kolyvagin prime $\ell_n$ for the representation $\bT_{r(n),s(n)}$ such that the localization maps
\[\begin{array}{cc}
\loc_{\ell_n}:\ H^1\Bigl(K,\bT_{r(n),s(n)}\Bigr)\to H^1(K_{\ell_n},\bT_{r(n),s(n)})\\ \loc_{\ell_n}^*:\ H^1\Bigl(K,(\bT_{r(n),s(n)})^*\Bigr)\to H^1(K_{\ell_n},(\bT_{r(n),s(n)})^*)
\end{array}\]
satisfy that

\[\begin{array}{ccc}
    \loc_{\ell_{n}}\Bigl(p^{r(n)-1} X^{s(n)-m_0} c_{r(n),s(n)}^{\iota(n)}\Bigr)\neq 0,\ &\loc_{\ell_{n}}\Bigl(p^{r(n)-n_0} X^{s(n)-1} c_{r(n),s(n)}^{(i)}\Bigr)\neq 0,\\
    \loc_{{\ell}_{n}} \Bigl(p^{\alpha-1} X^{n-1} \beta_n^{\iota(n)}\Bigr)\neq 0, &\loc_{{\ell}_{n}}\Bigl(p^{n-1} X^{d-1}\gamma_n^{\iota(n)}\Bigr)\neq 0
\end{array}\]



The image $\loc_{\ell_n}(c_{r(n),s(n)}^{\iota(n)})$ belongs to $H^1_\f(K_{\ell_n},\bT_{r(n),s(n)})$, which is isomorphic to the quotient $\Lambda/(p^{r(n)},X^{s(n)})$ since $\ell_n$ is a Kolyvagin prime. By the assumptions on the prime $\ell_n$, $\loc_{\ell_n}(c_{r(n),s(n)})^{\iota(n)}$ is not $p^{r(n)-1}$ torsion nor $X^{s(n)-1}$torsion. Therefore, $\loc_{\ell_n}(c_{r(n),s(n)})^{\iota(n)}$ generates the whole $H^1(K_{\ell_n},\bT_{r(n),s(n)})$.

On the dual side, $\loc_{\ell_n}(\beta_n)$ and $\loc_{\ell_n}(\gamma_n)$ belong to $H^1_{\f}(K_{\ell_n},(\bT_{r(n),s(n)})^*)$, which is isomorphic to $\Lambda^\vee[p^\alpha,X^n]$. Since $p^{\alpha-1} X^{n-1}\loc_{\ell_n}(\beta_n)\neq 0$, then 
\[\Lambda\loc_{\ell_n}(\beta_n)\supset H^1(K_{\ell_n},(\bT_{r(n),s(n)})^*)[p^\alpha, X^{n-1}]\]

Similarly, $\lambda_n\loc_{\ell_n}(\gamma_n)\neq 0$. Since $\Lambda/(p^{r(n)},X^{s(n)})$ is a Gorenstein ring, we can find some $z_n\in \Lambda$ such that $z_n\lambda_n\loc_{\ell_n}(\gamma_n)\neq 0$ is a non-trivial $\m$-torsion element.
Consider the element $z_n \loc_{\ell_n}(\gamma_n)$. Its annihilator contains $\lambda_n\cdot \m$ but does not contain $\lambda_n$.

Construct the ultraprime $\ku$ as the equivalence class of the sequence $(\ell_{n})$. Then, the image of $\loc_{\ku}$ contains $\bH^1_f(\Q_\ku,\bT^*)[f^*]$. This implies that there is a surjection
 \[\loc_\ku(\bH^1_{\FF}(\Q,\bT^*)^\vee) \twoheadrightarrow \bH^1_f(\Q_\ku,\bT^*)^\vee/(f)\]
Since $\loc_{\ku}^\vee$ factors through
\[\bH^1_s(\Q_\ku,\bT^*)^\vee \twoheadrightarrow \loc_\ku(\bH^1_{\FF}(\Q,\bT^*))^\vee \hookrightarrow \bH^1_{\FF}(\Q,\bT^*)^\vee\]
then $\ker(\loc_\ku)$ is contained in $f\bH^1_s(\Q_\ku,\bT^*)^\vee$.
\end{proof}

\begin{corollary}
Assume that $\bH^1_{\FF}(K,\bT)\neq 0$ and that the dual Selmer group admits a pseudo-isomorphims
\[\bH^1_{\FF^*}(K,\bT^*)^\vee\approx \Lambda^r\times \prod_{i=1}^s \prod_{j=1}^{k_i} \Lambda/(f_i^{\alpha_{i,j}})\]
where $r$ is a positive integer and the $f_i$ are either distinguised irreducible polynomials or powers of $p$, pairwise different and $\alpha_{i,j}$ are integers satisfying that $\alpha_{i,1}\geq \cdots\geq \alpha_{i,k_i}$. For every $j$, there is an ultraprime $\ku$ such that 
\[\bH^1_{\FF^*(\ku)}(K_\ku,\bT^*)\approx \Lambda^{r-1}\times \prod_{i=1}^s \prod_{j=1}^{k_i} \Lambda/(f_i^{\alpha_{i,j}})\]
\label{cor:cheb_sel_rk}
\end{corollary}

\begin{proof}
The composite map
\[\xymatrix{\Lambda\ar[r] & \Lambda^r\times \prod_{i=1}^s \prod_{j=1}^{k_i} \Lambda/(f_i^{\alpha_{i,j}}) \ar[r] & \bH^1_{\FF^*(\ku)}(K_\ku,\bT^*)}\]
has pseudo-null cokernel. However, the lack of finite submodules in $\Lambda$ implies that the above map is injective. By Theorem \ref{th:chebotarev}, there exists an ultraprime $\ku$ such that
\[\loc_\ku:\ \bH^1_{\FF}(K,\bT)\to \bH^1_\f(K_\ku,\bT)\]
has finite cokernel and the dual localization
\[\loc_\ku^*:\ \bH^1_\f(K_\ku,\bT^*)^\vee\to \bH^1_\FF(K,\bT^*)^\vee\]
is injective. The cokernel of this map is the strict Selmer group, which is pseudo-isomorphic to 
\[\bH^1_{(\FF^*)(\ku)}(K,\bT^*)^\vee\cong \Lambda^{r-1}\times \prod_{i=1}^s \prod_{j=1}^{k_i} \Lambda/(f_i^{\alpha_{i,j}})\]

\end{proof}

\begin{corollary}
Assume that $\bH^1_{\FF}(K,\bT)\neq 0$ and that the dual Selmer group admits a pseudo-isomorphims
\[\bH^1_{\FF^*}(K,\bT^*)^\vee\approx \prod_{i=1}^s \prod_{j=1}^{k_i} \Lambda/(f_i^{\alpha_{i,j}})\]
where the $f_i$ are either distinguised irreducible polynomials or powers of $p$o, pairwise different and $\alpha_{i,j}$ are integers satisfying that $\alpha_{i,1}\geq \cdots\geq \alpha_{i,k_i}$. For every $j$, there is an ultraprime $\ku$ such that 
\[\bH^1_{\FF^*(\ku)}(K_\ku,\bT^*)\approx \prod_{i=1}^s \prod_{j=2}^{k_i} \Lambda/(f_i^{\alpha_{i,j}})\]
\label{cor:cheb_sel_tors}
\end{corollary}

\begin{proof}
Let $f$ be the product of all $f_i^{\alpha_{i,1}}$. The pseudo isomorphism induces a map
\[\xymatrix{\Lambda/(f)\ar[r] & \prod_{i=1}^s \Lambda/f_i^{\alpha_{i,1}} \ar[r] & \prod_{i=1}^s \prod_{j=1}^{k_i} \Lambda/(f_i^{\alpha_{i,j}}) \ar[r] & \bH^1_{\FF^*}(K,\bT^*)^\vee}\]
By construction, the kernel of this map is pseudo-null. However, $\Lambda/(f)$ contains no finite submodules, so the map is injective.

By Theorem \ref{th:chebotarev}, there exists an ultraprime $\ku$ such that
\[\loc_\ku:\ \bH^1_{\FF}(K,\bT)\to \bH^1_\f(K_\ku,\bT)\]
has finite cokernel and the dual localiztion
\[\loc_\ku^*:\ \bH^1_\f(K_\ku,\bT^*)^\vee\to \bH^1_\FF(K,\bT^*)^\vee\]
is contained in $f\Lambda$. The structure of $\bH^1_\FF(K,\bT^*)^\vee$ implies that $f\Lambda/\ker\loc_\ku^*$ is pseudo-null, so 
\begin{equation} 
    \prod_{i=1}^s \prod_{j=2}^{k_i} \Lambda/(f_i^{\alpha_{i,j}})\approx\bH^1_{\FF^*_\ku}(K_\ku,\bT^*)^\vee 
    \label{eq:pseudo}
\end{equation}

The finiteness of the cokernel of $\loc_\ku$ implies that $\bH^1_{\FF^*}(K,\bT^*)$ has finite index $\bH^1_{\FF^\ku}(K,\bT^*)$. Since
\[\bH^1_{\FF^*_\ku}(K,\bT^*)= \bH^1_{\FF^*}(K,\bT^*)\cap \bH^1_{\FF^*(\ku)}(K,\bT^*)\]
then $\bH^1_{\FF^*_\ku}(K,\bT^*)$ has finite index in $\bH^1_{\FF^*(\ku)}(K,\bT^*)$, so their Pontryagin duals are pseudo-isomorphic. %Hence the map in \ref{eq:pseudo} is also a pseudo-isomorphims when consider $\bH^1_{\FF^*(\ku)}(K,\bT^*)$ as the codomain.
\end{proof}



\section{Stark Systems}

Let $m,n\in \NN$ be square-free products of Kolyvagin ultraprimes such that $m\mid n$. There is an exact sequence
\[\xymatrix{0\ar[r] & \bH^1_{\FF^m}(K,T)\ar[r] & \bH^1_{\FF^n}(K,T)\ar[r] & \prod_{\ell\mid \frac{n}{m}} \bH^1_{\s}(K_\ell,T)}\]
By proposition \ref{prop:bidual_map}, this exact sequence induces a map
\[\Phi_{n,m}:\ \bigcap^{r+\nu(n)} \bH^1_{\FF^n}(K,T)\to \bigcap^{r+\nu(m)} \bH^1_{\FF^m}(K,T)\]

\begin{remark}
Note that the map $\Phi_{n,m}$ is dependent on a choice of an isomorphism $\bH^1_{\s}(\Q,T)\cong R$, or equivalently, an element in $\bH^1_{\s}(K_\ell,T)^\times$. From now on, we assume we have fixed such isomorphism for every $\ell\in \PP$.
\end{remark}

\begin{lemma}
Let $\m,n,r\in \NN$ be square-free products of Kolyvagin ultraprimes such that $m\mid n\mid r$. Then
\[\phi_{r,m}=\phi_{r,n}\circ \phi_{n,m}\]
\label{lem:com} 
\end{lemma}

\begin{proof}
\textcolor{red}{do}
\end{proof}

Therefore, the set of maps $\phi_{n,m}$ forms an inverse system, so it makes sense to consider the elements in the inverse limit.



\begin{definition}
The set of Stark systems of $\FF$ is defined as the inverse limit
\[\bSS(\FF):=\varprojlim_{n\in \NN} \bigcap^{r+\nu(n)} \bH^1_{\FF^n}(\Q,T)\]
\label{def:stark_systems}
\end{definition}







\subsection{The module of Stark systems}

Definition \ref{def:stark_systems} might seem abstract, but the Stark systems can be controlled by their values at some particular $n\in \NN$.

\begin{definition}
A \emph{weak core vertex} of rank $r$ is a square-free product of ultraprimes $n\in \NN$ such that $\bH^1_{\FF^*_n}(\Q,\bT^*)=0$ and $\bH^1_{\FF^n}(\Q,\bT)$ is a free $\Lambda$-module of rank $r+\nu(n)$.
\end{definition}

\textcolor{red}{prove existence of weak core vertices}

\textcolor{red}{comments on the rank}

The main theorem of this section shows that the Starks systems are controlled by their values at weak core vertices.

\begin{theorem}
Let $n\in \NN$ be a core vertex. Then the projection map
\[\bSS(\FF)\to \bigcap^{r+\nu(n)} \bH^1_{\FF^n}(\Q,\bT)\]
is an isomorphism.
\label{th:stark_core_projection}
\end{theorem}


\begin{lemma}
Let $n\in \NN$ be a weak core vertex and let $m\in \NN$ be such that $n\mid m$. Then $m$ is also a weak core vertex.
\label{lem:core_vertex_div}
\end{lemma}

\begin{proof}
Since $n\mid m$, then $\bH^1_{\FF_m}(\Q,T)$ is contained in $\bH^1_{\FF_n}(\Q,T)$, so it also vanishes. The exact sequence
\[\xymatrix{0\ar[r] & \bH^1_{\FF^n}(\Q,\bT) \ar[r] & \bH^1_{\FF^m}(\Q,\bT) \ar[r] & \bigoplus_{\ku\mid \frac{m}{n}}\bH^1_s(\Q_{\ku},\bT)\ar[r] & 0}\]
The first and third terms of this exact sequence are free modules of ranks $r+\nu(n)$ and $\nu(m)-\nu(n)$. Hence $\bH^1_{\FF^m}(\Q,\bT)$ is free of rank $r+\nu(n)$.
\end{proof}





\begin{proof}[Proof of Theorem \ref{th:stark_core_projection}]
We only need to prove that if $n\in \NN$ is a core vertex and $\ell\in \PP$ does not divide $n$, the map
\[\bigcap^{r+\nu(n\ell)} \bH^1_{\FF^{n\ell}} (\Q,\bT)\to \bigcap^{r+\nu(n)} \bH^1_{\FF^n} (\Q,\bT)\]
is an isomorphism. This map is induced by the exact sequence
\[\xymatrix{0\ar[r] & \bH^1_{\FF^n}(\Q,\bT)\ar[r]&\bH^1_{\FF^{n\ell}}(\Q,\bT)\ar[r] & \bH^1_s(\Q_\ell, \bT) \ar[r] &0}\]
Since $\Ext^1(\Lambda,\Lambda)=0$, the dual map $\bH^1_{\FF^{n\ell}}(\Q,\bT)^+\to \bH^1_{\FF^n}(\Q,\bT)^+$ is surjective. Hence we can construct an injective map 
\[\bigwedge^{r+\nu(n)} \bH^1_{\FF^n}(\Q,\bT)^+ \to \bigwedge^{r+\nu(n\ell)} \bH^1_{\FF^{n\ell}}(\Q,\bT)^+\]
which turns out to be an isomorphism since both are free $\Lambda$-modules of rank $1$. Therefore, its dual map is also an isomorphism.
\end{proof}


If we assume the existence of core vertices, we know that the module of Stark systems is free of rank one.

\begin{assumption}
    There exist an integer $r$ and $n\in \NN$ such that $\bH^1_{\FF^*}(\Q,\bT^*)=0$ and $\bH^1_{\FF^n}(\Q,\bT)$ is a free $\Lambda$-module of rank $r+\nu(n)$.
    \label{ass:core_vertex}
\end{assumption}

\begin{corollary}
Under assumption \ref{ass:core_vertex}, the module of Stark systems $\bSS(\FF)$ is a free $\Lambda$-module of rank one. The generators of $\bSS(\FF)$ are called \emph{primitive} Stark systems.
\end{corollary}

\begin{theorem}
Let $\varepsilon=(\varepsilon)_{n\in \NN}$ be a generator of $\bSS(\FF)$. For every $m\in \NN$, the image of $\varepsilon_m\in \Hom\left(\bigwedge^{r+\nu(m)} \bH^1_{\FF^m}(K,\bT)^+,\Lambda\right)$ contains the $0^{\textrm{th}}$ Fitting ideal of $\bH^1_{\FF_m^*}(\Q,\bT^*)$ with finite index.
\end{theorem}

\begin{proof}
By Assumption \ref{ass:core_vertex} and Lemma \ref{lem:core_vertex_div}, there exists a core vertex $n$ such that $m\mid n$, which leads to the following exact sequence
\[\xymatrix{0\ar[r] & \bH^1_{\FF^n}(K,\bT) \ar[r] & \bH^1_{\FF^m}(K,\bT)\ar[r] & \prod_{\ku\mid n/m} \bH^1_\s(K_\ku,\bT) \ar[r] & \bH^1_{\FF^*_m}(K,\bT^*)\ar[r] & 0}\]
which induces a map 
\[\phi_{n,m}:\ \bigcap^{r+\nu(n)} \bH^1_{\FF^n}(K,\bT)\to \bigcap^{r+\nu(m)} \bH^1_{\FF^m}(K,\bT)\]

Since $\varepsilon$ generates $\bSS(\FF)$, Theorem \ref{th:stark_core_projection} implies that $\varepsilon_n$ generates $\bigcap^{r+\nu(n)}$. Since $\varepsilon_m=\phi_{n,m}(\varepsilon_n)$, then proposition \ref{prop:bidual_fitting} implies that the image of $\varepsilon_n$ contains $\Fitt^0(\bH^1_{\FF_m}(\Q,\bT^*))$ with finite index. 
\end{proof}

\subsection{Higher Fitting ideals of the Selmer group}

Stark systems also determine the higher Fitting ideals of Selmer group. For that, we need to define a sequence of theta ideals associated to a Stark system, similarly to the ones defined previously to Kolyvagin systems.
\begin{definition}
Let $\varepsilon=(\varepsilon_n)_{n\in \NN}\in \SS(\FF)$ be an Stark system. For each non-negative integer $i$, we define the $i^{\textrm{th}}$ theta ideal by
\[\Theta_i(\varepsilon)=\sum_{n\in \NN_i} \Im(\varepsilon_n)\]
where $\varepsilon_n$ is understood as an element of $\Hom\left(\bigwedge^{r+\nu(m)} \bH^1_{\FF^m}(K,\bT)^+,\Lambda\right)$.
\end{definition}

\section{Kolyvagin systems}

\subsection{Definition of Kolyvagin systems}

The exact sequence
\[\xymatrix{0\ar[r] & \bH^1_{{\FF(n)_{\ell}}}(\Q,T)\ar[r] & \bH^1_{\FF(n)}(\Q,T)\ar[r]& \bH^1_{\f}(\Q_\ell, T)\cong \Lambda}\] 
induces a map, using proposition \ref{prop:bidual_map}
\[\nu_\ell:\ \bigcap^r \bH^1_{\FF(n)}(\Q,T) \to \bigcap^{r-1} \bH^1_{\FF(n)_\ell}(\Q,T)\] 

On the other hand, the exact sequence
\[\xymatrix{0\ar[r] & \bH^1_{{\FF(n)_{\ell}}}(\Q,T)\ar[r] & \bH^1_{\FF(n\ell)}(\Q,T)\ar[r]& \bH^1_{\tr}(\Q_\ell, T)\cong \Lambda}\] 
induces a map
\[\varphi_\ell^\fs:\ \bigcap^r \bH^1_{\FF(n\ell)}(\Q,T) \to \bigcap^{r-1} \bH^1_{\FF(n)_\ell}(\Q,T)\]

\begin{definition}
A \emph{Kolyvagin system} of rank $r$ is an element
\[(\kappa_n)_{n\in \NN} \in \prod_{n\in \NN} \bigcap_r \bH^1_{\FF(n)}(\Q,T)\]
satisfying for all $n\in \NN$ and $\ell\in \PP$ not dividing $n$ that 
\[\varphi_\ell^\fs (\kappa_{n\ell})=\nu_\ell(\kappa_n)\]
\end{definition}

\subsection{Regulator map}

The exact sequence 
\[\xymatrix{0\ar[r] & \bH^1_{\FF(n)}(\Q,T)\ar[r] &\bH^1_{\FF^n}(\Q,T) \ar[r] & \prod_{\ell\mid n} \bH^1_{\f}(\Q_\ell, T)}\cong \Lambda^{\nu(n)}\] 
induces, by propositon \ref{prop:bidual_map}, a map
\[\Reg_n:\ \bigcap^{r+\nu(n)} \bH^1_{\FF^n}(\Q,T)\to \bigcap^r \bH^1_{\FF(n)}(\Q,T)\]

\begin{theorem}
Combined for all $n\in \NN$, the maps $\Reg_n$ induce a regulator map
\[\Reg:\ \bSS(\FF)\to \bKS(\FF)\]
\label{th:regulator_bijective}
\end{theorem}

\begin{proof}

\end{proof}

\begin{theorem}
The regulator map $\Reg:\ \bSS(\FF)\to \bKS(\FF)$ is an isomorphism.
\end{theorem}

\begin{theorem}
Let $\kappa$ be a primitive Kolyvagin system. Then $\textrm{Im}(\kappa_n)$ contains the Fitting ideal $\Fitt_0\left(\bH^1_{\FF(n)^*}(\Q,T^*)^\vee\right)$ with finite index.
\label{th:kol_fitting0}
\end{theorem}

\begin{proof}
Since $\kappa$ is a primitive Kolyvagin system, theorem \ref{th:regulator_bijective} implies the existence of a primitive Stark system $\varepsilon=(\varepsilon_n)_{n\in \NN}$ such that $\Reg(\epsilon)=\kappa$.


Let $m$ be a core vertex such that $n\mid m$. Consider the exact sequence
\[\xymatrix{0 \ar[r] & \bH^1_{\FF(n)} (\Q,T)\ar[r] & \bH^1_{\FF^m}(\Q,T) \ar[r] & \bigoplus_{\ell\mid n} \bH^1_\f(\Q_\ell, T)\oplus_{\ell\mid {m/n}}\bH^1_{\tr}(\Q_\ell, T) \\ & \bH^1_{\FF(n)^*}(\Q,T^*)^\vee\ar[r] & 0}\]

The map induced by proposition \ref{prop:bidual_map} is $\Reg_n\circ \phi_{m,n}$ and sends $\varepsilon_m$ to $\kappa_n$. Hence, by proposition \ref{prop:bidual_fitting}, $\textrm{Im}(\kappa_n)$ contains $\Fitt_0\left(\bH^1_{\FF}(\Q,T^*)^\vee\right)$ with finite index.
\end{proof}

\begin{corollary}
For every $\kappa\in \KS(\bT)$ and every height $1$ prime ideal of $\Lambda$, the localization $\Im(\kappa_n)$ is contained in $\Fitt_0\left(\bH^1_{\FF(n)^*}(\Q,T^*)^\vee\right)_\beta$.
\end{corollary}

\begin{proof}
Then $\kappa$ is primitive, the corollary follows from Theorem \ref{th:kol_fitting0} since the localization at $\beta$ of every finite $\Lambda$-module vanishes.

When $\kappa$ is not primitive, then $\kappa=f \widetilde \kappa$ for some $f\in \Lambda$ and some primitive Kolyvagin system $\widetilde \kappa$. Since the result is true for $\widetilde \kappa$ and $\textrm{Im}(\kappa_n)\subset \textrm{Im}(\widetilde \kappa_n)$, the corollary also holds for $\kappa$.
\label{cor:kol_fitting_0}
\end{proof}



\section{Structure of the Selmer group}


\subsection{Fitting ideal and $\Lambda$-modules up to pseudo-isomorphism}

Let $M$ be a $\Lambda$ module which admits a pseudo-isomorphism
\[M\cong \Lambda^r \times \prod_{i=1}^s {\Lambda}^s\]

\subsection{Fitting ideals of the Selmer groups}
\begin{lemma}
Let $n$ be a core vertex. Then
\[\Fitt^i(\bH^1_{\FF^*}(K,\bT))=\sum_{\ku\mid n} \Fitt^{i-1}(\bH^1_{(\FF^*)_\ku}(K,\bT))\]
\end{lemma}

\begin{corollary}
\[\Fitt^i(\bH^1_{\FF^*}(K,\bT)^\vee)=\sum_{\ku\in \PP} \Fitt^{i-1}(\bH^1_{(\FF^*)_\ku}(K,\bT)^\vee)\]
\end{corollary}

\begin{corollary}
\[\Fitt^i(\bH^1_{\FF^*}(K,\bT^*)^\vee)=\sum_{n\in \NN^i(\PP)} \Fitt^{0}(\bH^1_{(\FF^*)_n}(K,\bT^*)^\vee)\]
\label{cor:fitt_str}
\end{corollary}

\begin{corollary}
\[\sum_{n\in \NN^i(\PP)} \Fitt^{0}(\bH^1_{(\FF^*)(n)}(K,\bT^*)^\vee)\subset \Fitt^i(\bH^1_{\FF^*}(K,\bT^*)^\vee)\]
\textcolor{red}{with finite index.}
\label{cor:fitt_tw}
\end{corollary}

\begin{proof}
Since $\Fitt^{0}(\bH^1_{(\FF^*)_n}(K,\bT^*)^\vee)$ is contained in $\bH^1_{(\FF^*)(n)}(K,\bT^*)^\vee$, then 
\[\Fitt^{0}(\bH^1_{(\FF^*)(n)}(K,\bT^*)^\vee)\subset \Fitt^{0}(\bH^1_{(\FF^*)_n}(K,\bT^*)^\vee)\]
Hence the proposition follows from 
\end{proof}


Let $\kappa$ be a primitive Kolyvagin system.

\begin{definition}
Let $\kappa\in \bKS(\bT)$ be a Kolyvagin system. The $i^{\textrm{th}}$ Theta ideal of $\kappa$ is 
\[\Theta_i(\kappa):=\sum_{n\in \NN^i(\PP)}\textrm{Im}(\kappa_n)\]
\end{definition}

\begin{proposition}
Let $\kappa\in \bKS(\bT)$. Then 
\[\Theta_i(\kappa)\subset_\f \Fitt^i(\bH^1_{\FF^*}(K,\bT^*))\] 
\end{proposition}

\begin{proof}

By Theorem \ref{th:kol_fitting0}, 
\[\textrm{Im}(\kappa_n)\subset_\f\Fitt^0(\bH^1_{\FF(n)^*}(\Q,\bT^*)^\vee)\] 
Equivalently, for every height $1$ prime ideal,
\[\textrm{Im}(\kappa_n)_\beta\subset\Fitt^0(\bH^1_{\FF(n)^*}(\Q,\bT^*)^\vee)_\beta\]
Hence, by corollary \ref{cor:fitt_tw},
\[\Theta_i(\kappa)_\beta=\sum_{n\in \NN^i(\PP)}\textrm{Im}(\kappa_n)_\beta \subset \sum_{n\in \NN^i(\PP)}\Fitt^0(\bH^1_{\FF(n)^*}(\Q,\bT^*)^\vee)_\beta \subset\Fitt^i(\bH^1_{\FF^*}(K,\bT^*)^\vee)_\beta\]
Since it holds for all height $1$ prime ideals $\beta$,
\[\Theta_i(\kappa) \subset_\f \Fitt^i(\bH^1_{\FF^*}(K,\bT^*)^\vee)\]
\end{proof}

\begin{theorem}
Let $\kappa\in \bKS(\bT)$ be a primitive Kolyvagin system. Then 
\[ \Fitt^i(\bH^1_{\FF^*}(K,\bT^*)^\vee)\subset_\f\Theta_i(\kappa)\]
\end{theorem}

\begin{proof}
$\bH^1_{\FF^*}(K,\bT^*)^\vee$ is a finitely generated $\Lambda$-module, so it admits a pseudo-isomorphism 
\[\bH^1_{\FF^*}(K,\bT^*)^\vee \approx \Lambda^r\times \prod_{i=1}^s \prod_{j=1}^{k_i} \Lambda/(f_i^{\alpha_{i,j}})\]
where $r$ is a non-negative integer and $f_i$ are either the prime $p$ or irreducible distinguished polynomials.

An inductive application of Corollaries \ref{cor:cheb_sel_rk} and \ref{cor:cheb_sel_tors} proves the existence of ultraprimes $\ku_1,\ldots,\ku_r,\kv_1,\ldots \kv_t$, where $t$ is the maximum of $k_1,\ldots, k_s$ such that 
\[\begin{aligned}
& \bH^1_{(\FF^*)(\ku_1\cdots\ku_\alpha)}(\Q,\bT)\approx \Lambda^{r-i}\times  \prod_{i=1}^s \prod_{j=1}^{k_i} \Lambda/(f_i^{\alpha_{i,j}})\ \ \forall \alpha=0,\ldots,r\\
&\bH^1_{(\FF^*)(\ku_1\cdots\ku_r\kv_1\cdots\kv_j)}(K,\bT^*)\approx \prod_{i=1}^s \prod_{j=\beta+1}^{k_i} \Lambda/(f_i^{\alpha_{i,j}})\ \ \forall \beta=0,\ldots,t
\end{aligned}\]
If we define $kn_i$ as the formal square-free product of the first $i$ primes of $(\ku_1,\ldots,\ku_r,\kv_1,\ldots,kv_s)$, we obtain that
\[\Fitt^i(\bH^1_{\FF^*}(K,\bT^*)^\vee)\subset_\f \Fitt^0(\bH^1_{\FF^*(\kn_i)})\]
By Theorem \ref{th:kol_fitting0}, the latter is contained up to finite index in $\Theta_i$, so this proof is concluded.
\end{proof}


