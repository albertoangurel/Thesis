
\documentclass[a4paper,11pt]{book}

\usepackage[usenames]{xcolor} %for the color
\usepackage{amssymb} %maths
\usepackage{mathtools} %maths
\usepackage{amsfonts}
\usepackage{amsthm}
\usepackage{authblk}
\usepackage[utf8]{inputenc} %useful to type directly diacritic characters
\usepackage[british]{babel}

%\usepackage[spanish,es-tabla]{babel} %spanish language
\usepackage{array} %tablas
\usepackage{multirow} %para las tablas
%\selectlanguage{spanish}
\newcommand{\abs}[1]{\lvert#1\rvert}
\newcommand{\norm}[1]{\lVert#1\rVert}
\DeclareMathOperator{\Sel}{Sel}
\DeclareMathOperator{\Gal}{Gal}
\DeclareMathOperator{\cor}{cor}
\DeclareMathOperator{\res}{res}
\DeclareMathOperator{\Ind}{Ind}
\DeclareMathOperator{\Gl2}{GL_2}
\DeclareMathOperator{\im}{\textrm{Im}}
\newcommand{\Frob}{\textrm{Frob}}
\newcommand{\Fitt}{\textrm{Fitt}}
\newcommand{\ord}{\textrm{ord}}
\renewcommand{\char}{\textrm{char}}
\DeclareMathOperator{\Hom}{Hom}
\DeclareMathOperator{\Map}{Map}
\DeclareMathOperator{\Aut}{Aut}
\DeclareMathOperator{\KS}{KS}
\DeclareMathOperator{\ES}{ES}
\newcommand{\loc}{\textrm{loc}}
\newcommand{\m}{\mathfrak{m}}
\newcommand{\p}{\mathfrak{p}}
\newcommand{\du}{^\vee}
\newcommand{\coker}{\textrm{coker}}
\newcommand{\length}{\textrm{length}}
\newcommand{\kk}{^{(k)}}
\newcommand{\dinf}{\delta^{(\infty)}}
\newcommand{\corank}{\textrm{corank}}
\newcommand{\rank}{\textrm{rank}}
\DeclareMathOperator{\cond}{cond}
\newcommand{\Q}{\mathbb Q}
\newcommand{\R}{\mathbb R}
\newcommand{\C}{\mathbb C}
\newcommand{\Z}{\mathbb Z}
\newcommand{\F}{\mathbb F}
\newcommand{\N}{\mathbb N}
\newcommand{\FF}{\mathcal F}
\newcommand{\Fcl}{{\mathcal{F}_{\textrm{cl}}}}
\newcommand{\Fcan}{{\mathcal{F}^{\textrm{can}}}}
\newcommand{\FLambda}{{\mathcal{F}_\Lambda}}
\newcommand{\FBK}{{\mathcal{F}_{\textrm{BK}}}}
\newcommand{\PP}{\mathcal P}
\newcommand{\NN}{\mathcal N}
\newcommand{\DD}{\mathcal D}
\newcommand{\GG}{\mathcal G}
\newcommand{\RR}{\mathcal R}
\newcommand{\OO}{\mathcal O}
\newcommand{\LL}{\mathcal L}
\newcommand{\II}{\mathcal I}
\newcommand{\mupi}{\mu_{p^\infty}}
\newcommand{\mupn}{\mu_{p^N}}
\newcommand{\tr}{\textrm{tr}}
\newcommand{\ur}{\textrm{ur}}
\newcommand{\f}{\textrm{f}}
\newcommand{\s}{\textrm{s}}
\newcommand{\Tr}{\textrm{Tr}}
\newcommand{\fs}{\textrm{fs}}
\renewcommand{\f}{\textrm{f}}
\newcommand{\Iw}{\textrm{Iw}}
\newcommand{\GL}{\textrm{GL}}
\newcommand{\tors}{\textrm{tors}}
\newcommand{\per}{\textrm{per}}
\newcommand{\crys}{\textrm{crys}}
\newcommand{\dR}{\textrm{dR}}
\renewcommand{\sf}{\textrm{sf}}
\newcommand{\chibar}{{\overline{\chi}}}
\newcommand{\psibar}{{\overline{\psi}}}

\usepackage[margin=1.2in]{geometry}
\usepackage{enumerate}


%referencias interactivas
\usepackage{parskip}
\usepackage[all,cmtip]{xy}

\usepackage{tikz}
\usetikzlibrary{matrix,arrows}

    \DeclareFontFamily{U}{wncy}{}
    \DeclareFontShape{U}{wncy}{m}{n}{<->wncyr10}{}
    \DeclareSymbolFont{mcy}{U}{wncy}{m}{n}
\DeclareMathSymbol{\Sha}{\mathord}{mcy}{"58} 


\theoremstyle{definition}
\newtheorem{theorem}{Theorem}[subsection]
\newtheorem{theorem*}{Theorem}[section]
\newtheorem{definition}[theorem]{Definition}
\newtheorem{example}[theorem]{Example}
\newtheorem{remark}[theorem]{Remark}
\newtheorem{remark*}[theorem*]{Remark}
\newtheorem{proposition}[theorem]{Proposition}
\newtheorem{corollary}[theorem]{Corollary}
\newtheorem{lemma}[theorem]{Lemma}
\newtheorem{conjecture}[theorem]{Conjecture}


\makeatletter
\def\namedlabel#1#2{\begingroup
    #2%
    \def\@currentlabel{#2}%
    \phantomsection\label{#1}\endgroup
}
\makeatother


\usepackage[backend=biber, style=alphabetic]{biblatex}
%\bibliographystyle{plain}
\addbibresource{refs.bib}
\usepackage{csquotes}

\usepackage{hyperref} 

\begin{document}


\chapter{Modular symbols}


\section{Neron lattice}

Let 
\[y^2+a_1 xy+a_3 y=x^3+a_2 x^2+a_4 x+a_6\]
be a minimal Weierstrass equation for the elliptic curve. The Néron differential on $E$ is defined as 
\[\omega_E=\frac{dx}{2y+a_1x+a_3} \in \Omega_E\]

$E(\C)$ is a Riemann surface of genus $1$, so its fundamental group
\[\pi^1(E,O)\cong \Z\oplus \Z\]

\begin{definition}
The Néron lattice of $E$ is defined as 
\[\LL:=\Biggl\{\int_{\gamma} \omega_E\Biggm| \gamma\in \pi^1(E,O)\Biggr\}\] 
\end{definition}

\cite{angurel}
%For every $\Q\in E(\C)$, let $\tau_Q$ be the endomorphism of $E$ describing the translation by $Q$, i.e.,
%\[\tau_Q:\ E\to E,\ P\mapsto P+Q\]

%By \cite[proposition III.5.1]{Silverman1}, there exists a differential f

\section{Modular symbols}

Throughout this section, let $E$ be an elliptic curve defined over $\Q$ and let $N$ be its conductor.By the modularity theorem from \cite{Wiles}, there exists a Hecke eigenform $f\in S_k(\Gamma_0(N))$ with Fourier expansion $f=\sum_{n} a_n q^n$ such that for all primes $p$ of good reduction,
\[a_p=(p+1)-\# \widetilde E(\F_p)\]
where $\widetilde E(\F_p)$ is the number of points in the reduced curve at $p$. In this section, $f$ 

\begin{definition}
Let $r$ be a rational number. The modular symbol $\left[\frac{a}{m}\right]$ is defined as 
\[\lambda:=\int_{\infty}^{a/m} f(z)\,dz\]
where the integral is taking along any path in the upper half plane from the cusp at infinity to the cusp at $\frac{a}{m}$.
\end{definition}

\begin{proposition}
The modular symbols of $\frac{a}{m}$ and $\frac{-a}{m}$ are conjugates of each other:
\[\lambda(-r)=\overline{\lambda(r)}\]
\end{proposition}

In order to define the algebraic part of modular symbols, we need to consider first its real and imaginary part:
\[\begin{aligned}
&\lambda^+(r):=\frac{\lambda(r)+\lambda(-r)}{2}=\textrm{Re}\left(\lambda(r)\right)\\
&\lambda^-(r):=\frac{\lambda(r)-\lambda(-r)}{2}=\textrm{Im}\left(\lambda(r)\right)
\end{aligned}\]

The algebraic modular symbols are the above quantities normalized by the Néron periods.
\begin{definition}
Let $\frac{a}{m}$ be a rational number expressed in its simplest terms. The algebraic modular symbols of $\frac{a}{m}$ are:
\[\begin{array}{ccc}
\displaystyle{\left[\frac{a}{m}\right]^+:=\frac{\lambda^+\left(\frac{a}{m}\right)}{\Omega_E^+}};
& &\displaystyle{\left[\frac{a}{m}\right]^-:=\frac{\lambda^-\left(\frac{a}{m}\right)}{\Omega_E^-}}
\end{array}\]
\end{definition}


\end{document}